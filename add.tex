\section{Main Results}
\begin{lemma}\label{addstep}
\begin{enumerate}[label=(\roman*)]
\item\label{ascase} For any $H$ that is AS there exists an $\alpha_{\as}>0$ for which it holds that $\rho(F(\alpha))<1,~\forall \alpha\in (0,\alpha_{as})$.
\item\label{pscase} For any $H$ that is PS there exists an $\alpha_{\ps}>0$ for which it holds that $\E[x^\top(H-\alpha H_t^\top H_t)x|\F_{t-1}]> 0$, $~\forall \alpha\in (0,\alpha_{ps})$.
\end{enumerate}
\end{lemma}
We begin by stating a result related to norms of matrices
\begin{lemma}
Let $A=(A_{ij})\in \R^{n\times n}$ be any real matrix, then $\norm{Ax}^2_2\leq n^2 \norm{x}^2_2 \us{\sup}{i,j}|(A)_{ij}|^2$ and $\ip{x,Ax}\leq n \norm{x}^2_2\us{\sup}{i,j}|(A)_{ij}|$, where $(A)_{ij}$ denotes the $ij^{th}$ entry of matrix $A$.
\end{lemma}
\begin{lemma}
For $\alpha \in (0,\alpha_{as})$ $B_1\eqdef\us{\us{\sup}{i,j}}{t\geq 0}|(F^t)_{ij}|$ and $B_2\eqdef\us{\us{\sup}{i,j}}{t\geq 0}|(F_D^{t\top}F_D^t)_{ij}|$ are finite.
\end{lemma}
\begin{theorem}\label{maintheorem}
The values of $\E[\norm{C\eb_t}]$ as listed in \Cref{maintable} hold.
\end{theorem}
\begin{proof}
See supplementary section.
\end{proof}
\section{Discussion of \Cref{maintheorem}}
We can recurse backwards in \eqref{errprelim} to obtain the following:
\begin{align}\label{errrec}
e_t=F_{t,1}e_0+\alpha\ous{\sum}{k=1}{t} F_{t,k+1}\zeta_k.
\end{align}
Looking at \eqref{errrec} it is clear that quantities $F_{t,j}, j=1,\ldots,t$ influence the dynamics of $e_t$. The results of \Cref{maintheorem} can be interpreted as below.
\begin{itemize}[leftmargin=*]
\item \textbf{Bias vs Variance:} All the entries in the \Cref{maintable} have two terms namely $\B$ and $\V$. The $\B$ is the \emph{Bias} term that captures the rate at which the initial condition $\norm{\theta_t-\ts}$ is forgotten. The $\V$ is the \emph{Variance} term that captures the rate at which noise is filtered. The $\B$ and $\V$ terms capture two different sources of errors. To see this, note that in the absence of noise i.e., when $\zeta_t=0,~\forall t\geq 0$, we have $\E[\norm{C\eb_t}^2]=\B$ and in the presence of noise, i.e., $\zeta\neq 0,\forall t\geq 0$ and perfect initial condition, i.e., $\theta_t=\ts$, we have $\E[\norm{C\eb_t}^2]=\V$.
\item \textbf{Error Recursion:} The expression for $\eb_t$ differs in the additive and the multiplicative noise cases and are given as below.
\begin{enumerate}
\item \textbf{Multiplicative Noise:}
\begin{align}
\eb_t&=\frac{1}{t+1}(\ous{\sum}{i=0}{t} F_{i,1} e_0+ \alpha\ous{\sum}{i=1}{t} \ous{\sum}{k=i}{t} F_{k,i+1}  \zeta_i )
\end{align}
Here recall that $F_{i,j}=(I-\alpha H_i)\ldots (I-\alpha H_j)$ are product of random matrices.
\end{enumerate}
\item \textbf{Additive Noise:}
\begin{align}\label{addexpand}
\eb_t&=\frac{1}{t+1}(\ous{\sum}{i=0}{t} F^i e_0+ \alpha\ous{\sum}{i=1}{t} \ous{\sum}{k=i}{t} F^{k-i}  \zeta_i )
\end{align}
Thus in the additive case involves only products of deterministic matrices since $F_{ij}=F^{i-j},\forall i\geq j$.
In comparison to the additive case, the multiplicative case is harder to analyze.

\item \textbf{AS vs PS vs SPDS:} As observed before, the rates in \Cref{maintheorem} are dependent on the nature of the quantities $F_{i,j}$, which in turn depends on the choice of step size and the spectral properties of $H$. Asking for $H$ to be AS is the weakest condition, followed by PS and the strongest condition is to ask for $H$ to be SPD. The relative strength of the three cases depends on the amount knowledge one has over the eigen values. For example, when $H$ is SPD, we know that it has all real and positive eigen values. Similarly, when $H$ is PS then we can ensure that the operator norm of $I-\alpha H$ i.e., $\norm{I-\alpha H}$ to be less than unity. Finally, when $H$ is only AS we can only ensure that the spectral radius is less than unity. It turns out that the differences in spectral properties impact the analysis and also practice.
\item \textbf{Spectral  Radius vs Operator Norm:} All the results in \Cref{maintable} are expressed for a given step size $\alpha>0$. However, the range of valid $\alpha$ changes from case to case.
\begin{enumerate}
\item \textbf{Additive Case:}In \Cref{addstep}-\ref{ascase}, the range $(0,\alpha_{as})$ ensures that the spectral radius of $F$ is less than unity, which in turn enables us to use $\ous{\sum}{i=1}{t}F^{i}=(I-(I-\alpha H)^t)$. Since $SPDS \Ra PS \Ra AS$, the range $(0,\alpha_{as})$ applies to all the three cases. The range $(0,\alpha_{ps})$ only holds for the PS and SPDS cases. It ensures that the operator norm of $F$ is less than unity.
\item \textbf{Multiplicative Case:} Here we deal only with product of random matrices and cannot directly apply use  $\ous{\sum}{i=1}{t}F^{i}=(I-(I-\alpha H)^t)$, and hence the spectral radius does not appear in the analysis. While the condition for $(0,\alpha_{ps})$ reads the same (see \Cref{addstep}-\ref{pscase}) it is stricter in the case of multiplicative noise, since $\E[H^\top_t H_t|\F_{t-1}]$ will be larger in the case of multiplicative noise.  Further, in the case of multiplicative noise, we have to deal with products of random matrices, i.e., $F_{ij}$ and ensure that they are contraction maps. This translates to the condition that $\E[\norm{F_{ij}}^2]= \E[(I-\alpha H_j)\ldots (I-\alpha H_i)^\top (I-\alpha H_i) \ldots (I-\alpha H_j)]<1$. Using \Cref{genlsa}-\ref{secondmom}  $\E[\norm{F_{ij}}]<1$ can be ensured when
\begin{align}\label{mulcond}
\E[(I-\alpha H_i)^\top (I-\alpha H_i)|\F_{i-1}]<1
\end{align}
Note that \eqref{mulcond} is much more stricter than ensuring that the spectral radius of $F$ is less than unity.
\end{enumerate}
\item \textbf{Role of $C$:} The rates are dependent on the choice of $C$. By choosing $C=H$, we have $H(\eb_t)=H(\tb_t-\ts)=H\tb_t-g$, which is the error in the prediction $H\tb_t$. Choosing $C=I$ measures the error in the deviation of the current estimated parameter $\tb_t$ to the ideal parameter $\ts$. As the \Cref{maintable} reveals, $C=H$ seems to yield better rates because while the deviation in the parameter estimate $\tb_t-\ts$ the predicted output $H(\tb_t-\ts)$ is typically scaled by $H$.
\end{itemize}
\begin{table*}
%\resizebox{0.5\textwidth}{!}{
\begin{tabular}{|c|c|c|c|}\hline
Algorithm& Single-Sample & $\sigma^2$ & Rate \\ \hline
LS-LSA& \ding{53} &$\begin{aligned}(\norm{A}^2+\sigma_1^2) (\sigma_2^2+\sigma_1^2\parallel \ts\parallel^2)\end{aligned}$&  $\begin{aligned} \B^2=\frac{\norm{(A^\top A)^{-1}}}{\alpha^2(t+1)^2},\\\V^2=(\sigma^2)\frac{\norm{(A^\top A)^{-1}}}{t+1}\end{aligned}$\\ \hline
FP-LSA& \ding{51} &$\begin{aligned} (\sigma_2^2+\sigma_1^2\norm{\ts}^2)\end{aligned}$& $\begin{aligned}\B^2=(1+4(\alpha\rho_A)^{-1}) (\alpha\rho_A)^{-1} \Big(\frac{\norm{\theta_0-\ts}^2}{(t+1)^2}\Big),~\\ \V^2=(1+4(\alpha\rho_A)^{-1}) (\alpha\rho_A)^{-1} \Big(\frac{\alpha^2\sigma^2}{t+1} \Big)\end{aligned}$\\ \hline
SP-LSA& \ding{51} &\ding{53}& \ding{53}\\ \hline
\end{tabular}
%}
\end{table*}


\comment{
\subsection{Discussion: Additive Noise Case}
 In the case of additive noise, since $M_t=0$, we have $F_{i,j}=F^{i-j}_D$ and hence
\textbf{Role of Contractions:} In this noise model, we need $F$ to have
\subsection{Discussion: Multiplicative Noise Case}
$\E\ip{H\eb_t,H\eb_t}=\frac{1}{(t+1)^2}\E \ip{\ous{\sum}{i=0}{t} F_{i,1}e_0+\alpha \ous{\sum}{k=1}{t} \ous{\sum}{i=k}{t} F_{i,k+1} \zeta_k,\ous{\sum}{i=0}{t} F_{i,1}e_0+\alpha \ous{\sum}{k=1}{t} \ous{\sum}{i=k}{t} F_{i,k+1} \zeta_k}$
\begin{enumerate}
\item \emph{Type I:} Terms involving $\E<F_{i,1}e_0,F_{i,1}e_0>, i,j=0,\ldots, t$ or $<F_{i,k+1}\zeta_k, F_{j,k+1}\zeta_k>,~ k=1,\ldots,t, i,j=k,\ldots,t$.
\item \emph{Type II:} Terms involving $\E<F_{i,k+1}\zeta_k, F_{j,k+1}\zeta_k>, k=1,\ldots,t, i,j=k,\ldots,t$ or inner products of form $<F_{i,1}e_0, F_{j,1}e_0>, i,j=0,\ldots, t$.
\item \emph{Type III:} Terms involving $\E<F_{i,1}e_0, F_{j,k+1}\zeta_k>, i=0,\ldots,t, 1\leq k \leq t$.
\item \emph{Type IV:} Terms involving $\E<F_{i,k+1}\zeta_{k'}, F_{j,k+1}\zeta_k>, k\neq k'$ and $k',k=1,\ldots,t, i,j=k,\ldots,t$.
\end{enumerate}
\textbf{Role of Contractions:}
The \emph{Type III} and \emph{Type IV} terms are zero due to the independence assumption.
\begin{lemma}
For all $i>j$ and $x\in \R^n$ $\E <F_{j,k+1} x,F_{i,k+1}x>=\E<F_{j,k+1}k,(I-\alpha H)^{i-j} F_{j,k+1}x>$
\end{lemma}
}
