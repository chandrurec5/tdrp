%!TEX root =  lsa.tex
%%%%%%%%%%%%%%%%%%%%%%%%%%%%%%%%%%%%%%%%
\section{Notation and Common Definitions}
\label{sec:not}

The set of reals number is denoted by $\R$.
For a matrix $A$, $A^\top$ denotes its transpose.
We use $A\succeq 0$ to denote that the real-valued \todoc{all matrices will be real-valued?} 
square matrix $A$ is symmetric and positive semidefinite (SPSD):
$A = A^\top$, $\inf_x x^\top A x\ge 0$.
We also use $A \succ 0$ to denote that the real square matrix $A$ is symmetric and positive definite (SPD):
$A = A^\top$, $\lambda_{\min}(A) =\inf_{x:\norm{x}=1} x^\top A x>0$. 
Here, $\norm{x}$ denotes the $2$-norm of vector $x$: $\norm{x} = x^\top x$ and 
$\lambda_{\min}(A)$ denotes the minimum eigenvalue of $A$.
For $A,B$ SPSD matrices, $A\succeq B$ holds if $A-B\succeq 0$.
We also use $A\succ B$ similarly to denote that $A-B \succ 0$.
We also use $\preceq$ and $\prec$ analogously. We will use $\eye$ to denote the identity matrix.
%
A matrix $A$ is called positive definite (PD) if $\inf_{x:\norm{x}=1} x^\top A x >0$.
%
We will also use $\ip{\cdot,\cdot}$ to denote inner products: $\ip{x,y} = x^\top y$.
%
$\E$ denotes mathematical expectation.


