%!TEX root =  lsa.tex
\usepackage[textsize=tiny]{todonotes}
\setlength{\marginparwidth}{13ex}
\newcommand{\todoc}[2][]{\todo[size=\scriptsize,color=blue!20!white,#1]{Csaba: #2}}
\newcommand{\todoch}[2][]{\todo[size=\scriptsize,color=red!20!white,#1]{Chandru: #2}}
\newcommand{\norm}[1]{\left\| #1\right\|}
\newcommand{\ip}[1]{\langle#1\rangle}
\newcommand{\ber}{\begin{enumerate}[label=(\roman*)]}
\newcommand{\eer}{\end{enumerate}}
\newcommand{\rr}{\rho^{\alpha}_{R}}
\newcommand{\rd}{\rho^{\alpha}_{D}}
\newcommand{\ld}{\lambda_{D}}
\newcommand{\V}{\mathcal{V}}
\newcommand{\B}{\mathcal{B}}
\renewcommand{\P}{\mathcal{P}}
\renewcommand{\H}{\mathcal{H}}
\usepackage{hyperref}
\hypersetup{
    bookmarks=true,         % show bookmarks bar?
    unicode=false,          % non-Latin characters in AcrobatÕs bookmarks
    pdftoolbar=true,        % show AcrobatÕs toolbar?
    pdfmenubar=true,        % show AcrobatÕs menu?
    pdffitwindow=false,     % window fit to page when opened
    pdfstartview={FitH},    % fits the width of the page to the window
    pdftitle={My title},    % title
    pdfauthor={Author},     % author
    pdfsubject={Subject},   % subject of the document
    pdfcreator={Creator},   % creator of the document
    pdfproducer={Producer}, % producer of the document
    pdfkeywords={keyword1} {key2} {key3}, % list of keywords
    pdfnewwindow=true,      % links in new window
    colorlinks=true,       % false: boxed links; true: colored links
    linkcolor=red,          % color of internal links (change box color with linkbordercolor)
    citecolor=blue,        % color of links to bibliography
    filecolor=magenta,      % color of file links
    urlcolor=cyan           % color of external links
}
\usepackage{times}
\usepackage{natbib}
\usepackage{nicefrac}
\usepackage{wrapfig}
\usepackage{mathtools}

\let\temp\epsilon
\let\epsilon\varepsilon

\usepackage{diagbox}
\usepackage{makecell}
%\usepackage{pgfplots}
%\usepackage{pgf}
\usepackage{amsmath}
\usepackage{amsthm}
\usepackage{amsfonts}
%\usepackage{comment}
%\renewenvironment{comment}{}{}
\usepackage[capitalize]{cleveref}
\DeclareMathOperator{\supp}{supp}
\usepackage{enumitem}
%\usepackage{tikz}
%\usepackage{placeins}
\newcommand{\nn}{\nonumber}
\newcommand{\onp}{\emph{on-policy~}}
\newcommand{\ofp}{\emph{off-policy~}}
\newcommand{\Ra}{\Rightarrow}
\renewcommand{\L}{\emph{L}}
\newcommand{\T}{\mathcal{T}}
\newcommand{\I}{\mathcal{I}}
\newcommand{\D}{\mathcal{D}}
\newcommand{\R}{\mathbb{R}}
\newcommand{\E}{\mathbf{E}}
\newcommand{\EE}[1]{\mathbf{E}[#1]}
\newcommand{\F}{\mathcal{F}}
\newcommand{\M}{\mathcal{M}}

\newcommand{\ra}{\rightarrow}
\newcommand{\la}{\leftarrow}
\newcommand{\tdo}{TD$(0)$}
\newcommand{\err}{\emph{err}}
\newcommand{\xb}{\bar{x}}
\newcommand{\xs}{x^*}
\newcommand{\tb}{\bar{\theta}}
\newcommand{\ts}{\theta^*}
\newcommand{\ab}{\bar{\alpha}}
\newcommand{\eqdef}{\stackrel{\cdot}{=}}
\newcommand{\eb}{\bar{e}}
\newcommand{\eye}{I}


\theoremstyle{plain}
\newtheorem{theorem}{Theorem}[]
\newtheorem{lemma}{Lemma}[]
\newtheorem{corollary}{Corollary}
\newtheorem{proposition}{Proposition}

\theoremstyle{definition}
\newtheorem{assumption}{Assumption}[]
\newtheorem{example}{Example}
\newtheorem{remark}{Remark}
\newtheorem{domain}{Domain}
\newtheorem{condition}{Condition}

\if0
% Comment by csaba: This screws up inverse search in TeXShop.. I prefer not to have these.
\usetikzlibrary{intersections}
\usetikzlibrary{arrows,calc,fit,patterns,plotmarks,shapes.geometric,shapes.misc,shapes.symbols,   shapes.arrows,   shapes.callouts,   shapes.multipart,   shapes.gates.logic.US,   shapes.gates.logic.IEC,   er,   automata,   backgrounds,   chains,   topaths,   trees,   petri,   mindmap,   matrix,   calendar,   folding, fadings,   through,   positioning,   scopes,   decorations.fractals,   decorations.shapes,   decorations.text,   decorations.pathmorphing,   decorations.pathreplacing,   decorations.footprints,   decorations.markings, shadows}
\usetikzlibrary{arrows,calc,shapes, snakes, intersections,patterns,shadows}
\tikzstyle{decision}=[diamond,draw]
\tikzstyle{line}=[draw]
\tikzstyle{elli}=[draw,ellipse]
\tikzstyle{arrow} = [thick]
\fi
\newcommand{\us}[2]{\underset{#2}{#1}~}
\newcommand{\ous}[3]{\overset{#3}{\underset{#2}{#1}}~}

