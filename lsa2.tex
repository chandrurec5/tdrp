%%%%%%%%%%%%%%%%%%%%%%%%%%%%%%%%%%%%%%%%%%%%%%%%%%%%%%%%%%%%%%%%%%
%%%%%%%% ICML 2016 EXAMPLE LATEX SUBMISSION FILE %%%%%%%%%%%%%%%%%
%%%%%%%%%%%%%%%%%%%%%%%%%%%%%%%%%%%%%%%%%%%%%%%%%%%%%%%%%%%%%%%%%%

% Use the following line _only_ if you're still using LaTeX 2.09.
%\documentstyle[icml2016,epsf,natbib]{article}
% If you rely on Latex2e packages, like most moden people use this:
\documentclass{article}
%!TEX root =  lsa.tex
\usepackage[textsize=tiny]{todonotes}
\setlength{\marginparwidth}{13ex}
\newcommand{\todoc}[2][]{\todo[size=\scriptsize,color=blue!20!white,#1]{Csaba: #2}}
\newcommand{\todoch}[2][]{\todo[size=\scriptsize,color=red!20!white,#1]{Chandru: #2}}
\newcommand{\todox}[2][]{\todo[size=\scriptsize,color=green!20!white,#1]{Xiaowei: #2}}
\newcommand{\norm}[1]{\left\| #1\right\|}
\newcommand{\ip}[1]{\langle#1\rangle}
\newcommand{\ber}{\begin{enumerate}[label=(\roman*)]}
\newcommand{\eer}{\end{enumerate}}
\newcommand{\rr}{\rho^{\alpha}_{R}}
\newcommand{\rd}{\rho^{\alpha}_{D}}
\newcommand{\ld}{\lambda_{D}}
\newcommand{\V}{\mathcal{V}}
\newcommand{\B}{\mathcal{B}}
\renewcommand{\P}{\mathcal{P}}
\renewcommand{\H}{\mathcal{H}}
\usepackage{hyperref}
\hypersetup{
    bookmarks=true,         % show bookmarks bar?
    unicode=false,          % non-Latin characters in AcrobatÕs bookmarks
    pdftoolbar=true,        % show AcrobatÕs toolbar?
    pdfmenubar=true,        % show AcrobatÕs menu?
    pdffitwindow=false,     % window fit to page when opened
    pdfstartview={FitH},    % fits the width of the page to the window
    pdftitle={My title},    % title
    pdfauthor={Author},     % author
    pdfsubject={Subject},   % subject of the document
    pdfcreator={Creator},   % creator of the document
    pdfproducer={Producer}, % producer of the document
    pdfkeywords={keyword1} {key2} {key3}, % list of keywords
    pdfnewwindow=true,      % links in new window
    colorlinks=true,       % false: boxed links; true: colored links
    linkcolor=red,          % color of internal links (change box color with linkbordercolor)
    citecolor=blue,        % color of links to bibliography
    filecolor=magenta,      % color of file links
    urlcolor=cyan           % color of external links
}
\usepackage{times}
\usepackage{natbib}
\usepackage{nicefrac}
\usepackage{wrapfig}
\usepackage{mathtools}

\let\temp\epsilon
\let\epsilon\varepsilon

\usepackage{diagbox}
\usepackage{makecell}
%\usepackage{pgfplots}
%\usepackage{pgf}
\usepackage{amsmath}
\usepackage{amsthm}
\usepackage{amsfonts}
%\usepackage{comment}
%\renewenvironment{comment}{}{}
\usepackage[capitalize]{cleveref}
\usepackage{enumitem}
%\usepackage{tikz}
%\usepackage{placeins}
\newcommand{\nn}{\nonumber}
\newcommand{\onp}{\emph{on-policy~}}
\newcommand{\ofp}{\emph{off-policy~}}
\newcommand{\Ra}{\Rightarrow}
\renewcommand{\L}{\emph{L}}
\newcommand{\T}{\mathcal{T}}
\newcommand{\I}{\mathcal{I}}
\newcommand{\D}{\mathcal{D}}
\newcommand{\R}{\mathbb{R}}
\newcommand{\E}{\mathbf{E}}
\newcommand{\F}{\mathcal{F}}
\newcommand{\M}{\mathcal{M}}

\newcommand{\ra}{\rightarrow}
\newcommand{\la}{\leftarrow}
\newcommand{\tdo}{TD$(0)$}
\newcommand{\err}{\emph{err}}
\newcommand{\xb}{\bar{x}}
\newcommand{\xs}{x^*}
\newcommand{\tb}{\bar{\theta}}
\newcommand{\ts}{\theta^*}
\newcommand{\ab}{\bar{\alpha}}
\newcommand{\eqdef}{\stackrel{\cdot}{=}}
\newcommand{\eb}{\bar{e}}
\theoremstyle{plain}
\newtheorem{theorem}{Theorem}[]
\newtheorem{lemma}{Lemma}[]
\newtheorem{corollary}{Corollary}

\theoremstyle{definition}
\newtheorem{assumption}{Assumption}[]
\newtheorem{example}{Example}
\newtheorem{remark}{Remark}
\newtheorem{domain}{Domain}
\newtheorem{condition}{Condition}

\if0
% Comment by csaba: This screws up inverse search in TeXShop.. I prefer not to have these.
\usetikzlibrary{intersections}
\usetikzlibrary{arrows,calc,fit,patterns,plotmarks,shapes.geometric,shapes.misc,shapes.symbols,   shapes.arrows,   shapes.callouts,   shapes.multipart,   shapes.gates.logic.US,   shapes.gates.logic.IEC,   er,   automata,   backgrounds,   chains,   topaths,   trees,   petri,   mindmap,   matrix,   calendar,   folding, fadings,   through,   positioning,   scopes,   decorations.fractals,   decorations.shapes,   decorations.text,   decorations.pathmorphing,   decorations.pathreplacing,   decorations.footprints,   decorations.markings, shadows}
\usetikzlibrary{arrows,calc,shapes, snakes, intersections,patterns,shadows}
\tikzstyle{decision}=[diamond,draw]
\tikzstyle{line}=[draw]
\tikzstyle{elli}=[draw,ellipse]
\tikzstyle{arrow} = [thick]
\fi
\newcommand{\us}[2]{\underset{#2}{#1}~}
\newcommand{\ous}[3]{\overset{#3}{\underset{#2}{#1}}~}


% use Times
\usepackage{times}
% For figures
\usepackage{graphicx} % more modern
%\usepackage{epsfig} % less modern
\usepackage{subfigure} 

% For citations
\usepackage{natbib}

% For algorithms
\usepackage{algorithm}
\usepackage{algorithmic}

% As of 2011, we use the hyperref package to produce hyperlinks in the
% resulting PDF.  If this breaks your system, please commend out the
% following usepackage line and replace \usepackage{icml2016} with
% \usepackage[nohyperref]{icml2016} above.
\usepackage{hyperref}

% Packages hyperref and algorithmic misbehave sometimes.  We can fix
% this with the following command.
%\newcommand{\theHalgorithm}{\arabic{algorithm}}

% Employ the following version of the ``usepackage'' statement for
% submitting the draft version of the paper for review.  This will set
% the note in the first column to ``Under review.  Do not distribute.''
\usepackage{icml2016} 

% Employ this version of the ``usepackage'' statement after the paper has
% been accepted, when creating the final version.  This will set the
% note in the first column to ``Proceedings of the...''
%\usepackage[accepted]{icml2016}


% The \icmltitle you define below is probably too long as a header.
% Therefore, a short form for the running title is supplied here:
\icmltitlerunning{Iterate averaging}

\begin{document} 

\twocolumn[
\icmltitle{Iterate Averaging in Linear Stochastic Approximation}

%\icmlauthor{Your Name}{email@yourdomain.edu}
%\icmladdress{Your Fantastic Institute,314159 Pi St., Palo Alto, CA 94306 USA}
%\icmlauthor{Your CoAuthor's Name}{email@coauthordomain.edu}
%\icmladdress{Their Fantastic Institute,27182 Exp St., Toronto, ON M6H 2T1 CANADA}

% You may provide any keywords that you 
% find helpful for describing your paper; these are used to populate 
% the "keywords" metadata in the PDF but will not be shown in the document
\icmlkeywords{
stochastic approximation,
iterate averaging,
Polyak-Rupert,
linear stochastic approximation,
linear stochastic recursions,
finite-time expected error bounds
}

\vskip 0.3in
]
%%!TEX root =  lsa.tex
\section{Introduction}
%%%%%%%%%%%%%%%%%%%%%%%%%%%%%%%%%%%%%%%%
We consider the setting of constant stepsize linear stochastic approximation (LSA) algorithms with multiplicative $i.i.d$ noise and iterate averaging for finding the solution of linear systems based on noisy data. \todoc{Add blah blah about stochastic approximation: Citations, why care, ..}
The setting arises in applications in science and engineering, where the aim is to handle large volumes of high-dimensional data linear-in-the-dimension computation per time step. Two example applications which will use
as running examples are linear prediction with quadratic loss and using linear value functions to predict total reward 
of some fixed policy in reinforcement learning. \todoc{Add refs}
Recently, \citet{bachharder} achieved remarkable results in analyzing the expected loss of 
that arise in the solving linear least-squares prediction with squared loss and $i.i.d$ sampling.
According to this result, there exists a constant step-size such that for any problem where the noisy data is bounded by a known constant, the expected squared prediction error after $t$ updates is at most $\frac{C}{t}$ with a constant $C>0$ that depends \emph{only} on the bound on the data. \todoc{Dimension?} 
In particular, as opposed to many earlier results available in the literature \todoc{Add citations}, $C$ does \emph{not} depend on the conditioning of the underlying system, while the total runtime of the algorithm scales linearly with both $t$ and $n$. 
\todoc{Comment on: Was this known before for $O(n^2)$ or higher complexity algorithms?}

In this paper, we ask whether these ``remarkable" properties hold in our setting of general class of LSAs (other than the linear least-squares prediction setting). We are especially interested in linear value function approximation in reinforcement learning using temporal difference learning (TD) either from experience replay in a batch setting or solving linear systems using TD-style algorithms. \todoc{TD-style, or just TD?} \todoc{Later we need to comment on TD with eligibility traces and other variations of TD, e.g., new variants.}

Our proof technique for \cref{maintheorem} has been adopted from Appendix $B$ of \citet{bachharder}. \todoc{I believe it goes way back to the first proofs for stochastic approximation.} 
However, there are some critical differences which are listed below. \todoc{What is the difference to their earlier paper? We should also compare ourselves to that.}
\begin{itemize}
\item In the linear least-squares problem is considered by \citet{bachharder} and the random matrices involved are known to be symmetric positive definite. This enables \citet{bachharder} to define operators that act between spaces of \emph{symmetric} matrices and carry out the computations making use of such operators. In our case, the matrices are known only to be (laxly) positive definite, i.e., they lack symmetry. \todoc{We should point out after our analysis in which step the symmetry helps \citet{bachharder}. We should also make this somehow clear here if possible.}
Thus, the above approach does not work, and instead we have resort to an analysis that makes use of only the expected norms of the random matrices involved.
\item Another significant difference is that in linear least-squares, the ``noise'' has a favorable ``structure''. Again, due to the lack of symmetry, this favorable structure does not apply. \todoc{Again, after the analysis, we should point out how the structured noise would have helped. I am wondering whether a more graceful degradation should happen? Is this really all or nothing?}
\end{itemize}


%%%%%%%%%%%%%%%%%%%%%%%%%%%%%%%%%%%%%%%%
\section{Introduction}
%%%%%%%%%%%%%%%%%%%%%%%%%%%%%%%%%%%%%%%%
What is the problem setup:
Iterate averaging in linear stochastic approximation with multiplicative i.i.d. noise.

Why should we care?
SA has wide range of applications in science and engineering. Low-cost alternative,
highly suitable to process large data volumes and work in large dimensions.

Motivation: Linear prediction with squared loss and iid sampling. Bach et al. obtained remarkable result:
there exists a constant step-size 
such that expected squared prediction error
after $n$ updates is at most $C/n$ with a universal constant $C>0$, 
uniformly over all problem instances
such that the magnitude of features vectors is bounded by a known constant.
Why remarkable? No dependence on the conditioning of the underlying system.

Constant stepsize helps intuitively because $\dots$ (add explanation).

Question asked: To what extent can this remarkable result generalized beyond linear prediction 
with squared loss?
One potential application domain is
linear value function approximation in reinforcement learning using temporal difference (TD) learning.
Either experience replay in a batch setting, or solving linear systems using TD-style algorithms.

%%%%%%%%%%%%%%%%%%%%%%%%%%%%%%%%%%%%%%%%
\section{Problem Setup and Result}
%%%%%%%%%%%%%%%%%%%%%%%%%%%%%%%%%%%%%%%%
Linear stochastic approximation.

Define only those quantities that are needed to state the main result.

Example: TD(0) learning (don't get blogged down on explaining where the equations are coming from,
just state them, state that they are a special case of ours).

Main result: Expected loss bound.

Theorem: Blah blah
\todoc[inline]{
Can we do the calculation for the structured and unstructured noise together? 
What should happen is that the condition on the stepsize should have two
parts: One part, that depends on the condition number should disappear
when the structured noise condition is met.
%
Also, can we bound the error in $C$-norm for general positive definite $C$?
}

Proof will follow in \cref{sec:proof}.
Make comment on how and why we depart from the analysis of Bach (add citation).


%%%%%%%%%%%%%%%%%%%%%%%%%%%%%%%%%%%%%%%%
\section{Discussion}
%%%%%%%%%%%%%%%%%%%%%%%%%%%%%%%%%%%%%%%%
Goal: Discuss result, implications, limitations.

Say the good things about the result: Fast rate, small constants.
Better than using decreasing step-size?

We could not eliminate the dependence on the condition number:
The result only holds when the stepsize is smaller than a problem-dependent quantity.

Discuss where this is coming from and why this is hard to avoid.

Our recommendation: Keep track of error; if error is blowing up (larger than a preset value), decrease the stepsize
multiplicatively, reset the parameter vector. After $\log(1/\alpha_0)$ resets, each taking at most BLAH time(?),
the algorithm finds $\alpha<\alpha_0$. Good thing: The dependence on $1/\alpha_0$ is mild.

We do not know whether the condition we have is really necessary,
or a uniform bound with a universal step-size is possible.
Discuss proof attempt: Switching stable linear systems.
No known necessary and sufficient condition for the multiplicative ergodic theorem (cite source).
The existing necessary conditions are too stringent (example).

Why i.i.d. noise? 
Extension to martingale noise.
Extension to Markov noise.

What happens under additive noise?

\section{Related Work}
We place our result into the context of related results.
Compare with papers by Bach. Can our result reproduce theirs?
If not why not? 

Compare with Prashanth and Korda:
Only work known to us to bound finite-time error of LSA,
specifically for TD (special case of our setup,
as far as the form of the updates are concerned).

They use decreasing stepsizes -- they must use decreasing stepsizes
as discussed because they assume Markov noise.
Show their bound here (simplified maybe).
Comment on the main qualities of the bound.
Do they get the same rate?
Do they need the same conditions?
Do they depend on the same constants? No.. reciproc of minimum eigenvalue creeps in.
Why is this a problem? Reciproc of minimum eigenvalue is at least $d$ -- bad for huge-dimensional systems.
But it could be much smaller, too, when system is ill-conditioned.

%%%%%%%%%%%%%%%%%%%%%%%%%%%%%%%%%%%%%%%%
\section{Proof}
%%%%%%%%%%%%%%%%%%%%%%%%%%%%%%%%%%%%%%%%
\label{sec:proof}
%%%%%%%%%%%%%%%%%%%%%%%%%%%%%%%%%%%%%%%%
\section{Conclusions}
Iterate averaging: Extremely powerful in linear prediction under quadratic noise.
Here, power is tested more generally.
BLAH-BLAH

Further work:
TD($\lambda$) -- do the results extend? Why not? What can be done?

GTD -- do the results extend? Why not? What can be done?
%%%%%%%%%%%%%%%%%%%%%%%%%%%%%%%%%%%%%%%%
\if0
\section{Problem Setup}
A Linear Stochastic Approximation (LSA) algorithm is given by
\begin{align}\label{lsa}
\theta_{t+1}=\theta_t+\alpha_t(L_t(\theta)),
\end{align}
where $\{\alpha_t>0,t\geq 0\}$ is a sequence of step-sizes and $L_t\in \R^n$ is a noisy sample of some function $L(\theta)$. Without loss of generality one can assume that $L_t(theta)=g_t-H_t\theta$, where $g_t\in\R^n$ and $H_t\in \R^n$ are noisy samples of some $g\in \R^n$ and $H\in \R^n$ (note that $L=g-H\theta$). We wish to study the convergence of $\theta_t$ to a $\ts\in\R^n$ such that the expected update $\E[L_t(\ts)]=L(\ts)=0$. Depending on the application, such $\ts$ can either be a fixed point or extrememum. In order to understand the error $e_t\eqdef\theta_t-\ts$, we can look at is dynamics as follows:
\begin{align}\label{lser}
\theta_{t+1}-\ts&=\theta_t-\ts+\alpha_t(g_t- H_t (\theta_t+\ts-\ts)),\nn\\
e_{t+1}&=(I-\alpha_t)e_t+\alpha_t M_{t+1},
\end{align}
where $M_{t+1} is$ some noise function. We call the recursion in \eqref{lser} as a linear stochastic error recursion (LSER) and we now proceed to study the same under the following assumptions.
\begin{assumption}\label{fpassump}
\begin{enumerate}[leftmargin=*]
\item $\alpha_t=\alpha>0,~\forall t\geq 0$.
\item $H_t=H+N_{t+1}$ such that $\{N_t\}$ and $\{M_t\}$ are martingale difference sequences measurable with respect to an increasing family of $\sigma$-fields $\mathcal{F}_{t}\stackrel{\cdot}{=}\sigma(x_0,M_1,N_1\ldots,N_t,M_t),t\geq 0$, such that $\E[N_{t+1}|\F_t]=\E[M_{t+1}|\F_t]=0$, $\E[\parallel M_{t+1} \parallel^2|\F_t]\leq \sigma^2, \E[\parallel N_{t+1} \parallel^2|\F_t]\leq \sigma^2, ~\forall t\geq 0$ for some variance $\sigma>0$.
\end{enumerate}
\end{assumption}
Notice that once a constant step-size is chosen the noise term in \eqref{lser} is $\alpha M_{t+1}$, i.e., there is constant addition of noise each iteration, a scenario under which we cannot hope the iterates to converge to $\ts$. However, we can look at the behavior of the average of the iterates, formally known as Rupert-Polyak averaging defined below.
\begin{align}\label{erp}
\begin{split}
\tb_{t}=\frac{1}{t}\overset{t-1}{\underset{s=0}{\sum}}\theta_t
\bar{e}_{t}=\frac{1}{t}\overset{t-1}{\underset{s=0}{\sum}}e_t
\end{split}
\end{align}
We are interested in the mean squared error given by
\begin{align}\label{errfun}
\err(t)&= \E[\parallel \eb_t\parallel^2]
%&=\B(t)+\V(t),
\end{align}
\subsection{Error Analysis - Bias, Variance and Product of Random Matrices}
We now expand $\err(t)$ to understand the structure of LSERs. For the purpose of our analysis, we define the product of random matrices $F_{j,i}\eqdef\Pi_{t=i}^{j} (I-\alpha H_t), \forall j\geq i$ and $F_{j,i}\eqdef I,~\forall i<j$.
\begin{align}
<\eb_t,\eb_t>=\frac{1}{t^2}\E <\overset{t-1}{\underset{i=0}{\sum}}e_i ,\overset{t-1}{\underset{i=0}{\sum}} e_i>,
\end{align}
Then we have from \eqref{lser} that for any $j>i$,
\begin{align}
e_{j+1}=F_{j,i}e_i + \alpha\sum_{k=i}^j F_{j,k+1}M_{k+1}
\end{align} and hence notice that
\begin{align}
\overset{t-1}{\underset{i=0}{\sum}}e_i=\overset{t-1}{\underset{i=0}{\sum}} F_{i,0}e_0+\alpha \overset{t-1}{\underset{i=0}{\sum}} \overset{i}{\underset{k=0}{\sum}} F_{i,k+1}M_{k+1}\end{align} Thus the error can be naturally split into two terms namely $\err(t)=\B(t)+\V(t)$, where $\B(t)$ is the \emph{bias} term which satisfies $\err(t)=\B(t)$ when $M_{t+1}=0~\forall t\geq 0$ and $e_0\neq 0$,  and, $\V(t)$ is the variance term which satisfies $\err(t)=\V(t)$ when $e_0=0$ and $M_{t+1}\neq 0$.
%Now, for any $j>i$ we have $<e_i,e_j>=<e_i,F_{j,i}e_i+\sum_{k=i}^j F_{j,k+1}M_{k+1}>$
\begin{lemma}
For any $x_i\in \R^n$ that is $\F_i$ measurable and $\forall ~j \geq k> i$ it follows that $\E[x_i^\top F_{j,k+1}M_{k+1}]=0$
\end{lemma}
\begin{lemma}
$\E <e_i,F_{j,i}e_i>=(I-\alpha H)^{j-i}e_i$
\end{lemma}
\begin{remark}
$<e_j,e_j>=<F_{j,i}e_i+\sum_{k=i}^j F_{j,k+1}M_{k+1},F_{j,i}e_i+\sum_{k=i}^j F_{j,k+1}M_{k+1}>$, which involves expectation of product of random matrices of the form
$
\E[(I-\alpha H_i)^\top\ldots (I-\alpha H_j)^\top (I-\alpha H_j)\ldots(I-\alpha H_i)]
$.
\end{remark}
\begin{assumption}
Let $\rho_{s}(\alpha)\eqdef \underset{{x\in \R^n}}{\sup}x^\top (2H-\alpha E[H_t^\top H_t])x $ and $\rho_d\eqdef\parallel H \parallel$ (where $\parallel \cdot\parallel$ is the operator norm of the matrix). Then we assume that exist an $\alpha_{\max}$ such that $~\forall \alpha\in (0,\alpha_{\max})$
\begin{enumerate}[leftmargin=*]
\item $2H- \alpha E[H_t^\top H_t] >0$.
\item $|1-\rho_s(\alpha)|<1$.
\item $|1-\alpha\rho_d|<1$.
\end{enumerate}
\end{assumption}
\begin{theorem}
For $\alpha\in(0,\alpha_{\max})$ and $H$ non-symmetric
\begin{align}
\err(t)\leq &\frac{(\rho_s\alpha)^{-1}(1+2(\rho_d\alpha)^{-1})}{t^2}\nn\\&+\frac{\alpha^2(\alpha\rho_s)^{-1}(1+2(\rho_d\alpha)^{-1})}{t}
\end{align}
\end{theorem}
\comment{\begin{theorem}\label{sym}
For $\alpha\in(0,\alpha_{\max})$ and $H$ symmetric
\begin{align}
\err(t)\leq &\frac{(\rho_s\alpha)^{-1}(2(\rho_d\alpha)^{-1})}{t^2}\nn\\&+\frac{\alpha^2(\alpha\rho_s)^{-1}(2(\rho_d\alpha)^{-1})}{t}
\end{align}
\end{theorem}
}
\begin{proof}
\begin{align*}
&\E<\overset{t-1}{\underset{i=0}{\sum}}e_i ,\overset{t-1}{\underset{i=0}{\sum}} e_i>\nn\\
=&\E < \overset{t-1}{\underset{i=0}{\sum}} F_{i,0}e_0+\alpha \overset{t-1}{\underset{i=0}{\sum}} \overset{i}{\underset{k=0}{\sum}} F_{i,k+1}M_{k+1},\nn\\ &\overset{t-1}{\underset{i=0}{\sum}} F_{i,0}e_0+\alpha \overset{t-1}{\underset{i=0}{\sum}} \overset{i}{\underset{k=0}{\sum}} F_{i,k+1}M_{k+1} >
\end{align*}
Bias term $\E < \overset{t-1}{\underset{i=0}{\sum}} F_{i,0}e_0, \overset{t-1}{\underset{i=0}{\sum}} F_{i,0}e_0>$
\begin{align*}
=\E\overset{t-1}{\underset{i=0}{\sum}}<F_{i,0}e_0,F_{i,0}>+2 \E\overset{t-2}{\underset{i=0}{\sum}}\overset{t-1}{\underset{j\geq i}{\sum}}<F_{i,0}e_0, F_{j,0}e_0>
\end{align*}
Now we know that for $j>i$ $\E<F_{i,0}e_0, F_{j,0}e_0>\leq (1-\alpha \rho_d)^{j-i}\E<F_{i,0}e_0,F_{i,0}e_0>$. Thus
\begin{align*}
&\E\overset{t-2}{\underset{i=0}{\sum}}\overset{t-1}{\underset{j\geq i}{\sum}}<F_{i,0}e_0, F_{j,0}e_0>\nn\\
&\leq \E\overset{t-2}{\underset{i=0}{\sum}}\overset{\infty}{\underset{j\geq i}{\sum}} (1-\alpha \rho_d)^{j-i} <F_{i,0}e_0, F_{i,0}e_0>\nn\\
&\leq (\alpha\rho_d)^{-1} 2 \E<F_{i,0}e_0, F_{i,0}e_0>
\end{align*}
Now
\begin{align*}
&\E\overset{t-1}{\underset{i=0}{\sum}}<F_{i,0}e_0,F_{i,0}e_0>\nn\\
&\leq\E\overset{\infty}{\underset{i=0}{\sum}}<F_{i,0}e_0,F_{i,0}e_0>\nn\\
&\leq\E\overset{\infty}{\underset{i=0}{\sum}}(1-\rho_s(\alpha))^i <e_0,e_0>\nn\\
&\leq \rho_s(\alpha)^{-1} \parallel e_0\parallel^2\nn
\end{align*}


%\begin{align*}
%\E < \overset{t-1}{\underset{i=0}{\sum}} F_{i,0}e_0, \overset{t-1}{\underset{i=0}{\sum}} F_{i,0}e_0>
%\end{align*}
\comment{The proof is on similar lines of \cite{}
$
\E<\overset{t-1}{\underset{i=0}{\sum}}e_i ,\overset{t-1}{\underset{i=0}{\sum}} e_i>=\E\overset{t-1}{\underset{i=0}{\sum}} <e_i,e_i>+ 2\E\overset{t-2}{\underset{i=0}{\sum}}\overset{t-1}{\underset{j\geqi}{\sum}} <e_i,e_{j+1}>
$. Now,

\begin{align*}
\E\overset{t-2}{\underset{i=0}{\sum}}\overset{t-1}{\underset{j\geq i}{\sum}} <e_i,e_{j+1}>\nn\\
=\E\overset{t-2}{\underset{i=0}{\sum}}\overset{t-1}{\underset{j\geq i}{\sum}} <e_i,F_{j,i}e_i + \alpha\sum_{k=i}^{j} F_{j,k+1}M_{k+1}>\nn\\
=\E\overset{t-2}{\underset{i=0}{\sum}}\overset{t-1}{\underset{j\geq i}{\sum}} <e_i,(I-\alpha H)^{j-i}e_i >\nn\\
=\E\overset{t-2} <e_i,\alpha^{-1}H^{-1}[(I-\alpha H)-(I-\alpha H)^{t-i}e_i >\nn\\
=\E\overset{t-2} (\alpha H)^{-1}<e_i,e_i > - <e_i,e_i >- <e_i,\alpha^{-1}H^{-1}[(I-\alpha H)-(I-\alpha H)^{t-i}e_i >\nn\\
\end{align*}
}
\end{proof}
\section{LSAs to solve linear system of equations}
In what follows, we assume that we have access to the stochastic orcales which provide $i.i.d$ samples of $A$ and $b$. We now look at the three different LSAs to solve \eqref{linsys}.
\subsection{Fixed Point Linear Stochastic Approximation (FP-LSA)}
We call the following stochastic recursion to compute $\theta^*$ a FP-LSA algorithm
\begin{align}\label{fplsa}
\theta_{t+1}=\theta_t+\alpha(b_t-A_t),
\end{align}
where we assume that
\begin{assumption}
\begin{enumerate}[leftmargin=*]
\item $b_t$ and $A_t$ are $i.i.d$ samples of $b$ and $A$
\item \label{pd}$\underset{x\in\R^n}{\inf} x^\top A x>0$.
\end{enumerate}
\end{assumption}
Notice that
We now present the LSER corresponding the FP-LSA in \eqref{fplsa}.
\begin{align}\label{fplser}
e_{t+1}=(I-\alpha A_t)e_t+\alpha(M^{(2)}_{t+1}-M^{(1)}_{t+1}\ts),
\end{align}
where $M^{(1)}_{t+1}=A_t-A$, $M^{(2)_{t+1}}=b_t-b$,  and $\ts=A^{-1}b$. Now picking $\sigma=\sigma_1^2+\sigma^2_2 \parallel\ts\parallel^2$ it follows that $\E[\parallel M_{t+1} \parallel^2|\F_t]<\sigma^2$ results of \Cref{nonsym} applies to the LSER in \eqref{fplser}.
\comment{The error term splits into $\err(t)= \B(t)+\V(t)$, (we have dropped $\Sigma$ for the sake of brevity) where $\B(t)$ is the bias term which satisfies $\err(t)=\B(t)$ when $M^{(i)}_{t+1}=0,i=1,2,~\forall t\geq 0$ and $x_0\neq x^*$ and $\V(t)$ is the variance term which satisfies $\err(t)=\V(t)$ when $x_0=x^*$ and $M^{(i)}_{t+1}\neq 0$.}
\subsection{Least-Squares Linear Stochastic Approximation (LS-LSA)}
In general \Cref{pd} need not hold and we would still wish to solve \eqref{linsys}. One way to ensure \Cref{pd} is to ensure that the system is positive definite which can be achieved by solving for $A^TA\theta^*=A^\top b$. We call the following stochastic recursion to compute $\theta^*$ a LS-LSA algorithm
\begin{align}\label{lslsa}
\theta_{t+1}=\theta_t+A^{(1)\top}_t(b_t-A^{(2)}_t\theta_t),
\end{align}
where we assume
\begin{assumption}\label{lsassump}
$b_t$, $A^{(1)}_t$and $A^{(2)}_t$ are $i.i.d$ samples of $b$ and $A$.
\end{assumption}
The LSA in \eqref{lslsa} is a stochastic gradient since the $\nabla_\theta \parallel b-A\theta\parallel^2=A^\top(b-A\theta)$. The LSER corresponding to LS-LSA in \eqref{lslsa} is given by
\begin{align}
e_{t+1}=&(I-\alpha A^{(1)\top}_tA^{(2)}_t)e_t+\alpha(A^\top M^{(3)}_{t+1}+M^{(1)\top}_{t+1}M^{(3)}_{t+1}\nn\\&-A^\top M^{(2)}_{t+1}\ts-M^{(1)\top}_{t+1}M^{(2)}_{t+1}\ts),
\end{align}
where $M^{(1)}_{t+1}=A^{(1)}_t-A$, $M^{(2)}_{t+1}=A^{(2)}_t-A$ and $M^{(3)}_{t+1}=b_t-b$. Choosing $\sigma=\parallel A\parallel^2 (\sigma_2^2+\sigma_1^2\parallel \ts\parallel^2)+ \sigma_1^4(1+\parallel \ts\parallel^2)$ the results of \Cref{sym} applies to the LSER in \eqref{lslser}.

\subsection{Saddle-Point Linear Stochastic Approximation (SP-LSA)}
An important downside of \Cref{lslsa} is that two independent samples of $A$ are available in the form of $A^{(1)\top}_t$ and $A^{(2)}_t$ which might be possible in certain applications. This issue can be alleviated by looking at the following Saddle-Point objective function
\begin{align}
L(\theta,y)=\min_{\theta\in \R^n}\max{y\in \R^n}<b-A\theta,y>-\frac{1}{2}\parallel y\parallel_M ^2
\end{align}
By noticing the fact that $\nabla_\theta L =-A^\top y$ and $\nabla_{y}L=-A\theta-M$, we now specify the SP-LSA algorithm as follows:
\begin{align}\label{splsa}
\begin{split}
y_{t+1}&=y_t+\alpha (b_t-A_t\theta_t- M y_t)\\
\theta_{t+1}&=\theta_t+\alpha(A_t^\top)y_t
\end{split}
\end{align}
The SLER corresponding to the SP-LSA in \eqref{splsa} is given by
\begin{align}
e_{t+1}=\Big(\begin{bmatrix} I & 0 \\ 0 &I\\\end{bmatrix} -\begin{bmatrix} M & A_t \\ -A_t^\top &0\\\end{bmatrix}\Big)e_t+\Big(\begin{bmatrix} M^{(2)}_{t_1}-M^{(1)}_{t+1}\ts  \\ 0\\\end{bmatrix}\Big),
\end{align}
where $M^{(1)}_{t+1}=A_t-A$, $M^{(2)_{t+1}}=b_t-b$,  and $\ts=A^{-1}b$.
Let $x_t=(y_t,\theta_t)$, $g_t=\begin{bmatrix} b_t\\ 0\\\end{bmatrix}$ and $G^{(i)}_t=\begin{bmatrix} M & A^{(i)}_t \\ -A^{(i)\top}_t &0\\\end{bmatrix}$ for $i=1,2$, then consider the following updates
\begin{align}
x'_{t}&=x_t+\alpha (g_t-G^{(1)}_t x_t)\\
x_{t+1}&=x_t+\alpha(g_t- G^{(2)}_t x'_t);
\end{align}
%\subsection{Related Work}

%\subsection{Maximum Allowable step size}

\subsection{Additive Noise}
In the case of additive noise (AN), we have $F_{i,j}= (I-\alpha H)^{i-j+1},~\forall i\geq j$ to be deterministic. In what follows we resort to abuse of notation by letting $F_D=F_{0,0}=I-\alpha H$. The convergence of \eqref{lsa} in the AN case then depends on the spectral properties of $F$, which in turn depends on the choice of $\alpha$ for a given $H$. We are interested in the following two quantities
\begin{enumerate}
\item The \emph{spectral radius} $\rho_s(F_D)$, where $\rho(A)\eqdef\max\{|\lambda_1|,\ldots,|\lambda_k|,k\leq n\}$ with $|\lambda_i|$ denoting the absolute value of the Eigen value $\lambda_i$ of any given matrix $A$.
\item The \emph{operator norm} of $\norm{F_D}_{op}$, where $\norm{A}_{op}\eqdef\us{\sup}{x^\top x\leq 1}\sqrt{x^\top A^\top A x}$ is the operator norm of any given matrix $A$.
\end{enumerate}
\begin{lemma}\label{addstep}
For any $H$ that is AS there exists an $\alpha_{as}>0$ for which it holds that $\rho_{s}(F_D)<1,~\forall \alpha\in (0,\alpha_{as})$.
\end{lemma}
\begin{lemma}\label{addstep}
For any $H$ that is PS there exists an $\alpha_{ps}>0$ for which it holds that $\norm{F_D}_{op}<1$, $~\forall \alpha\in (0,\alpha_{ps})$.
\end{lemma}
We begin by stating a result related to norms of matrices
\begin{lemma}
Let $A\in \R^{n\times n}$ be any real matrix then $\norm{Ax}^2_2\leq \us{\sup}{i,j}n^2|(A)_{ij}|^2\norm{x}^2_2$ and $<x,Ax>\leq \us{\sup}{i,j}n|(A)_{ij}|\norm{x}^2_2$, where $(A)_{ij}$ denotes the $ij^{th}$ entry of matrix $A$.
\end{lemma}
\begin{lemma}
For $\alpha \in (0,\alpha_{as})$ there exists constants $B_1>0$ and $B_2>0$ such that $B_1=\us{\us{\sup}{i,j}}{t\geq 0}|(F_D^t)_{ij}|$ and $B_2=\us{\us{\sup}{i,j}}{t\geq 0}|(F_D^{t\top}F_D^t)_{ij}|$
\end{lemma}
\begin{align}
\eb_t&=\frac{1}{t+1}(\ous{\sum}{i=0}{t} F_{i,1} e_0+ \alpha\ous{\sum}{i=1}{t} \ous{\sum}{k=i}{t} F_{k,i+1}  \zeta_i )\nn\\
&=\frac{1}{t+1}\big((\alpha H)^{-1}(I-F_D^t)e_0+ H^{-1}\ous{\sum}{i=1}{t}(I-F_D^{t-i}) \zeta_i\big)
\end{align}
\begin{theorem}[AS-AN]
When $H$ is AS and $M_t=0,\forall t\geq 0$, we have
\begin{align}
E\norm{H\eb_t}\leq \Big(\frac{\norm{\theta_0-\ts}^2_2}{\alpha^2(t+1)^2}+\frac{\sigma^2}{t+1}\Big)\big(1+2nB_1+n^2B_2\big)
\end{align}
\end{theorem}
\begin{theorem}[PS-AN]
When $H$ is PS, $M_t=0,\forall t\geq 0$ and $\alpha\in (0,\alpha_{ps})$
\begin{align}
E\norm{H\eb_t}\leq \Big(\frac{\norm{\theta_0-\ts}^2_2}{\alpha^2(t+1)^2}+\frac{\sigma^2}{t+1}\Big)\big(1+\frac{2}{1-\norm{F_D}_{op}}+\frac{1}{1-\norm{F_D}^2_{op}}\big)
\end{align}
\end{theorem}
\begin{theorem}[SPDS-AN]
When $H$ is SPDS and $M_t=0,\forall t\geq 0$, and $\alpha\in (0,\alpha_{ps})$
\begin{align}
E\norm{H\eb_t}\leq \Big(\frac{\norm{\theta_0-\ts}^2_2}{\alpha^2(t+1)^2}+\frac{\sigma^2}{t+1}\Big)
\end{align}
\end{theorem}
\begin{theorem}[SPDS-Structured Noise]
When $H$ is SPDS and $M_t=0,\forall t\geq 0$, and $\alpha\in (0,\alpha_{ps})$
\begin{align}
E\norm{H^{\frac{1}{2}}\eb_t}\leq \Big(\frac{\norm{H^{-\frac{1}{2}}(\theta_0-\ts)}^2_2}{\alpha^2(t+1)^2}+\frac{\sigma^2}{t+1}\Big)
\end{align}
\end{theorem}
\begin{theorem}[SPDS-Unstructured Noise]
When $H$ is SPDS and $M_t=0,\forall t\geq 0$, and $\alpha\in (0,\alpha_{ps})$
\begin{align}
E\norm{H^{\frac{1}{2}}\eb_t}\leq \norm{H^{-1}}^2_op\Big(\frac{\norm{(\theta_0-\ts)}^2_2}{\alpha^2(t+1)^2}+\frac{\sigma^2}{t+1}\Big)
\end{align}
\end{theorem}



\section{LSAs to solve linear system of equations}
 We now look at the three different LSAs to solve the linear system given by
\begin{align}\label{linsys}
A\theta=b.
\end{align}
In what follows, we assume that we have access to the stochastic orcales which provide $i.i.d$ samples of $A$ and $b$.
\subsection{Least Squares Linear Stochastic Approximation (LS-LSA)}
When $A$ is not known to satisfy any specific assumption such as AS, PS or SPDS, it is important to enforce stability by designing the LSA appropriately. One way to ensure SPDS is to solve for $A^TA\theta^*=A^\top b$. We call the following stochastic recursion to compute $\theta^*$ a LS-LSA algorithm
\begin{align}\label{lslsa}
\theta_{t+1}=\theta_t+\alpha A^{(1)\top}_t(b_t-A^{(2)}_t\theta_t),
\end{align}
where we assume $b_t$, $A^{(1)}_t$and $A^{(2)}_t$ are $i.i.d$ samples of $b$ and $A$. The LSER corresponding to LS-LSA in \eqref{lslsa} is given by
\begin{align}\label{lslser}
e_{t+1}=&(I-\alpha A^{(1)\top}_tA^{(2)}_t)e_t+\alpha(A^\top M^{(3)}_{t+1}+M^{(1)\top}_{t+1}M^{(3)}_{t+1}\nn\\&-A^\top M^{(2)}_{t+1}\ts-M^{(1)\top}_{t+1}M^{(2)}_{t+1}\ts),
\end{align}
where $M^{(1)}_{t+1}=A^{(1)}_t-A$, $M^{(2)}_{t+1}=A^{(2)}_t-A$ and $M^{(3)}_{t+1}=b_t-b$. Choosing $H_t=A^{(1)\top}_tA^{(2)}_t$, $\zeta_t= A^\top M^{(3)}_{t+1}+M^{(1)\top}_{t+1}M^{(3)}_{t+1}-A^\top M^{(2)}_{t+1}\ts-M^{(1)\top}_{t+1}M^{(2)}_{t+1}\ts$ and $\sigma^2=\parallel A\parallel^2 (\sigma_2^2+\sigma_1^2\parallel \ts\parallel^2)+ \sigma_1^4(1+\parallel \ts\parallel^2)$ we can identify \eqref{lslser} with \eqref{lsergen}, and  the results of \Cref{spdsmn} applies to the LSER in \eqref{lslser}.
\subsection{Fixed Point Linear Stochastic Approximation (FP-LSA)}
When it is known that $A$ is PS,  we can use the following LSA which we call the fixed point LSA (FP-LSA) to compute $\theta^*$:
\begin{align}\label{fplsa}
\theta_{t+1}=\theta_t+\alpha(b_t-A_t),
\end{align}
where we assume that $b_t$ and $A_t$ are $i.i.d$ samples of $b$ and $A$. We now present the LSER corresponding the FP-LSA in \eqref{fplsa}.
\begin{align}\label{fplser}
e_{t+1}=(I-\alpha A_t)e_t+\alpha(M^{(2)}_{t+1}-M^{(1)}_{t+1}\ts),
\end{align}
where $M^{(1)}_{t+1}=A_t-A$, $M^{(2)_{t+1}}=b_t-b$,  and $\ts=A^{-1}b$. Now letting $H_t=A_t$, $\zeta_t=M^{(2)}_{t+1}-M^{(1)}_{t+1}\ts$ and $\sigma=\sigma_1^2+\sigma^2_2 \parallel\ts\parallel^2$ we can identity \eqref{fplser} wit the \eqref{lsergen} and the results of \Cref{psmn} applies to the LSER in \eqref{fplser}.
\subsection{Saddle-Point Linear Stochastic Approximation (SP-LSA)}
An important downside of \Cref{lslsa} is that two independent samples of $A$ are available in the form of $A^{(1)\top}_t$ and $A^{(2)}_t$ which might be possible in certain applications. This issue can be alleviated by looking at the following Saddle-Point objective function
\begin{align}
L(\theta,y)=\min_{\theta\in \R^n}\max{y\in \R^n}<b-A\theta,y>-\frac{1}{2}\parallel y\parallel_M ^2
\end{align}
By noticing the fact that $\nabla_\theta L =-A^\top y$ and $\nabla_{y}L=-A\theta-M$, we now specify the SP-LSA algorithm as follows:
\begin{align}\label{splsa}
\begin{split}
y_{t+1}&=y_t+\alpha (b_t-A_t\theta_t- M y_t)\\
\theta_{t+1}&=\theta_t+\alpha(A_t^\top)y_t
\end{split}
\end{align}
The SLER corresponding to the SP-LSA in \eqref{splsa} is given by
\begin{align}
e_{t+1}=\Big(\begin{bmatrix} I & 0 \\ 0 &I\\\end{bmatrix} -\begin{bmatrix} M & A_t \\ -A_t^\top &0\\\end{bmatrix}\Big)e_t+\Big(\begin{bmatrix} M^{(2)}_{t_1}-M^{(1)}_{t+1}\ts  \\ 0\\\end{bmatrix}\Big),
\end{align}
where $M^{(1)}_{t+1}=A_t-A$, $M^{(2)_{t+1}}=b_t-b$,  and $\ts=A^{-1}b$. Now letting $H_t=\begin{bmatrix} M & A_t \\ -A_t^\top &0\\\end{bmatrix}$, we notice that $H=\E[H_t]=\begin{bmatrix} M & A \\ -A^\top &0\\\end{bmatrix}$, is not $PS$ and hence it is not straightforward to derive finite time bounds. However, by replacing $\alpha$ in \eqref{splsa} by $\alpha_t$ which satisfies \Cref{dimassmp}, the behaviour of the iterates can then be analysed using \Cref{sadim}.
\comment{
Let $x_t=(y_t,\theta_t)$, $g_t=\begin{bmatrix} b_t\\ 0\\\end{bmatrix}$ and $G^{(i)}_t=\begin{bmatrix} M & A^{(i)}_t \\ -A^{(i)\top}_t &0\\\end{bmatrix}$ for $i=1,2$, then consider the following updates
\begin{align}
x'_{t}&=x_t+\alpha (g_t-G^{(1)}_t x_t)\\
x_{t+1}&=x_t+\alpha(g_t- G^{(2)}_t x'_t);
\end{align}
}

\section{Background: Reinforcement Learning}
We now present a brief overview of Markov Decision Processes (MDPs) which is a framework to describe the RL setting, i.e., the agent's interaction with the environment. An MDP is a five tuple and is denoted by $M=\{S,A,P,R,\gamma\}$, where $S$ is the state space, $A$ is the action space, $P=(p_a(s,s'),a\in A,s,s’\in S)$ is the probability transition kernel that specifies the probability of transitioning from state $s$ to $s'$ when the agent taking an action $a$, and $R(s,a)\colon S\times A\ra \R$ is the reward for taking an action $a$ in state $s$ and $0<\gamma<1$ is a given discount factor. A stationary deterministic policy (or simply a policy) is a denoted by $\pi=(\pi(s,\cdot),s\in S)$, where $\pi(s,\cdot)$ is a probability distribution over the set of actions. The infinite horizon discounted reward of a policy $\pi$ is given by, $V_\pi=\mathbf{E}[\sum_{t=0}^\infty \gamma^t R(s_t,a_t)| s_0=s, \pi]$,
and is the discounted sum of rewards starting from state $s$ and taking actions according to policy $\pi$. The value function $V_\pi$ can be computed by solving the Bellman equation (BE) given below:
\begin{align}\label{be}
V_\pi=R_\pi+\gamma P_\pi V_\pi,
\end{align}
where $R_\pi(s)=\sum_{a\in A}\pi(s,a) R(s,a)$ and $p_\pi(s,s’)=\sum_{a\in A}\pi(s,a)p_a(s,s’)$.\par
\subsection{Projected Bellman Equation}
The BE is a linear sytem of equations, and needs to be solved from the samples. However, when the number of states $|S|$ is large, it is common to use a linear representation for value function i.e., to let $V_\pi\approx\Phi \ts$. The value function can then be learnt by solving a projected Bellman equation (PBE) instead of the BE in \eqref{be}. The PBE is given as follows
\begin{align}
\Phi\ts=\Pi (R_\pi+\gamma P_\pi \Phi \ts),
\end{align}
where $\Pi=\Phi(\Phi^\top \Xi\Phi)^{-1}\Phi^\top\Xi$, where is a diagonal matrix with positive diagonal entries $\xi_{s}, s\in S$ such that $\sum_{s\in S} \xi=1$. Note that $\xi_s$ here stands for the weight of the state $s$, and $\Pi V= {\arg\min}_{\theta\in \R^n}\parallel \Phi \theta -V\parallel_{\xi}$. One can re-write the PBE as below
\begin{align}\label{tdlin}
\Phi\ts&= (\Phi^\top \Xi\Phi)^{-1}\Phi^\top\Xi (R_\pi+\gamma P_\pi \Phi \ts)\nn\\
(\Phi^\top \Xi\Phi)\ts&= \Phi^\top\Xi (R_\pi+\gamma P_\pi \Phi \ts)\nn\\
\underbrace{\Phi^\top \Xi(I-\gamma P_\pi)\Phi}_{A}\ts&= \underbrace{\Phi^\top\Xi R_\pi}_{b}
\end{align}
We consider the scenario where we have data $\D=\{(s_i,a_i,s'_i),\forall i=1,\ldots,N\}$, where $s_i\sim \xi$, $a_i\sim \pi_b(s_i,\cdot)$, $s'_i\sim p_{a_i}(s_i,\cdot)$, where $\pi_b$ is the \emph{behavior} policy, which might be different from the \emph{target} policy $\pi$ whose value function $V_\pi$ we are interested in computing. This scenario is known as the \ofp setting. A special case is that of \onp setting wherein $\pi_b=\pi$. A quantity of interest in this context is the importance sampling ratio given by $\rho(s,a)=\frac{\pi(s,a)}{\pi_b(s,a)}$.\par
In the RL setting, model parameters $P^\pi$ and $R^\pi$ are not explicitly available. This implies that we have to solve for \eqref{tdlin} using noisy samples of $A=\Phi^\top \Xi(I-\gamma P_\pi)\Phi$ and $b=\Phi^\top\Xi R_\pi$.\par
\begin{lemma}
Let $A_t=\rho(s_i,a_i)phi^\top(s_i)(\phi(s_i)-\gamma \phi(s'_i))$ and $b_t=\rho(s_t,a_t)\phi^\top(s_t) R(s_t,a_t)$ then we have $\E[A_t|\xi]=A$ and $\E[b_t|\xi]=b$.
\end{lemma}
\section{LSA algorithms in RL}
We now look at LSA algorithms occuring in RL that solve \eqref{tdlin} using noisy samples.
\subsection{Fixed Point LSA}
The vanilla temporal difference (simply TD) algorithm is a LSA to solve \eqref{tdlin}. The TD algorithm is given as follows:
\begin{align}\label{vtd}
\theta_{t+1}=\theta_t+\alpha_t \rho(s_t)\phi^\top(s_t)(R(s_t,a_t)+\gamma \phi(s'_t)\theta_t- \phi(s_t)\theta_t),
\end{align}
Thus results in \Cref{sec:fp} apply to this algorithm
\subsection{Saddle Point LSA}
The gradient temporal difference (GTD) learning algorithms were shown to be gradient algorithms with respect to the saddle point objectives. We present GTD and GTD2 with their saddle point objective functions
\begin{align}\label{gtd}
\begin{split}
\textbf{GTD(GTD2):\quad}y_{t+1}&=y_t+\alpha \rho_t(b_t-A_t\theta_t - M y_t)\\
\theta_{t+1}&=\theta_t+\alpha\rho_t(A^\top_ty_t),
\end{split}
\end{align}
where GTD/GTD2 can be obtained by letting $M=I/C$ respectively in \eqref{gtd}. Results of \Cref{sec:splsa} apply to the GTD algorithms
\comment{
\subsection{Least Squares LSA}
We now present a novel algorithm called the \emph{incremetal-gradient} least-squares temporal difference leanring (iGLTSD), named so owing to the fact that it makes use of gradient the least-squares objective and at the same times makes incremental updates like the iLSTD algorithm.
\FloatBarrier
\begin{algorithm}[H]
\begin{algorithmic}[1]
\STATE{$s\la s_0, b\la 0, A\la 0, \hat{A}\la 0 \mu\la 0, t\la 0$}
\STATE{Initialize $r_0$ arbitrarily}
\FOR{$t=0,1,2,\ldots,N$}
%\STATE{$\Delta b_t= \phi(s_t) r_t$}
%\STATE{$\Delta A_t= \phi(s_t)(\phi(s_t)-\gamma\phi(s_t))^\top$}
%\STATE{$ b_{t+1}= b_t+\Delta b_t$}
%\STATE{$ A_{t+1}= A_t+\Delta A_t$}
\STATE{$ \mu_{t}= \rho(s_t)\phi^\top(s_t)(\phi(s_t)-\gamma\phi(s'_t))$}
\STATE {$\theta_{t+1}= \theta_t+\alpha \hat{A}\top\mu_t$}
\STATE{$j\la$ choose an random index from $1,\ldots,n$}
\STATE {$\hat{A}= \phi(s_t)(j)(\phi(s_t)-\gamma\phi(s'_t))$}
\ENDFOR
\end{algorithmic}
\end{algorithm}
The analysis of this algorithm follows from \Cref{sec:lslsa}.

}

\section{Price of Off-Policy Stability}
We discuss the price paid by GTD variants to achieve off-policy stability.
For the purpose of our discussions, we study the following modified version of GTD given by:
\begin{align}\label{gtdbeta}
\begin{split}
\textbf{GTD($\beta$)\quad}y_{t+1}&=y_t+\alpha \rho_t(b_t-A_t\theta_t - M y_t)\\
\theta_{t+1}&=\theta_t+\alpha\rho_t\beta(A^\top_ty_t)
\end{split}
\end{align}
We call the above variant GTD($\beta$).
In order to keep the discussion intuitive, we first consider a simple example MDP and then
\begin{example}\label{onestatemdp}
Consider the MPD with $1$ state, $1$ action, the probability transition $p=1$, reward $R=1$, discount factor $\gamma 1-\epsilon$ (for small $\epsilon>0$ ) and feature matrix $\Phi=1$. Here $A=1-\gamma=\epsilon$, $R=1$, $V^*=\frac{1}{\epsilon}$.
\end{example}
We now write down the TD and GTD($\beta$) algorithms for this $1$-state MDP in \Cref{onestatemdp}.
\begin{align}\label{tdonestate}
\begin{split}
\theta_{t}&=\theta_{t-1}+\alpha(1-\epsilon\theta_{t-1})\\
&=(1-\alpha\epsilon)\theta_{t-1}+\alpha 1
\end{split}
\end{align}
The Eigen value of the above TD update is given by $\mu=1-\alpha\epsilon$.
%The Eigen value decay is given by $\mu^t\approx e^{-\alpha\epsilon t}$
\begin{align}\label{gtdbeta}
\begin{bmatrix} y_t \\ \theta_t\\\end{bmatrix}=\Big(\begin{bmatrix} 1&0 \\ 0& 1\\\end{bmatrix} - \alpha\begin{bmatrix} 1&\epsilon \\ -\beta\epsilon& 0\\\end{bmatrix}\Big)\begin{bmatrix} y_t \\ \theta_t\\\end{bmatrix}+\alpha\begin{bmatrix} 1 \\ 0\\\end{bmatrix}
\end{align}
%The Eigen values of $H=\begin{bmatrix} 1-\alpha& -\alpha\epsilon \\ \alpha\beta\epsilon& 1\\\end{bmatrix}$ is given by:
\begin{align}
1-\frac{\alpha}{2}(1\pm\sqrt{1-4\beta\epsilon^2}),
\end{align}
and consider the various cases of $\beta$ as follows.\\
\comment{
\begin{table}
\resizebox{0.5\textwidth}{!}{
\begin{tabular}{|c|c|c|c|}\hline
&$0<\beta\leq1 $ &     $\beta=\frac{1}{4\epsilon^2}$  & $\beta>\frac{1}{4\epsilon^2}$ \\ \hline
$\begin{aligned} \text{Eigen}\\ \text{Value}\end{aligned}$ &$\begin{aligned} &\mu_1=1- \alpha(1-\beta\epsilon^2) \\ &\mu_2=1-\alpha(\beta\epsilon^2) \\ & \end{aligned}$ & $\begin{aligned}\mu_1=1- \frac{\alpha}{2} \\ &\mu_2=1-\frac{\alpha}{2} \\ \end{aligned}$ & $\begin{aligned}\mu_{1}=1-\frac{\alpha}{2}(1+ i\sqrt{4\beta\epsilon^2-1})\\ \mu_2=1-\frac{\alpha}{2}(1- i\sqrt{4\beta\epsilon^2-1})\end{aligned}$\\ \hline
$\begin{aligned} \text{Spectral}\\ \text{Radius}\end{aligned}$ &$\begin{aligned} 1-\alpha(\beta\epsilon^2)  \end{aligned}$ & $1-\frac{\alpha}{2}$ & $\begin{aligned} \sqrt{1-\frac{\alpha}{2}+\alpha^2\beta\epsilon^2}\end{aligned}$\\ \hline
\end{tabular}
}
\end{table}
}
\begin{table}
\resizebox{0.5\textwidth}{!}{
\begin{tabular}{|c|c|c|}\hline
$0<\beta\leq1 $ &     $\beta=\frac{1}{4\epsilon^2}$  & $\beta>\frac{1}{4\epsilon^2}$ \\ \hline
 $\begin{aligned} &\mu_1=1- \alpha(1-\beta\epsilon^2) \\ &\mu_2=1-\alpha(\beta\epsilon^2) \\ & \end{aligned}$ & $\begin{aligned}&\mu_1=1- \frac{\alpha}{2} \\ &\mu_2=1-\frac{\alpha}{2} \\ \end{aligned}$ & $\begin{aligned}\mu_{1}=1-\frac{\alpha}{2}(1+ i\sqrt{4\beta\epsilon^2-1})\\ \mu_2=1-\frac{\alpha}{2}(1- i\sqrt{4\beta\epsilon^2-1})\end{aligned}$\\ \hline
$\begin{aligned} 1-\alpha(\beta\epsilon^2)  \end{aligned}$ & $1-\frac{\alpha}{2}$ & $\begin{aligned} \sqrt{1-\frac{\alpha}{2}+\alpha^2\beta\epsilon^2}\end{aligned}$\\ \hline
\end{tabular}
}
\end{table}


\comment{
\textbf{Case $\beta=1$: Why is GTD slow?} Since $\epsilon>0$ is very small, we can then approximate $\sqrt{1-4\epsilon^2}\approx 1-\frac{4\epsilon^2}{2}$ and the roots of $H$ in this case are then given by
\begin{align}
1-\frac{\alpha}{2}\big(1\pm (1-\frac{4\epsilon^2}{2})\big)=\underbrace{1- \alpha(1-\epsilon^2)}_{\mu_1}, \underbrace{1-\alpha(\epsilon^2)}_{\mu_2}
\end{align}
%The Eigevn value decay is given by $\mu_1^t\approx e^{-\alpha(1-\epsilon^2)t}$ and $\mu_2^t\approx e^{-\alpha\epsilon^2 t}$. Note that $\mu_2^t$ is much slower compared to $\mu^t$.\\
\textbf{Case $0<\beta<1$} The Eigen values of $H$ are then approximately given by
\begin{align*}
1-\frac{\alpha}{2}(1\pm(1-4\beta\epsilon^2))=\underbrace{1- \alpha(1-\beta\epsilon^2)}_{\mu_1}, \underbrace{1-\alpha(\beta\epsilon^2)}_{\mu_2}
\end{align*}
%Using arguments similar for the $\beta=1$ case, we can see that the $\beta<1$ is also worse.\\
\textbf{Case $\beta>>1$ Faster GTD} When $\beta>\frac{1}{4\epsilon^2}$ the Eigen values of $H$ split into two complex conjugates and are given by
\begin{align*}
1-\frac{\alpha}{2}(1\pm i\sqrt{4\beta\epsilon^2-1})=\rho e^{i\psi}
\end{align*}
where the modulus $\rho$ is given by $\rho=\sqrt{1-\frac{\alpha}{2}+\alpha^2\beta\epsilon^2}$, choosing $\beta=\frac{1}{4\epsilon^2}$ we have $\rho=(1-\frac{\alpha}{2})$. Thus, in this example atleast by choosing the right value of $\beta$ it looks like we can eliminate the dependence of.
\textbf{Comparison of Eigen Values}
}
\subsection{Effect of Condition Number}
\textbf{1-dimension}

\subsection{Maximum Allowable Step-Size}\label{opti}
The condition that $0<\alpha<\alpha_{\max}$ ensures that $\rho_{\alpha}<1$ and in \cite{bachaistats} authors conjecture that this bound on $\alpha_{\max}$ is strict i.e., there exists some initial condition $x_0$ for which the LSA in \eqref{linearrec} is unstable. We now present a theorem from \cite{logexp} and simple counter examples to falsify this conjecture.
\begin{theorem}\label{explog}
Let $\mathcal{H}=(H_t), t\geq 0$ be a stationary process of $n\times n$ real-valued matrices over some probability space $(\Omega,\F,\mathcal{P})$. If $\E\log^+\parallel H_0\parallel<\infty$ (where $\log^+ x$ denotes the positive part of $\log x$), then there exists a $\lambda\in \R$ that satisfies
\begin{align}\label{lambda}
\lambda=\lim_{t}\frac{1}{t}\log\parallel H_t H_{t-1}\ldots H_0\parallel
\end{align}
\end{theorem}
In the case when the implicit relation in \eqref{lambda} yields a $\lambda<0$, it is also implied that $\{z_t\},t\geq 0$ such that $z_t=H_t H_{t-1}\ldots H_0z_0$ is stable.
\begin{example}[Linear Prediction]
Let the data represented as $(input,output)$ be $(X_t,Y_t)\in \{(2,0), (4,0)\}$ with equal probability. The problem of linear prediction is then to find $\theta^*\in R$ such that it minimizes the loss $\E(X_t \theta^* -Y_t)^2$, and the SGD algorithm to solve find $\theta^*$ is given by
\begin{align}
\theta_{t+1}=\theta_t+\alpha(y_t X_t-X_t\otimes X_t\theta_t),
\end{align}
where $\alpha$ is the constant step-size. Now the condition on $\alpha_{\max}$ presented in \Cref{alphacond} and in \cite{bachaistats}, translates to the following numerical condition in this specific example
\begin{align*}
\alpha_{\max}<\frac{2H}{\E[H_t^\top H_t]}=\frac{10}{17}
\end{align*}
We now derive $\alpha^{\lambda}_{\max}$ which is the maximum allowable constant step-size as suggested by the implicit relation in \eqref{lambda}. We have
\begin{align}
\lambda=\frac{1}{2}\log(1-\alpha)+\frac{1}{2}\log(1-\alpha 4).
\end{align}
For stability we need $\lambda<0$, i.e., $-1<(1-\alpha)(1-alpha 4)<1$, which translates to the condition $0<\alpha<\alpha^{\lambda}_{\max}=\frac{5}{4}$. It is clear that $\alpha^{\lambda}_{\max}>\frac{2}{\E[H]}=0.8>\alpha_{\max}$.
\end{example}
A similar counter example can be provided in the asymmetric case as follows
\begin{example}[General LSA]
Consider the LSA in \eqref{linearrec} with $x_t\in \R$ and $H_t\sim \{-1, 2\}$ (with equal probability)  and $g_t=0$. Then 
$
\alpha_{\max}<\frac{2H}{\E[H_t^\top H_t]}=\frac{2}{5}
$
and since 
$
\lambda=\frac{1}{2}\log(1+\alpha)+\frac{1}{2}\log(1-\alpha 2).
$
we have $\alpha^{\lambda}_{\max}=0.78078$. It is clear that $\alpha^{\lambda}_{\max}>\alpha_{\max}$. However, in the asymmetric case we have $\frac{2}{\E[H]}=1>\alpha^{\lambda}_{\max}>\alpha_{\max}$.
\end{example}
%\FloatBarrier
\begin{figure}[htp]
\begin{minipage}{0.5\textwidth}
\resizebox{1.0\textwidth}{!}{
\begin{tabular}{cc}
\begin{tikzpicture}[scale=1,font=\Large,]
    \begin{axis}[
        xlabel=$t$,
        ylabel=$\theta_t$,legend style={at={(0.5,-0.1)},anchor=north}
]

    \addplot[only marks,mark=square,red] plot file {./experiments/symm_stable_samp};
    \addplot[only marks,mark=diamond,blue] plot file {./experiments/asymm_stable_samp};

\addlegendentry{{\color{black}{Example~1,$\alpha=1.1$}}}
\addlegendentry{{\color{black}{Example~2,$\alpha=0.7$}}}


    \addplot[thick,dashed,mark=.,red] plot file {./experiments/symm_stable};
    \addplot[thick,dashed,mark=.,blue] plot file {./experiments/asymm_stable};

    \end{axis}
    \end{tikzpicture}

&
\begin{tikzpicture}[scale=1,font=\Large]
    \begin{axis}[
        xlabel=$t$,
        ylabel=$\theta_t$,legend style={at={(0.5,-0.1)},anchor=north}
]

    \addplot[only marks,mark=square,red] plot file {./experiments/symm_unstable_samp};
    \addplot[only marks,mark=diamond,blue] plot file {./experiments/asymm_unstable_samp};

\addlegendentry{{\color{black}{$\theta_t$ Example~1,$\alpha=1.3$}}}
\addlegendentry{{\color{black}{$\theta_t\times 10^{25}$Example~2,$\alpha=0.8$}}}


    \addplot[thick,dashed,mark=.,red] plot file {./experiments/symm_unstable};
    \addplot[thick,dashed,mark=.,blue] plot file {./experiments/asymm_unstable};

    \end{axis}
    \end{tikzpicture}

\end{tabular}
}
\end{minipage}
\end{figure}



\begin{table}
\begin{center}
\resizebox{0.5\textwidth}{!}{
\begin{tabular}{|c|c|c|c|c|c|c|}\hline
System  &   Matrix      & Noise & $H^{(1)}$ &  $H^{(2)}$   &   $\zeta^{(1)}$   &   $\zeta^{(2)}$   \\\hline
$S_1$   & AS          &AUS    &$\begin{bmatrix} 1 & 100 \\ 0 &1\\\end{bmatrix}$ &$\begin{bmatrix} 1 & 100 \\ 0 &1\\\end{bmatrix}$ &$\begin{bmatrix} U_1 \\ U_2\\\end{bmatrix}$ &$\begin{bmatrix} U_1 \\ U_2\\\end{bmatrix}$ \\ \hline
$S_2$   & AS          &AUS    &$\begin{bmatrix} 1 & 0.1 \\ -0.1 &0\\\end{bmatrix}$ &$\begin{bmatrix} 1 & 0.1 \\ -0.1 &0\\\end{bmatrix}$ &$\begin{bmatrix} U_1 \\ U_2\\\end{bmatrix}$ &$\begin{bmatrix} U_1 \\ U_2\\\end{bmatrix}$ \\ \hline

$S_3$   &PS          & AUS   &$\begin{bmatrix} 1 & 1 \\ 0 &1\\\end{bmatrix}$ &$\begin{bmatrix} 1 & 1 \\ 0 &1\\\end{bmatrix}$ &$\begin{bmatrix} U_1 \\ U_2\\\end{bmatrix}$ &$\begin{bmatrix} U_1 \\ U_2\\\end{bmatrix}$ \\ \hline
$S_4$   &SPDS          & AS   &$\begin{bmatrix} 1 & 0 \\ 0 &1\\\end{bmatrix}$ &$\begin{bmatrix} 1 & 0 \\ 0 &1\\\end{bmatrix}$ &$\begin{bmatrix} U_1 \\ U_2\\\end{bmatrix}$ &$\begin{bmatrix} U_1 \\ U_2\\\end{bmatrix}$ \\ \hline
$S_5$   &SPDS          & AUS   &$\begin{bmatrix} 1 & 0 \\ 0 &0.01\\\end{bmatrix}$ &$\begin{bmatrix} 1 & 0 \\ 0 &0.01\\\end{bmatrix}$ &$\begin{bmatrix} U_1 \\ U_2\\\end{bmatrix}$ &$\begin{bmatrix} U_1 \\ U_2\\\end{bmatrix}$ \\ \hline
$S_6$   &SPDS          & AS   &$\begin{bmatrix} 1 & 0 \\ 0 &0.01\\\end{bmatrix}$ &$\begin{bmatrix} 1 & 0 \\ 0 &0.01\\\end{bmatrix}$ &$\begin{bmatrix} U_1 \\ 0\\\end{bmatrix}$ &$\begin{bmatrix} 0 \\ 0.1 U_2\\\end{bmatrix}$ \\ \hline
$S_7$   &AS          & MUS   &$\begin{bmatrix} 1 & 1.1 \\ -0.9 &0\\\end{bmatrix}$ &$\begin{bmatrix} 1 & -0.9 \\ -0.8 &0 \\\end{bmatrix}$ &$\begin{bmatrix} U_1 \\ U_2\\\end{bmatrix}$ &$\begin{bmatrix} U_1 \\ U_2\\\end{bmatrix}$ \\ \hline
$S_8$   &PS          & MUS   &$\begin{bmatrix} 1 & 0.2 \\ 0.8 &0.1\\\end{bmatrix}$ &$\begin{bmatrix} 1 & 1.8 \\ -0.8 &0.1\\\end{bmatrix}$ &$\begin{bmatrix} U_1 \\ U_2\\\end{bmatrix}$ &$\begin{bmatrix} U_1 \\ U_2\\\end{bmatrix}$ \\ \hline
$S_9$   &SPDS          & MS   &$\begin{bmatrix} 1 & 0 \\ 0 &0\\\end{bmatrix}$ &$\begin{bmatrix} 0 & 0 \\ 0 &0.01\\\end{bmatrix}$ &$\begin{bmatrix} U_1 \\ 0\\\end{bmatrix}$ &$\begin{bmatrix} 0 \\ 0.1 U_2\\\end{bmatrix}$ \\ \hline
$S_{10}$   &SPDS          & MUS   &$\begin{bmatrix} 1 & 0 \\ 0 &0\\\end{bmatrix}$ &$\begin{bmatrix} 0 & 0 \\ 0 &0.01\\\end{bmatrix}$ &$\begin{bmatrix} U_1 \\ U_2\\\end{bmatrix}$ &$\begin{bmatrix} U_1 \\ U_2\\\end{bmatrix}$ \\ \hline
$S_{11}$   &SPDS          & MS   &$\begin{bmatrix} 1 & 0 \\ 0 &0\\\end{bmatrix}$ &$\begin{bmatrix} 0 & 0 \\ 0 &1\\\end{bmatrix}$ &$\begin{bmatrix} U_1 \\ U_2\\\end{bmatrix}$ &$\begin{bmatrix} U_1 \\ U_2\\\end{bmatrix}$ \\ \hline
\end{tabular}
}
\end{center}
\end{table}


\fi
%\input{pltadd}
%\input{back}
%\section{Implication of the Results}
\emph{Temporal Difference} (TD) learning algorithms \cite{} are widely used to learn value functions. A desirable feature of these algorithms that they are LSA algorithms that perform only $O(n)$ computations per time step. The vanilla TD (or simply TD) algorithm was first introduced in \cite{}. An unresolved issue for quite some time was the divergent behavior of TD in the \emph{off-policy} setting, where the sample were obtained from a policy different from the one whose value function needs to be computed. The Gradient Temporal Difference (GTD) learning algorithm \cite{} addressed the issue of divergence and are provably stable in \emph{off-policy} scenarios.
%Thus the GTD methods solve \eqref{linsys} in the general case when $\pi_t\neq\pi_b$.
%The \ofp convergence issue has also been addressed differently by the least-squares TD (LSTD) algorithm, which constructs $A$ and $b$ in \eqref{lsa}, and obtain $\theta^*=A^{-1}b$. As a consequence of the %matrix inversions involved, the LTSD performs a minimum of $O(n^2)$ computations, which can be a possible downside when $n$ is large.\par
Recently, newer variants of GTD namely, projected GTD2 and GTD-\emph{Mirror-Prox} were proposed in \cite{}. The authors in \cite{} observe that the GTD algorithm can be derived as a true stochastic gradient algorithms with repsect to a primal-dual saddle point objective function. This observation enables application of finite-time results for stochastic gradient (SG) algorithms to the GTD algorithm and at the same time, derive the newer variants based on the stochastic \emph{Mirror-Prox} algorithm \cite{}.\par
While several variants of TD algorithms have been developed, and have been empirically tested, as far as we gather, there are a number of poorly understood aspects of these algorithms, which we list as under.
\begin{itemize}[leftmargin=*]
\item Stochastic approximation \cite{SA} theory typically needs diminishing and square summable step-sizes. Further, SA theory deals with study of ordinary differnetial equations (ODEs) and lead only to asymptotic rates. The diminishing nature of the step-size schedules results in poor asymptotics, with the rates of convergence depending on the spectrum of the matrices involved in the SA updates. This is undesirable since the matrix (see \eqref{linsys}) involves quantities which $i$) are unkown and $ii$) vary across environment.
\item It is quite common in literature to sweep the step-size to choose the right setting that results in best performance or sometimes even to achieve a convergent behavior \cite{}.
\item It has been observed in experiments that the GTD is slow. While its variant GTD2 is faster than GTD, it is still slower compared to TD in the \onp setting. While, it is fine to draw understanding from experiments, it is however necessary to tie such observations to a theoretical phenomenon in a principled manner.
\item The GTD-MP is based on the SMP algorithm, which as \cite{} notes is especially effective when the original non-smooth function admits a smooth saddle-point representation. However, in the case of GTD the original as well as the saddle-point objective functions are, and hence the intuition why GTD-MP is effective in the GTD setting is absent.
\end{itemize}
Almost all the TD algorithms (except the LSTD) perform only $O(n)$ computations per time step and their updates are linear in $\theta$. Hence, we feel that it makes sense to study them (TD and GTD algorithms) under the framework of linear stochastic approximation (LSA) schemes. Acknowledging the linear nature of the updates helps us to obtain a qualitative and quantitative understanding of the performance of the various TD algorithms. In particular, the spectral nature of the matrices involved in the linear updates dictate the behavior of these algorithm. For example, it is well known that the divergence of TD in the \ofp setting can be directly attributed to the fact that the $A$ matrix cannot be ensured to have all eigen values with strictly positive real parts. Also, the LSA framework helps throws light on the fact that finite time behavior of these algorithms involves understanding products of random matrices (an aspect unnoticed in TD literature).\par
In this paper, we make the following specific contributions
\begin{itemize}[leftmargin=*]
\item \textbf{Towards TD:} We study the vanilla TD algorithm with constant step-size and Rupert-Polyak (CSRP) iterate averaging. Specifically, we study the behavior of $\theta_{t+1}=\theta_t+\alpha (b_t-A_t)$, where $b_t$ and $A_t$ are noisy samples of $b$ and $A$ (in \eqref{linsys}) respectively. The finite time error is split into two terms namely the bias term (due to the initial condition) and the variance term (due to noise). We show that CSRP results in a finite time performance of $O(1/t^2)$ and $O(1/t)$ respectively for the bias and the variance terms and the constant do not depend on the condition number of $A$.
\item \textbf{Towards general LSAs}: We show a weaker result for general linear iterates of the form $\theta_{t+1}=\theta_t+\alpha (g(b_t,A_t)-H(A_t)\theta_t)$, where $g$ and $H$ are appropriate functions. This formulation helps us to study other LSA algorithms such as GTD, GTD-MP, iLSTD. Even in this setting, finite time performance of $O(1/t^2)$ and $O(1/t)$ respectively for the bias and the variance terms holds, however the constants have dependence on the condition number of $H(A)$.
\item \textbf{Towards Step-sizes and Stability:} We turn towards theory of product of random matrices to comment about conditions on the constant step size that leads to convergent algorithms. In paritcular, we show that are two distinct regimes, one being conservative with a smaller constant step size resulting in convergence of iterates in the mean square sense, and the other being relaxed with a relatively larger constant step size guaranteeing only asymptotic stability in high probability.
\item \textbf{Towards GTD:} We show that the GTD does not admit convergence in the mean square sense, due to block zero entries in the leading diagonal of the update matrix, result in a sub-system that can diverge. We tie the Mirror-Prox idea to \emph{Predictor-Corrector} method in numerical analysis, which enables us to show that the GTD-MP eliminates the divergence issue by ensuring positive entries in the leading diagonal. This provides the much needed intuition that was lacking with respect to why the Mirror-Prox idea makes a difference in the case of GTD. Further, the results for general LSA schemes applies to GTD-MP, resulting in finite-time performance bounds for the mean squared error.
\item \textbf{Towards iLSTD:} The incremental version of LSTD algorithm (iLSTD) in \cite{} is also an LSA algorithm. It follows that iLSTD does not converge in the \ofp setting even when LSTD does. We propose a stable version of iLSTD which converges in the \ofp setting.
\item \textbf{Towards Open Issues:} We note that the GTD-MP algorithm is incorrect when the samples are obtained from a single trajectory.
\end{itemize}
The key take aways of the contributions are listed as under.
\begin{itemize}[leftmargin=*]
\item  The CSRP minimizes the overhead on tuning the step-sizes, in that it is enough to choose a constant step-size without bothering about the entire sequence of step-sizes. This constant step size can be found out by experiments.
\item The distinct regimes of step-size implies that observing a handful of convergent trajectories in practice might not mean that the algorithm is converging in mean squared sense.
\item The `slowness' of the GTD (and its variants) can be easily observed by looking at the spectrum of the update equation.
\end{itemize}
In addition to the contributions listed above, our aim is to stitch together relvant theoretical and practical aspects of the TD algorithm with an aim to gain an overall perspective of what makes them tick. We also present some experiments on simple domains to illustrate intended the message.
\comment{\subsection{Related Work}
The analsyis in this work is inspired by a related work in \cite{bachaistats}, in which the authors consider the problem of linear prediction with the penalty function as quadratic loss under an i.i.d  (with respect to some unknown distribution) assumption on the data. The linear prediction problem in \cite{bachaistats} is solved by an stochastic gradient descent (SGD) algorithm which is an LSA algorithm as well.  An important difference in our case is that unlike the linear prediction setting, the matrices involved in the TD updates are not symmetric. We nevertheless show that analysis on the lines of \cite{bachaistats} still holds for the LSAs in the RL setting, thereby presenting an analysis under more generalized assumptions i.e., in the absence of symmetry.
}

%\input{stab}
%\section{Implication of the Results}
\emph{Temporal Difference} (TD) learning algorithms \cite{} are widely used to learn value functions. A desirable feature of these algorithms that they are LSA algorithms that perform only $O(n)$ computations per time step. The vanilla TD (or simply TD) algorithm was first introduced in \cite{}. An unresolved issue for quite some time was the divergent behavior of TD in the \emph{off-policy} setting, where the sample were obtained from a policy different from the one whose value function needs to be computed. The Gradient Temporal Difference (GTD) learning algorithm \cite{} addressed the issue of divergence and are provably stable in \emph{off-policy} scenarios.
%Thus the GTD methods solve \eqref{linsys} in the general case when $\pi_t\neq\pi_b$.
%The \ofp convergence issue has also been addressed differently by the least-squares TD (LSTD) algorithm, which constructs $A$ and $b$ in \eqref{lsa}, and obtain $\theta^*=A^{-1}b$. As a consequence of the %matrix inversions involved, the LTSD performs a minimum of $O(n^2)$ computations, which can be a possible downside when $n$ is large.\par
Recently, newer variants of GTD namely, projected GTD2 and GTD-\emph{Mirror-Prox} were proposed in \cite{}. The authors in \cite{} observe that the GTD algorithm can be derived as a true stochastic gradient algorithms with repsect to a primal-dual saddle point objective function. This observation enables application of finite-time results for stochastic gradient (SG) algorithms to the GTD algorithm and at the same time, derive the newer variants based on the stochastic \emph{Mirror-Prox} algorithm \cite{}.\par
While several variants of TD algorithms have been developed, and have been empirically tested, as far as we gather, there are a number of poorly understood aspects of these algorithms, which we list as under.
\begin{itemize}[leftmargin=*]
\item Stochastic approximation \cite{SA} theory typically needs diminishing and square summable step-sizes. Further, SA theory deals with study of ordinary differnetial equations (ODEs) and lead only to asymptotic rates. The diminishing nature of the step-size schedules results in poor asymptotics, with the rates of convergence depending on the spectrum of the matrices involved in the SA updates. This is undesirable since the matrix (see \eqref{linsys}) involves quantities which $i$) are unkown and $ii$) vary across environment.
\item It is quite common in literature to sweep the step-size to choose the right setting that results in best performance or sometimes even to achieve a convergent behavior \cite{}.
\item It has been observed in experiments that the GTD is slow. While its variant GTD2 is faster than GTD, it is still slower compared to TD in the \onp setting. While, it is fine to draw understanding from experiments, it is however necessary to tie such observations to a theoretical phenomenon in a principled manner.
\item The GTD-MP is based on the SMP algorithm, which as \cite{} notes is especially effective when the original non-smooth function admits a smooth saddle-point representation. However, in the case of GTD the original as well as the saddle-point objective functions are, and hence the intuition why GTD-MP is effective in the GTD setting is absent.
\end{itemize}
Almost all the TD algorithms (except the LSTD) perform only $O(n)$ computations per time step and their updates are linear in $\theta$. Hence, we feel that it makes sense to study them (TD and GTD algorithms) under the framework of linear stochastic approximation (LSA) schemes. Acknowledging the linear nature of the updates helps us to obtain a qualitative and quantitative understanding of the performance of the various TD algorithms. In particular, the spectral nature of the matrices involved in the linear updates dictate the behavior of these algorithm. For example, it is well known that the divergence of TD in the \ofp setting can be directly attributed to the fact that the $A$ matrix cannot be ensured to have all eigen values with strictly positive real parts. Also, the LSA framework helps throws light on the fact that finite time behavior of these algorithms involves understanding products of random matrices (an aspect unnoticed in TD literature).\par
In this paper, we make the following specific contributions
\begin{itemize}[leftmargin=*]
\item \textbf{Towards TD:} We study the vanilla TD algorithm with constant step-size and Rupert-Polyak (CSRP) iterate averaging. Specifically, we study the behavior of $\theta_{t+1}=\theta_t+\alpha (b_t-A_t)$, where $b_t$ and $A_t$ are noisy samples of $b$ and $A$ (in \eqref{linsys}) respectively. The finite time error is split into two terms namely the bias term (due to the initial condition) and the variance term (due to noise). We show that CSRP results in a finite time performance of $O(1/t^2)$ and $O(1/t)$ respectively for the bias and the variance terms and the constant do not depend on the condition number of $A$.
\item \textbf{Towards general LSAs}: We show a weaker result for general linear iterates of the form $\theta_{t+1}=\theta_t+\alpha (g(b_t,A_t)-H(A_t)\theta_t)$, where $g$ and $H$ are appropriate functions. This formulation helps us to study other LSA algorithms such as GTD, GTD-MP, iLSTD. Even in this setting, finite time performance of $O(1/t^2)$ and $O(1/t)$ respectively for the bias and the variance terms holds, however the constants have dependence on the condition number of $H(A)$.
\item \textbf{Towards Step-sizes and Stability:} We turn towards theory of product of random matrices to comment about conditions on the constant step size that leads to convergent algorithms. In paritcular, we show that are two distinct regimes, one being conservative with a smaller constant step size resulting in convergence of iterates in the mean square sense, and the other being relaxed with a relatively larger constant step size guaranteeing only asymptotic stability in high probability.
\item \textbf{Towards GTD:} We show that the GTD does not admit convergence in the mean square sense, due to block zero entries in the leading diagonal of the update matrix, result in a sub-system that can diverge. We tie the Mirror-Prox idea to \emph{Predictor-Corrector} method in numerical analysis, which enables us to show that the GTD-MP eliminates the divergence issue by ensuring positive entries in the leading diagonal. This provides the much needed intuition that was lacking with respect to why the Mirror-Prox idea makes a difference in the case of GTD. Further, the results for general LSA schemes applies to GTD-MP, resulting in finite-time performance bounds for the mean squared error.
\item \textbf{Towards iLSTD:} The incremental version of LSTD algorithm (iLSTD) in \cite{} is also an LSA algorithm. It follows that iLSTD does not converge in the \ofp setting even when LSTD does. We propose a stable version of iLSTD which converges in the \ofp setting.
\item \textbf{Towards Open Issues:} We note that the GTD-MP algorithm is incorrect when the samples are obtained from a single trajectory.
\end{itemize}
The key take aways of the contributions are listed as under.
\begin{itemize}[leftmargin=*]
\item  The CSRP minimizes the overhead on tuning the step-sizes, in that it is enough to choose a constant step-size without bothering about the entire sequence of step-sizes. This constant step size can be found out by experiments.
\item The distinct regimes of step-size implies that observing a handful of convergent trajectories in practice might not mean that the algorithm is converging in mean squared sense.
\item The `slowness' of the GTD (and its variants) can be easily observed by looking at the spectrum of the update equation.
\end{itemize}
In addition to the contributions listed above, our aim is to stitch together relvant theoretical and practical aspects of the TD algorithm with an aim to gain an overall perspective of what makes them tick. We also present some experiments on simple domains to illustrate intended the message.
\comment{\subsection{Related Work}
The analsyis in this work is inspired by a related work in \cite{bachaistats}, in which the authors consider the problem of linear prediction with the penalty function as quadratic loss under an i.i.d  (with respect to some unknown distribution) assumption on the data. The linear prediction problem in \cite{bachaistats} is solved by an stochastic gradient descent (SGD) algorithm which is an LSA algorithm as well.  An important difference in our case is that unlike the linear prediction setting, the matrices involved in the TD updates are not symmetric. We nevertheless show that analysis on the lines of \cite{bachaistats} still holds for the LSAs in the RL setting, thereby presenting an analysis under more generalized assumptions i.e., in the absence of symmetry.
}

%\section{Asymptotic analysis: The ODE method}
We now review some of the results in stochastic approximation theory \cite{SA}. Consider the following LSA with diminishing step sizes given by
\begin{align}\label{lsadim}
\theta_t=\theta_{t-1}+\alpha_t(b_t-H_t\theta_{t-1}),
\end{align}
where $\alpha_t$ satifies \Cref{dimassmp}
\begin{assumption}\label{dimassmp}
$\us{\lim}{t\ra\infty }\alpha_t=0, \us{\sum}{t\geq 0}\alpha_t=\infty, \us{\sum}{t\geq 0}\alpha_t^2<\infty$
\end{assumption}
The results in this section hold for the AS case and diminishing step sizes (see \Cref{} ), and are based on the ordinary differential equation (ODE) method \cite{SA,Kush}. The ODE method concerns itself with asymptotic behaviour of \eqref{lsadim}.
The idea here is to associate the following ODE in \eqref{ode} with \eqref{lsadim}:
\begin{align}\label{ode}
\dot{\theta(t)}=g-H\theta(t),
\end{align}
where $t\in\R$ and $t\geq 0$. Note that time $t$ in \eqref{ode} is continuous and time $t$ in \eqref{lsadim} is integer valued. In order to maintain the disctinction between discrete and continuous times, we denote the former by $t_d$ and the latter by $t_c$.
The ODE method centres around the following transformation/interpretation of the `discrete' time in \eqref{ode} to/as the `continuous' time in \eqref{lsadim}.
\begin{align}\label{stepacc}
t_c(t_d)=\ous{\sum}{s=0}{t_d}\alpha_s,
\end{align}
i.e., the continuous time $t_c$ corresponding to the discrete time $t_d$ is merely the step sizes accumulated during the discrete time period from $0$ to $t_d$. The iterates of \eqref{lsadim} can then be used to define a continuous time trajectory $\hat{\theta(t)}$ as follows:
\begin{align}\label{inter}
\hat{\theta(t)}=\theta_{t_d}+(\theta_{t_d+1}-\theta_{t_d})\frac{t-t_c(t_d) }{t_c(t_d+1)-t_c(t_d)}, ~\forall~t\in[t_c(t_d),t_c(t_d+1)].
\end{align}
Note in \eqref{inter} the continuous time trajectory $\hat{\theta(t)}$ is obtained by \emph{interpolating} the iterates of \eqref{lsadim}.
We now state a result that shows the trajectory of the ODE \eqref{ode} and the interpolated trajectory in \eqref{inter} are `close' to each other asymptotically.
\begin{theorem}\label{sadim}
Let $\theta^s(t), t\geq s$ be the trajectory to the ODE \eqref{ode} with $\theta^s(s)=\hat{\theta(s)}$. It holds \emph{almost surely} that for any $T>0$
\begin{align}
\us{\lim}{s\ra\infty} \us{\sup}{t\in[s,s+T]}\norm{\hat{\theta(t)}- \theta^s(t)} = 0,
\end{align}
and the iterates $\theta_t \ra \ts$ as $t\ra\infty$
\end{theorem}
\begin{proof}
See Chapter~$2$, \cite{SA}.
\end{proof}
We can now comment on the asymptotic convergence rates that are expected to hold in lieu of the \Cref{track}.
\subsection{Asymptotic Convergence Rates}\label{initial}
We denote the spectrum of $H$ by $\{\mu_i,i=1,\ldots,n\}$. It is known from standard results that the trajectory $\theta(t)$ the ODE \eqref{ode} with initial condition $\theta(0)=\theta_0$ is given by
\begin{align}\label{oderate}
\theta(t)-\ts=\sum_{i=1}^n C_i e^{-\mu_i t},
\end{align}
where $C=(C_i,i=1,\ldots,n)\in \R^n$ are real coefficients. The time $t\geq 0$ in \eqref{oderate} is the continous time defined in \eqref{stepacc}. It is easy to see from \eqref{oderate} that terms in the summation are dependent on the Eigen values values and the accumulation of step sizes as given by \eqref{stepacc}. For instance, if the step-size are chosen to be $\alpha_t=C/t$, then
\begin{align}\label{biasforget}e^{-\mu_i\sum_{0\leq k<t}\alpha_t}\approx e^{-\mu_i Clog s}=O(1/s^{\mu_i C})\end{align}


%\subsection{Gradient Temporal Difference Learning}
The instability of TD($0$) is due to the fact that it is not a true gradient descent algorithm. The first gradient-TD (GTD) algorithm was proposed by \citet{sutton2009convergent} and is based on minimizing the \emph{norm of the expected TD update} (NEU) given by
\begin{align}\label{neu}
NEU(\theta)=\parallel b_\pi -A_\pi\theta\parallel^2
=\E[\rho_t\phi_t^\top\delta_t(\theta)]^\top\E[\rho_t\phi_t^\top\delta_t(\theta)]
\end{align}
The GTD scheme based is on the gradient of the above expression which is given by (dropping subscript $t$ for convenience) $-\frac{1}{2}\nabla NEU(\theta)=\E[\rho (\phi-\gamma\phi')^\top \phi]\E[\rho \phi^\top\delta(\theta)]$. Since the gradient is a product of two expectation we cannot use a sample product (due to the presence of correlations). The GTD addresses this issue by estimating $\E[\rho\delta\phi^\top]$ in a separate recursion. The GTD updates can be given by
\begin{align}
\begin{split}
\textbf{GTD:\quad}y_{t+1}&=y_t+\alpha_t(\rho_t\phi^\top\delta_t -y_t)\\
\theta_{t+1}&=\theta_t+\alpha_t\rho_t(\phi_t-\gamma\phi'_t)^\top\phi_ty_t
\end{split}
\end{align}
Notice that $y$ updates are noisy Euler discretization of the ODE $\dot{y}=\E[\rho\delta\phi^\top]-y(t)$. The overall design of GTD is given by $\D_{GTD}=\langle \begin{bmatrix}b_\pi\\ 0\\ \end{bmatrix},\begin{bmatrix}-I &-A^\mu_\pi \\ {A^\mu_\pi}^\top &0 \\ \end{bmatrix}\rangle$.\par
%$\{g_1,H_1\}$ where $g_1=\begin{bmatrix}b_\pi\\ 0\\ \end{bmatrix}$, $H_1=\begin{bmatrix}-I &-\Phi^\top D_\mu (\Phi -\gamma P_\pi\Phi) \\ (\Phi -\gamma P_\pi\Phi)^\top D_\mu\Phi &0 \\ \end{bmatrix}$.\par
Instead of NEU, the \emph{mean-square projected Bellman Error} (MSPBE) can also be minimized. The MSPBE is defined as
\begin{align}\label{mspbe}
MSPBE(\theta)=\parallel J_\theta-\Pi T_\pi J_\theta \parallel^2_D
\end{align}
The GTD2 algorithm was proposed in \cite{gtdref} based on minimizing \eqref{mspbe}. The GTD2 updates are given by
\begin{align}
\begin{split}
\textbf{GTD2:\quad}y_{t+1}&=y_t+\beta_t\phi_t^\top(\rho_t\delta_t-\phi_t y_t)\\
\theta_{t+1}&=\theta_t+\alpha_t\rho_t(\phi_t-\gamma\phi’_t)^\top\phi_t y_t
\end{split}
\end{align}
The design of GTD2 is given by $\D_{GTD2}=\langle \begin{bmatrix}b_\pi\\ 0\\ \end{bmatrix}, \begin{bmatrix}-M &-A^\mu_\pi \\ {A^\mu_\pi}^\top &0 \\ \end{bmatrix}\rangle$, where $M=\Phi^\top D_\mu\Phi$.\par
The GTD-\emph{Mirror Prox}(GTD-MP) algorithm is given by the following update rule:
\begin{align}\label{gtdmp}
\begin{split}
\textbf{GTD-MP:} y_t^m=y_t+\alpha_t{\phi_t}^\top(R(s_t,a_t)+\gamma (\phi'_t-\phi_t)\theta_t),\\ \theta_t^m=\theta_t+\alpha_t({\phi_t}-\gamma\phi'_t )^\top\phi_ty_t,\\
 y_{t+1}=y_t+\alpha_t{\phi_t}^\top(R(s_t,a_t)+\gamma(\phi'_t-\phi_t)\theta^m_t), \\ \theta_{t+1}=\theta_t+\alpha_t({\phi_t}-\gamma\phi'_t )^\top\phi_ty^m_t,
\end{split}
\end{align}
The GTD-MP algorithm in \eqref{gtdmp} is the PC discretization of the GTD algorithm with the design matrix as $\D_{GTD-MP}=\langle (I-\alpha_t H_{GTD})g_{GTD}, H_{GTD}-\alpha_t H^2_{GTD}\rangle$. In a similar fashion, one can derive the GTD2-MP algorithm as the PC discretization of GTD2 algorithm.\par

%%!TEX root =  lsa.tex
\begin{abstract}
Linear stochastic approximation (LSA) algorithms arise quite naturally in various applications such linear least-squares problem, solution to large scale linear systems and temporal difference learning algorithms. LSA schemes are often employed to solve for a desired parameter from noisy observations. In this paper, we are in the setting of LSA algorithms that employ a constant stepsize with iterate averaging under the presence of multiplicative noise and are interested in studying the mean-squared error (MSE) of the averaged iterates with respect to the desired solution. Our study is motivated by the recent results for an important sub-class of our setting namely linear least-squares problem  wherein ``amazing'' properties such as instance independent choice for the stepsize and instance independent rate of convergence of the MSE have been demonstrated. In this paper, we ask the question whether these ``amazing" properties hold in the general setting that we consider.\par
We show that in the setting considered stepsizes cannot be chosen in an instance independent fashion. Further, we show that while a rate of $O(\frac{1}{t})$ can be obtained for the MSE the constants are problem specific. Thus we observe that the ``amazing" properties that hold for the linear least-squares problem do not hold in general. On the positive side, constant stepsize with iterate averaging is still an improvement over diminishing stepsize schemes which can yield only a $O(\frac{1}{t^\mu})$ ($\mu<1$) rate for the MSE.
    
\end{abstract}


%\section{Problem Setup}
A Linear Stochastic Approximation (LSA) algorithm is given by
\begin{align}\label{lsa}
\theta_{t+1}=\theta_t+\alpha_t(L_t(\theta)),
\end{align}
where $\{\alpha_t>0,t\geq 0\}$ is a sequence of step-sizes and $L_t\in \R^n$ is a noisy sample of some function $L(\theta)$. Without loss of generality one can assume that $L_t(theta)=g_t-H_t\theta$, where $g_t\in\R^n$ and $H_t\in \R^n$ are noisy samples of some $g\in \R^n$ and $H\in \R^n$ (note that $L=g-H\theta$). We wish to study the convergence of $\theta_t$ to a $\ts\in\R^n$ such that the expected update $\E[L_t(\ts)]=L(\ts)=0$. Depending on the application, such $\ts$ can either be a fixed point or extrememum. In order to understand the error $e_t\eqdef\theta_t-\ts$, we can look at is dynamics as follows:
\begin{align}\label{lser}
\theta_{t+1}-\ts&=\theta_t-\ts+\alpha_t(g_t- H_t (\theta_t+\ts-\ts)),\nn\\
e_{t+1}&=(I-\alpha_t)e_t+\alpha_t M_{t+1},
\end{align}
where $M_{t+1} is$ some noise function. We call the recursion in \eqref{lser} as a linear stochastic error recursion (LSER) and we now proceed to study the same under the following assumptions.
\begin{assumption}\label{fpassump}
\begin{enumerate}[leftmargin=*]
\item $\alpha_t=\alpha>0,~\forall t\geq 0$.
\item $H_t=H+N_{t+1}$ such that $\{N_t\}$ and $\{M_t\}$ are martingale difference sequences measurable with respect to an increasing family of $\sigma$-fields $\mathcal{F}_{t}\stackrel{\cdot}{=}\sigma(x_0,M_1,N_1\ldots,N_t,M_t),t\geq 0$, such that $\E[N_{t+1}|\F_t]=\E[M_{t+1}|\F_t]=0$, $\E[\parallel M_{t+1} \parallel^2|\F_t]\leq \sigma^2, \E[\parallel N_{t+1} \parallel^2|\F_t]\leq \sigma^2, ~\forall t\geq 0$ for some variance $\sigma>0$.
\end{enumerate}
\end{assumption}
Notice that once a constant step-size is chosen the noise term in \eqref{lser} is $\alpha M_{t+1}$, i.e., there is constant addition of noise each iteration, a scenario under which we cannot hope the iterates to converge to $\ts$. However, we can look at the behavior of the average of the iterates, formally known as Rupert-Polyak averaging defined below.
\begin{align}\label{erp}
\begin{split}
\tb_{t}=\frac{1}{t}\overset{t-1}{\underset{s=0}{\sum}}\theta_t
\bar{e}_{t}=\frac{1}{t}\overset{t-1}{\underset{s=0}{\sum}}e_t
\end{split}
\end{align}
We are interested in the mean squared error given by
\begin{align}\label{errfun}
\err(t)&= \E[\parallel \eb_t\parallel^2]
%&=\B(t)+\V(t),
\end{align}
\subsection{Error Analysis - Bias, Variance and Product of Random Matrices}
We now expand $\err(t)$ to understand the structure of LSERs. For the purpose of our analysis, we define the product of random matrices $F_{j,i}\eqdef\Pi_{t=i}^{j} (I-\alpha H_t), \forall j\geq i$ and $F_{j,i}\eqdef I,~\forall i<j$.
\begin{align}
<\eb_t,\eb_t>=\frac{1}{t^2}\E <\overset{t-1}{\underset{i=0}{\sum}}e_i ,\overset{t-1}{\underset{i=0}{\sum}} e_i>,
\end{align}
Then we have from \eqref{lser} that for any $j>i$,
\begin{align}
e_{j+1}=F_{j,i}e_i + \alpha\sum_{k=i}^j F_{j,k+1}M_{k+1}
\end{align} and hence notice that
\begin{align}
\overset{t-1}{\underset{i=0}{\sum}}e_i=\overset{t-1}{\underset{i=0}{\sum}} F_{i,0}e_0+\alpha \overset{t-1}{\underset{i=0}{\sum}} \overset{i}{\underset{k=0}{\sum}} F_{i,k+1}M_{k+1}\end{align} Thus the error can be naturally split into two terms namely $\err(t)=\B(t)+\V(t)$, where $\B(t)$ is the \emph{bias} term which satisfies $\err(t)=\B(t)$ when $M_{t+1}=0~\forall t\geq 0$ and $e_0\neq 0$,  and, $\V(t)$ is the variance term which satisfies $\err(t)=\V(t)$ when $e_0=0$ and $M_{t+1}\neq 0$.
%Now, for any $j>i$ we have $<e_i,e_j>=<e_i,F_{j,i}e_i+\sum_{k=i}^j F_{j,k+1}M_{k+1}>$
\begin{lemma}
For any $x_i\in \R^n$ that is $\F_i$ measurable and $\forall ~j \geq k> i$ it follows that $\E[x_i^\top F_{j,k+1}M_{k+1}]=0$
\end{lemma}
\begin{lemma}
$\E <e_i,F_{j,i}e_i>=(I-\alpha H)^{j-i}e_i$
\end{lemma}
\begin{remark}
$<e_j,e_j>=<F_{j,i}e_i+\sum_{k=i}^j F_{j,k+1}M_{k+1},F_{j,i}e_i+\sum_{k=i}^j F_{j,k+1}M_{k+1}>$, which involves expectation of product of random matrices of the form
$
\E[(I-\alpha H_i)^\top\ldots (I-\alpha H_j)^\top (I-\alpha H_j)\ldots(I-\alpha H_i)]
$.
\end{remark}
\begin{assumption}
Let $\rho_{s}(\alpha)\eqdef \underset{{x\in \R^n}}{\sup}x^\top (2H-\alpha E[H_t^\top H_t])x $ and $\rho_d\eqdef\parallel H \parallel$ (where $\parallel \cdot\parallel$ is the operator norm of the matrix). Then we assume that exist an $\alpha_{\max}$ such that $~\forall \alpha\in (0,\alpha_{\max})$
\begin{enumerate}[leftmargin=*]
\item $2H- \alpha E[H_t^\top H_t] >0$.
\item $|1-\rho_s(\alpha)|<1$.
\item $|1-\alpha\rho_d|<1$.
\end{enumerate}
\end{assumption}
\begin{theorem}
For $\alpha\in(0,\alpha_{\max})$ and $H$ non-symmetric
\begin{align}
\err(t)\leq &\frac{(\rho_s\alpha)^{-1}(1+2(\rho_d\alpha)^{-1})}{t^2}\nn\\&+\frac{\alpha^2(\alpha\rho_s)^{-1}(1+2(\rho_d\alpha)^{-1})}{t}
\end{align}
\end{theorem}
\comment{\begin{theorem}\label{sym}
For $\alpha\in(0,\alpha_{\max})$ and $H$ symmetric
\begin{align}
\err(t)\leq &\frac{(\rho_s\alpha)^{-1}(2(\rho_d\alpha)^{-1})}{t^2}\nn\\&+\frac{\alpha^2(\alpha\rho_s)^{-1}(2(\rho_d\alpha)^{-1})}{t}
\end{align}
\end{theorem}
}
\begin{proof}
\begin{align*}
&\E<\overset{t-1}{\underset{i=0}{\sum}}e_i ,\overset{t-1}{\underset{i=0}{\sum}} e_i>\nn\\
=&\E < \overset{t-1}{\underset{i=0}{\sum}} F_{i,0}e_0+\alpha \overset{t-1}{\underset{i=0}{\sum}} \overset{i}{\underset{k=0}{\sum}} F_{i,k+1}M_{k+1},\nn\\ &\overset{t-1}{\underset{i=0}{\sum}} F_{i,0}e_0+\alpha \overset{t-1}{\underset{i=0}{\sum}} \overset{i}{\underset{k=0}{\sum}} F_{i,k+1}M_{k+1} >
\end{align*}
Bias term $\E < \overset{t-1}{\underset{i=0}{\sum}} F_{i,0}e_0, \overset{t-1}{\underset{i=0}{\sum}} F_{i,0}e_0>$
\begin{align*}
=\E\overset{t-1}{\underset{i=0}{\sum}}<F_{i,0}e_0,F_{i,0}>+2 \E\overset{t-2}{\underset{i=0}{\sum}}\overset{t-1}{\underset{j\geq i}{\sum}}<F_{i,0}e_0, F_{j,0}e_0>
\end{align*}
Now we know that for $j>i$ $\E<F_{i,0}e_0, F_{j,0}e_0>\leq (1-\alpha \rho_d)^{j-i}\E<F_{i,0}e_0,F_{i,0}e_0>$. Thus
\begin{align*}
&\E\overset{t-2}{\underset{i=0}{\sum}}\overset{t-1}{\underset{j\geq i}{\sum}}<F_{i,0}e_0, F_{j,0}e_0>\nn\\
&\leq \E\overset{t-2}{\underset{i=0}{\sum}}\overset{\infty}{\underset{j\geq i}{\sum}} (1-\alpha \rho_d)^{j-i} <F_{i,0}e_0, F_{i,0}e_0>\nn\\
&\leq (\alpha\rho_d)^{-1} 2 \E<F_{i,0}e_0, F_{i,0}e_0>
\end{align*}
Now
\begin{align*}
&\E\overset{t-1}{\underset{i=0}{\sum}}<F_{i,0}e_0,F_{i,0}e_0>\nn\\
&\leq\E\overset{\infty}{\underset{i=0}{\sum}}<F_{i,0}e_0,F_{i,0}e_0>\nn\\
&\leq\E\overset{\infty}{\underset{i=0}{\sum}}(1-\rho_s(\alpha))^i <e_0,e_0>\nn\\
&\leq \rho_s(\alpha)^{-1} \parallel e_0\parallel^2\nn
\end{align*}


%\begin{align*}
%\E < \overset{t-1}{\underset{i=0}{\sum}} F_{i,0}e_0, \overset{t-1}{\underset{i=0}{\sum}} F_{i,0}e_0>
%\end{align*}
\comment{The proof is on similar lines of \cite{}
$
\E<\overset{t-1}{\underset{i=0}{\sum}}e_i ,\overset{t-1}{\underset{i=0}{\sum}} e_i>=\E\overset{t-1}{\underset{i=0}{\sum}} <e_i,e_i>+ 2\E\overset{t-2}{\underset{i=0}{\sum}}\overset{t-1}{\underset{j\geqi}{\sum}} <e_i,e_{j+1}>
$. Now,

\begin{align*}
\E\overset{t-2}{\underset{i=0}{\sum}}\overset{t-1}{\underset{j\geq i}{\sum}} <e_i,e_{j+1}>\nn\\
=\E\overset{t-2}{\underset{i=0}{\sum}}\overset{t-1}{\underset{j\geq i}{\sum}} <e_i,F_{j,i}e_i + \alpha\sum_{k=i}^{j} F_{j,k+1}M_{k+1}>\nn\\
=\E\overset{t-2}{\underset{i=0}{\sum}}\overset{t-1}{\underset{j\geq i}{\sum}} <e_i,(I-\alpha H)^{j-i}e_i >\nn\\
=\E\overset{t-2} <e_i,\alpha^{-1}H^{-1}[(I-\alpha H)-(I-\alpha H)^{t-i}e_i >\nn\\
=\E\overset{t-2} (\alpha H)^{-1}<e_i,e_i > - <e_i,e_i >- <e_i,\alpha^{-1}H^{-1}[(I-\alpha H)-(I-\alpha H)^{t-i}e_i >\nn\\
\end{align*}
}
\end{proof}
\section{LSAs to solve linear system of equations}
In what follows, we assume that we have access to the stochastic orcales which provide $i.i.d$ samples of $A$ and $b$. We now look at the three different LSAs to solve \eqref{linsys}.
\subsection{Fixed Point Linear Stochastic Approximation (FP-LSA)}
We call the following stochastic recursion to compute $\theta^*$ a FP-LSA algorithm
\begin{align}\label{fplsa}
\theta_{t+1}=\theta_t+\alpha(b_t-A_t),
\end{align}
where we assume that
\begin{assumption}
\begin{enumerate}[leftmargin=*]
\item $b_t$ and $A_t$ are $i.i.d$ samples of $b$ and $A$
\item \label{pd}$\underset{x\in\R^n}{\inf} x^\top A x>0$.
\end{enumerate}
\end{assumption}
Notice that
We now present the LSER corresponding the FP-LSA in \eqref{fplsa}.
\begin{align}\label{fplser}
e_{t+1}=(I-\alpha A_t)e_t+\alpha(M^{(2)}_{t+1}-M^{(1)}_{t+1}\ts),
\end{align}
where $M^{(1)}_{t+1}=A_t-A$, $M^{(2)_{t+1}}=b_t-b$,  and $\ts=A^{-1}b$. Now picking $\sigma=\sigma_1^2+\sigma^2_2 \parallel\ts\parallel^2$ it follows that $\E[\parallel M_{t+1} \parallel^2|\F_t]<\sigma^2$ results of \Cref{nonsym} applies to the LSER in \eqref{fplser}.
\comment{The error term splits into $\err(t)= \B(t)+\V(t)$, (we have dropped $\Sigma$ for the sake of brevity) where $\B(t)$ is the bias term which satisfies $\err(t)=\B(t)$ when $M^{(i)}_{t+1}=0,i=1,2,~\forall t\geq 0$ and $x_0\neq x^*$ and $\V(t)$ is the variance term which satisfies $\err(t)=\V(t)$ when $x_0=x^*$ and $M^{(i)}_{t+1}\neq 0$.}
\subsection{Least-Squares Linear Stochastic Approximation (LS-LSA)}
In general \Cref{pd} need not hold and we would still wish to solve \eqref{linsys}. One way to ensure \Cref{pd} is to ensure that the system is positive definite which can be achieved by solving for $A^TA\theta^*=A^\top b$. We call the following stochastic recursion to compute $\theta^*$ a LS-LSA algorithm
\begin{align}\label{lslsa}
\theta_{t+1}=\theta_t+A^{(1)\top}_t(b_t-A^{(2)}_t\theta_t),
\end{align}
where we assume
\begin{assumption}\label{lsassump}
$b_t$, $A^{(1)}_t$and $A^{(2)}_t$ are $i.i.d$ samples of $b$ and $A$.
\end{assumption}
The LSA in \eqref{lslsa} is a stochastic gradient since the $\nabla_\theta \parallel b-A\theta\parallel^2=A^\top(b-A\theta)$. The LSER corresponding to LS-LSA in \eqref{lslsa} is given by
\begin{align}
e_{t+1}=&(I-\alpha A^{(1)\top}_tA^{(2)}_t)e_t+\alpha(A^\top M^{(3)}_{t+1}+M^{(1)\top}_{t+1}M^{(3)}_{t+1}\nn\\&-A^\top M^{(2)}_{t+1}\ts-M^{(1)\top}_{t+1}M^{(2)}_{t+1}\ts),
\end{align}
where $M^{(1)}_{t+1}=A^{(1)}_t-A$, $M^{(2)}_{t+1}=A^{(2)}_t-A$ and $M^{(3)}_{t+1}=b_t-b$. Choosing $\sigma=\parallel A\parallel^2 (\sigma_2^2+\sigma_1^2\parallel \ts\parallel^2)+ \sigma_1^4(1+\parallel \ts\parallel^2)$ the results of \Cref{sym} applies to the LSER in \eqref{lslser}.

\subsection{Saddle-Point Linear Stochastic Approximation (SP-LSA)}
An important downside of \Cref{lslsa} is that two independent samples of $A$ are available in the form of $A^{(1)\top}_t$ and $A^{(2)}_t$ which might be possible in certain applications. This issue can be alleviated by looking at the following Saddle-Point objective function
\begin{align}
L(\theta,y)=\min_{\theta\in \R^n}\max{y\in \R^n}<b-A\theta,y>-\frac{1}{2}\parallel y\parallel_M ^2
\end{align}
By noticing the fact that $\nabla_\theta L =-A^\top y$ and $\nabla_{y}L=-A\theta-M$, we now specify the SP-LSA algorithm as follows:
\begin{align}\label{splsa}
\begin{split}
y_{t+1}&=y_t+\alpha (b_t-A_t\theta_t- M y_t)\\
\theta_{t+1}&=\theta_t+\alpha(A_t^\top)y_t
\end{split}
\end{align}
The SLER corresponding to the SP-LSA in \eqref{splsa} is given by
\begin{align}
e_{t+1}=\Big(\begin{bmatrix} I & 0 \\ 0 &I\\\end{bmatrix} -\begin{bmatrix} M & A_t \\ -A_t^\top &0\\\end{bmatrix}\Big)e_t+\Big(\begin{bmatrix} M^{(2)}_{t_1}-M^{(1)}_{t+1}\ts  \\ 0\\\end{bmatrix}\Big),
\end{align}
where $M^{(1)}_{t+1}=A_t-A$, $M^{(2)_{t+1}}=b_t-b$,  and $\ts=A^{-1}b$.
Let $x_t=(y_t,\theta_t)$, $g_t=\begin{bmatrix} b_t\\ 0\\\end{bmatrix}$ and $G^{(i)}_t=\begin{bmatrix} M & A^{(i)}_t \\ -A^{(i)\top}_t &0\\\end{bmatrix}$ for $i=1,2$, then consider the following updates
\begin{align}
x'_{t}&=x_t+\alpha (g_t-G^{(1)}_t x_t)\\
x_{t+1}&=x_t+\alpha(g_t- G^{(2)}_t x'_t);
\end{align}
%\subsection{Related Work}

%\subsection{Maximum Allowable step size}

%\input{mainresults}
%\input{background}
%\input{ana}
%\input{relwork}
%\input{stab}
%\input{Discussion}
\if0
\begin{table*}[t]
\begin{center}
%\resizebox{1\textwidth}{!}{
\begin{tabular}{|c|c|c|c|c|}\hline
Matrix   & $\alpha$             &   $C$     & Noise     &   Theorem \\ \hline
AS       & $(0,\alpha_{as})$    &   $H$     & AUS        &   \Cref{asaus}\\    \hline

PS       & $(0,\alpha_{ps})$    &   $H$     & AUS        &   \Cref{psaus}\\ \hline

SPDS    & $(0,\alpha_{ps})$     &   $H$     & AUS        &   \Cref{spdsaus}\\ \hline
SPDS    & $(0,\alpha_{ps})$     &   $H^{\frac{1}{2}}$     &AS        &   \Cref{spdsas}\\ \hline

SPDS    & $(0,\alpha_{ps})$     &   $H^{\frac{1}{2}}$     & AUS        &   \Cref{spdsaushalf}\\ \hline

AS       & $(0,\alpha_{as})$    &   $I$     & AUS        &   \Cref{blanket}\\    \hline
PS       & $(0,\alpha_{ps})$    &   $I$     & AUS        &   \Cref{blanket}\\ \hline

SPDS    & $(0,\alpha_{ps})$     &   $I$     & AUS        &   \Cref{blanket}\\ \hline
SPDS    & $(0,\alpha_{ps})$     &   $I$     & AS        &   \Cref{blanket}\\ \hline
PS       & $(0,\alpha_{ps})$    &   $I$     & MUS        &   \Cref{psmus}\\ \hline
SPDS    & $(0,\alpha_{ps})$     &   $H$     & MUS        &  \Cref{spdsm}\\ \hline
SPDS    & $(0,\alpha_{ps})$     &   $H^{\frac{1}{2}}$     & MS        &   \Cref{spdsms}\\ \hline
SPDS    & $(0,\alpha_{ps})$     &   $H^{\frac{1}{2}}$     & MUS        &   \Cref{spdsmus}\\ \hline
\end{tabular}
%}
\end{center}
\caption{Main Result}
\label{maintable}
\end{table*}

\onecolumn
\section{Supplementary Material}
\begin{lemma}\label{addstep}
\begin{enumerate}[label=(\roman*)]
\item For any $H$ that is AS there exists an $\alpha_{as}>0$ for which it holds that $\rho(F_D(\alpha))<1,~\forall \alpha\in (0,\alpha_{as})$.
\item For any $H$ that is PS there exists an $\alpha_{ps}>0$ for which it holds that $\E[x^\top(H-\alpha H_t^\top H_t)x|\F_{t-1}]> 0$, $~\forall \alpha\in (0,\alpha_{ps})$.
\end{enumerate}
\end{lemma}
\begin{proof}
Let $\mu=a+i b$ be any Eigen value of $H$. Since $H$ is AS, we know that $a>0$, and that the Eigen value of $I-\alpha H$ is given by $\rho=1-\alpha a -i\alpha b$. The modulus is then given by
\begin{align*}
|\rho| =\sqrt{(1-\alpha a)^2+b^2}=\sqrt{1-2\alpha a +\alpha^2 (a^2+b^2)}
\end{align*}
Now $\rho<1$ can be achieved by ensuring $\alpha <\frac{2a}{a^2+b^2}$.
Since $H$ is PS, the result follows by choosing $\alpha<\frac{\us{\inf}{\norm{x}\leq 1} x^\top H x}{\us{\sup}{\norm{x}\leq 1}x^\top \E[H_t^\top H_t|\F_{t-1}]x}$.
\end{proof}


We begin by stating a result related to norms of matrices
\begin{lemma}
Let $A\in \R^{n\times n}$ be any real matrix then $\norm{Ax}^2_2\leq \us{\sup}{i,j}n^2|(A)_{ij}|^2\norm{x}^2_2$ and $\ip{x,Ax}\leq \us{\sup}{i,j}n|(A)_{ij}|\norm{x}^2_2$, where $(A)_{ij}$ denotes the $ij^{th}$ entry of matrix $A$.
\end{lemma}
\begin{proof}
For given $x\in \R^n$, let $y=(y_1,\ldots,y_n)\in \R^n$ such that $y=Ax$. Now $y_i=\ip{x,y}\ous{\sum}{j=1}{n} (A)_{ij}x_j\leq \us{\sup}{i,j}|(A_{ij})| \norm{x}_1$, where $\norm{\cdot}_1$ denotes the $L_1$ norm.
\begin{align*}
\ip{x,Ax}&=\ip{x,y}\ous{\sum}{i=1}{n} x_iy_i\\
&\leq\ous{\sum}{i=1}{n} |x_i||y_i|\\
&\leq \us{\sup}{i,j}|(A_{ij})| \norm{x}^2_1\\
&\leq \us{\sup}{i,j}|(A_{ij})| n\norm{x}^2_2
\end{align*}
\begin{align*}
\ip{y,y}&\leq n(\us{\sup}{i,j}|(A_{ij})| \norm{x}_1)^2\\
&\leq \us{\sup}{i,j}|(A_{ij})|^2 n^2 \norm{x}^2_2
\end{align*}
\end{proof}

\begin{lemma}
For $\alpha \in (0,\alpha_{as})$ there exists constants $B_1>0$ and $B_2>0$ such that $B_1=\us{\us{\sup}{i,j}}{t\geq 0}|(F^t)_{ij}|$ and $B_2=\us{\us{\sup}{i,j}}{t\geq 0}|(F^{t\top}F^t)_{ij}|$
\end{lemma}
\begin{proof}
Consider the Jordon decomposition of $F=P\Lambda P^{-1}$. The result follows from noting that, since $\rho(F)<1$ for any $\alpha\in (0,\alpha_{as})$, we have $\us{\lim}{t\ra \infty}F^t=\us{\lim}{t\ra \infty} P\Lambda^t P^{-1} =0$ and $\us{\lim}{t\ra \infty}F^{t\top}F^t=0$.
\end{proof}
\begin{align}\label{addexp}
\eb_t&=\frac{1}{t+1}(\ous{\sum}{i=0}{t} F_{i,1} e_0+ \alpha\ous{\sum}{i=1}{t} \ous{\sum}{k=i}{t} F_{k,i+1}  \zeta_i )
\end{align}
\textbf{Additive Noise:}
In the case when $M_t=0,\forall t\geq 1$, we have $~\forall i\geq j, F_{i,j}=F^{i-j}=(I-\alpha H)^{i-j}$.
\begin{lemma}
For $i\neq j$ $\E[\ip{(I-(I-\alpha H)^{t-i}) \zeta_i, (I-(I-\alpha H)^{t-i}) \zeta_j}]=0$
\end{lemma}
\begin{proof}
W.l.o.g $i>j$
\begin{align}
\E[\ip{(I-(I-\alpha H)^{t-i}) \zeta_i, (I-(I-\alpha H))^{t-i}) \zeta_j}]&= \E\big[\E[\zeta_i^\top (I-(I-\alpha H))^{t-i\top} (I-(I-\alpha H)^{t-i} \zeta_j|\F_{i-1}]\big]\\
&= \E\big[\E[\zeta_i^\top|\F_{i-1}] (I-(I-\alpha H)^{t-i\top} (I-(I-\alpha H)^{t-i} \zeta_j\big]\\
&=0
\end{align}
\end{proof}

\begin{theorem}
When $H$ is AS, $\alpha \in (0,\alpha_{as})$, we have
$\E[\norm{H\eb_t}^2]\leq \Big(\frac{\norm{\theta_0-\ts}^2_2}{\alpha^2(t+1)^2}\Big)\big(1+2nB_1+n^2B_2\big)+\Big(\frac{\sigma^2}{t+1}\Big)\big(1+2nB_1+n^2B_2\big)$
\end{theorem}
\begin{proof}
Since $\alpha\in (0,\alpha_{as})$, we have from \eqref{addexp}
\begin{align*}
H\eb_t&=H\Big(\frac{1}{t+1}(\ous{\sum}{i=0}{t} F_{i,1} e_0+ \alpha\ous{\sum}{i=1}{t} \ous{\sum}{k=i}{t} F_{k,i+1}  \zeta_i )\Big)\\
&=H\Big(\frac{1}{t+1}\big((\alpha H)^{-1}(I-(I-\alpha H)^t)e_0+ \alpha \alpha^{-1}H^{-1}\ous{\sum}{i=1}{t}(I-(I-\alpha H)^{t-i}) \zeta_i\big)\Big)\\
&=\frac{1}{t+1}\big((\alpha)^{-1}(I-(I-\alpha H)^t)e_0+ \ous{\sum}{i=1}{t}(I-(I-\alpha H)^{t-i}) \zeta_i\big)
\end{align*}
Now
\begin{align*}
&\E[\norm{H\eb_t}^2]\\
&=\E\ip{H\eb_t,H\eb_t}\\
&=\frac{1}{\alpha^2(t+1)^2}\E\ip{(I-(I-\alpha H)^t)e_0,(I-(I-\alpha H)^t)e_0}+\frac{1}{(t+1)^2}\ous{\sum}{i=1}{t}\E\ip{(I-(I-\alpha H)^{t-i}) \zeta_i,(I-(I-\alpha H)^{t-i}) \zeta_i}
\end{align*}
Note that for $x\in \R^n$, $\ip{(I-(I-\alpha H)^t)x,(I-(I-\alpha H)^t)x}\leq \ip{x,x}+2\ip{x,(I-\alpha H)^tx}+\ip{(I-\alpha H)^t x,(I-\alpha H)^t x}$. Thus it follows that
\begin{align*}
&\E[\norm{H\eb_t}^2]\leq \frac{\norm{\theta_0-\ts}^2}{(t+1)^2} (1+2nB_1+n^2B_2)+\ous{\sum}{i=1}{t}\frac{\sigma^2}{(t+1)^2}(1+2nB_1+n^2B_2)
\end{align*}
\end{proof}


\begin{theorem}
When $H$ is PS, $\alpha \in (0,\alpha_{ps})$, we have
$\E[\norm{H\eb_t}^2]\leq \Big(\frac{\norm{\theta_0-\ts}^2_2}{\alpha^2(t+1)^2}\Big)\big(1+2(1-\alpha\frac{\rho_H}{2})^t+(1-\alpha\rho_H)^t\big)+\Big(\frac{\sigma^2}{t+1}\Big)\big(1+5\alpha^{-1}{\rho_H}^{-1}\big)$
\end{theorem}
\begin{proof}
Note that $\ip{x,(I-\alpha H)^t,x}\leq (1-\alpha\frac{\rho_H}{2})^t \ip{x,x}$, $\ous{\sum}{i=1}{t} 2 (1-\alpha \frac{\rho_H}{2})^{t-i}+(1-\alpha {\rho_H})^{t-i}\leq \ous{\sum}{i=1}{\infty} 2 (1-\alpha \frac{\rho_H}{2})^{t-i}+(1-\alpha {\rho_H})^{t-i}=5\alpha^{-1}\rho_H^{-1}$
\end{proof}

\begin{theorem}
When $H$ is SDPS, $\alpha \in (0,\alpha_{ps})$, we have
$\E[\norm{H\eb_t}^2]\leq \frac{\norm{\theta_0-\ts}^2_2}{\alpha^2(t+1)^2}+\frac{\sigma^2}{t+1}$
\end{theorem}
\begin{proof}
Note that for $x\in \R^n$, $\ip{(I-(I-\alpha H)^t)x,(I-(I-\alpha H)^t)x}\leq \ip{x,x}$
\end{proof}

\begin{theorem}
When $H$ is SDPS and the noise is structured, $\alpha \in (0,\alpha_{ps})$, we have
$\E[\norm{H^{\frac{1}{2}}\eb_t}^2]\leq \frac{\norm{H^{-\frac{1}{2}}\theta_0-\ts}^2_2}{\alpha^2(t+1)^2}+\frac{\sigma^2}{t+1}$
\end{theorem}
\begin{proof}
\begin{align*}
&\E[\norm{H^{\frac{1}{2}}\eb_t}^2]\\
&=\frac{1}{\alpha^2(t+1)^2}\E\ip{H^{-\frac{1}{2}}(I-(I-\alpha H)^t)e_0,H^{-\frac{1}{2}}(I-(I-\alpha H)^t)e_0}+\frac{1}{(t+1)^2}\ous{\sum}{i=1}{t}\E\ip{H^{-\frac{1}{2}}(I-(I-\alpha H)^{t-i}) \zeta_i,H^{-\frac{1}{2}}(I-(I-\alpha H)^{t-i}) \zeta_i}
\end{align*}
Now
\begin{align*}
E\ip{H^{-\frac{1}{2}}(I-(I-\alpha H)^{t-i}) \zeta_i,H^{-\frac{1}{2}}(I-(I-\alpha H)^{t-i}) \zeta_i}&\leq \E\ip{H^{-\frac{1}{2}}\zeta_i,H^{-\frac{1}{2}}\zeta_i}\\
&=\E trace(H^{-1}\zeta_i\zeta_i^\top)\\
&\leq \sigma^2
\end{align*}
\end{proof}


\begin{theorem}
When $H$ is SDPS and the noise is unstructured, $\alpha \in (0,\alpha_{ps})$, we have
$\E[\norm{H^{\frac{1}{2}}\eb_t}^2]\leq \norm{H^{-1}} \Big(\frac{\norm{\theta_0-\ts}^2_2}{\alpha^2(t+1)^2}+\frac{\sigma^2}{t+1}\Big)$
\end{theorem}
\begin{proof}
\begin{align*}
&\E[\norm{H^{\frac{1}{2}}\eb_t}^2]\\
&=\frac{1}{\alpha^2(t+1)^2}\E\ip{H^{-\frac{1}{2}}(I-(I-\alpha H)^t)e_0,H^{-\frac{1}{2}}(I-(I-\alpha H)^t)e_0}+\frac{1}{(t+1)^2}\ous{\sum}{i=1}{t}\E\ip{H^{-\frac{1}{2}}(I-(I-\alpha H)^{t-i}) \zeta_i,H^{-\frac{1}{2}}(I-(I-\alpha H)^{t-i}) \zeta_i}
\end{align*}
\end{proof}



\begin{lemma}
For any $x_{i-1}\in \R^n$ that is $\F_{i-1}$ measurable and $\forall ~t > i$ if follows that $\E[x_i^\top F_{t,i+1}\zeta_{i}]=0$
\end{lemma}
\small
\begin{proof}
\begin{align*}
&\E[x_t^\top F_{i,t+1}\zeta_{i}]\\
%=\E\big[\E[ x_{i-1}^\top F_{i,t+1}\zeta_{i}|\F_t]\big]\\
=&\E\Bigg[\E\bigg[\E\Big[\E\big[ x_{i-1}^\top F_{t,i+1}\zeta_{i}|\F_{t}\big]|\F_{t-1}\Big]\ldots|\F_{i-1}\bigg]\Bigg]\\
=&\E\Bigg[ x_{i-1}^\top \E\bigg[\E\Big[\E\big[(I-\alpha H_t)|\F_{t}\big]|\ldots \F_{i}\Big]\zeta_{i}|\F_{i-1}\bigg]\Bigg]\\
=&\E\Bigg[ x_{i-1}^\top \E[(I-\alpha H)^{t-i+1}\zeta_{i}|\F_{i-1}]\Bigg]\\
=&\E\Bigg[ x_{i-1}^\top (I-\alpha H)^{t-i}\E\bigg[\zeta_{i}|\F_{i-1}\bigg]\Bigg]\\
=&0
\end{align*}
\end{proof}

\begin{lemma}
For $i\neq j$, $\E[\ip{F_{t,i+1}\zeta_{i},F_{t,j+1} \zeta_{j}}]=0$
\end{lemma}
\begin{proof}
W.l.o.g  $i>j$
\begin{align}
&\E[\ip{F_{t,i+1} \zeta_{i},F_{t,j+1} \zeta_{j}}]\nn\\
\label{diffindexinter}=&\E\big[\E[ \zeta^\top_{i}(I-\alpha H_{i+1})^\top\ldots(I-\alpha H_t)^\top(I-\alpha H_t)\ldots(I-\alpha H_{j+1}) \zeta_{j}|\F_{t-1}]\big]\\
=&\E\big[ \zeta^\top_{i}(I-\alpha H_{i+1})^\top\ldots\E[(I-\alpha H_t)^\top(I-\alpha H_t)|\F_{t-1}]\ldots(I-\alpha H_{j+1}) \zeta_{j}\big]\\
=&\E\big[ \zeta^\top_{i}(I-\alpha H_{i+1})^\top\ldots C\ldots(I-\alpha H_{j+1}) \zeta_{j}\big]\\
\label{diffindex}=&\E[ \sum_{l} G_l],
\end{align}
where $l$ is any index and $G_l$s in \eqref{diffindex} are terms obtained by expanding the product in \eqref{diffindexinter} using the fact that $H_t=H+M_{t},~\forall t\geq 0$. Note that $G_l$ is a product involving powers of $\alpha H$, matrices such as $M^\top_{q+1}$ and $M_{p+1}$ for some $ i\leq p<t, j\leq q < t$ and vectors $\zeta^\top_i$ and $\zeta^\top_j$.
For any given $G_l$ and let $r(l)$ be the largest index such that either $M^\top_{r(l)+1}$ or $M_{r(l)+1}$ or $\zeta^\top_{r(l)+1}$ is present in that term.
\begin{align*}
\E[ \sum_{l} G_l]=  \sum_{l} \E\big[\E[G_l|\F_{r(l)}]\big]=0
\end{align*}
\end{proof}
\begin{theorem}
When $H$ is PS and $\alpha\in (0,\alpha_{ps})$ we have $\E[\norm{H\eb_t}^2]\leq (1+4(\alpha\rho_H)^{-1}) (\alpha\rho_H)^{-1} \Big(\frac{\norm{\theta_0-\ts}^2}{(t+1)^2}+ \frac{\alpha^2\sigma^2}{t+1} \Big)
$
\end{theorem}

\begin{proof}
$\E[\norm{H\eb_t}^2]=\E\ip{H\eb_t,H\eb_t}=\frac{1}{(t+1)^2}\E\ip{H\ous{\sum}{i=0}{t}(\theta_i-\ts),H\ous{\sum}{i=0}{t}(\theta_i-\ts)}$.
$\E\ous{\sum}{i=0}{t}\ous{\sum}{j=0}{t}\ip{H(\theta_i-\theta^*),H(\theta_j-\theta^*)}=\E\ous{\sum}{i=0}{t}\ip{H(\theta_i-\theta^*),H(\theta_j-\theta^*)}+ 2\E\ous{\sum}{i=0}{t-1}\ous{\sum}{j=i+1}{t}\ip{H(\theta_i-\theta^*),H(\theta_j-\theta^*)}$. Now
\begin{align}\label{inter}
&\E\ous{\sum}{i=0}{t-1}\ous{\sum}{j=i+1}{t}\ip{H(\theta_i-\theta^*),H(\theta_j-\theta^*)}\nn\\
&=\E\ous{\sum}{i=0}{t-1}\ous{\sum}{j=i+1}{t}\ip{H(\theta_i-\theta^*),H\big[F_{j,i+1} (\theta_i-\theta^*)+\alpha\sum_{k=i+1}^j F_{j,k+1}\zeta_{k}\big]}\nn\\
&=\E\ous{\sum}{i=0}{t-1}\ous{\sum}{j=i+1}{t}\ip{H(\theta_i-\theta^*),H F_{j,i+1} (\theta_i-\theta^*)}\nn\\
&=\E\ous{\sum}{i=0}{t-1}\ous{\sum}{j=i+1}{t}\ip{H(\theta_i-\theta^*),H(I-\alpha H)^{j-i} (\theta_i-\theta^*)}\nn\\
&=2(\alpha\rho_H)^{-1}\E\ous{\sum}{i=0}{t-1}\ip{H(\theta_i-\theta^*),H (\theta_i-\theta^*)}\nn\\
&\leq2(\alpha\rho_H)^{-1}\E\ous{\sum}{i=0}{t}\ip{H(\theta_i-\theta^*),H (\theta_i-\theta^*)}\nn\\
\end{align}
$\E\ous{\sum}{i=0}{t}\ous{\sum}{j=0}{t}\ip{H(\theta_i-\theta^*),H(\theta_j-\theta^*)}=\E\ous{\sum}{i=0}{t}(1+4(\alpha\rho_H)^{-1})\ip{H(\theta_i-\theta^*),H(\theta_i-\theta^*)}$

Now
\begin{align*}
\E\ip{H(\theta_i-\theta^*),H(\theta_i-\theta^*)}=\E\ip{F_{i,1}e_0,F_{i,1}e_0}+\ous{\sum}{j=1}{i}\E\ip{F_{i,j+1}\zeta_j,F_{i,j+1}\zeta_j}
\leq (1-\alpha\rho_H)^i+ (\alpha\rho_H)^{-1}\sigma^2
\end{align*}
$\E\ous{\sum}{i=0}{t}\ous{\sum}{j=0}{t}\ip{H(\theta_i-\theta^*),H(\theta_j-\theta^*)}\leq (1+4(\alpha\rho_H)^{-1}) (\alpha\rho_H)^{-1}(t\sigma^2+\norm{\theta_0-\ts}^2)$
\begin{align}
\E[\norm{H\eb_t}]\leq (1+4(\alpha\rho_H)^{-1}) (\alpha\rho_H)^{-1} \Big(\frac{\norm{\theta_0-\ts}^2}{(t+1)^2}+ \frac{\alpha^2\sigma^2}{t+1} \Big)
\end{align}
\end{proof}

\begin{theorem}
When $H$ is SPDS and $\alpha\in (0,\alpha_{ps})$ we have $\E[\norm{H\eb_t}^2]\leq \frac{\norm{\theta_0-\ts}^2_2}{\alpha^2(t+1)^2}+\frac{\sigma^2}{t+1}$
\end{theorem}

\begin{proof}
$\E[\norm{H\eb_t}^2]=\E\ip{H\eb_t,H\eb_t}=\frac{1}{(t+1)^2}\E\ip{H\ous{\sum}{i=0}{t}(\theta_i-\ts),H\ous{\sum}{i=0}{t}(\theta_i-\ts)}$.
$\E\ous{\sum}{i=0}{t}\ous{\sum}{j=0}{t}\ip{H(\theta_i-\theta^*),H(\theta_j-\theta^*)}=\E\ous{\sum}{i=0}{t}\ip{H(\theta_i-\theta^*),H(\theta_j-\theta^*)}+ 2\E\ous{\sum}{i=0}{t-1}\ous{\sum}{j=i+1}{t}\ip{H(\theta_i-\theta^*),H(\theta_j-\theta^*)}$. Now
\begin{align}\label{inter}
&\E\ous{\sum}{i=0}{t-1}\ous{\sum}{j=i+1}{t}\ip{H(\theta_i-\theta^*),H(\theta_j-\theta^*)}\nn\\
&=\E\ous{\sum}{i=0}{t-1}\ous{\sum}{j=i+1}{t}\ip{H(\theta_i-\theta^*),H\big[F_{j,i+1} (\theta_i-\theta^*)+\alpha\sum_{k=i+1}^j F_{j,k+1}\zeta_{k}\big]}\nn\\
&=\E\ous{\sum}{i=0}{t-1}\ous{\sum}{j=i+1}{t}\ip{H(\theta_i-\theta^*),H F_{j,i+1} (\theta_i-\theta^*)}\nn\\
&=\E\ous{\sum}{i=0}{t-1}\ous{\sum}{j=i+1}{t}\ip{H(\theta_i-\theta^*),H(I-\alpha H)^{j-i} (\theta_i-\theta^*)}\nn\\
&=\E\ous{\sum}{i=0}{t-1}\ip{H(\theta_i-\theta^*),H(\alpha H)^{-1}[(I-\alpha H)-(I-\alpha H)^{j-i}] (\theta_i-\theta^*)}\nn\\
&\leq \E\ous{\sum}{i=0}{t}\ip{H(\theta_i-\theta^*),H(\alpha H)^{-1}[(I-\alpha H)] (\theta_i-\theta^*)}\nn\\
&\leq \E\ous{\sum}{i=0}{t}\ip{H(\theta_i-\theta^*),(\alpha)^{-1} (\theta_i-\theta^*)}- \E\ous{\sum}{i=0}{t}\ip{H(\theta_i-\theta^*),H(\theta_i-\theta^*)}\nn\\
\end{align}
Now it is enough to bound $(\alpha)^{-1} \E\ous{\sum}{i=0}{t}\ip{H(\theta_i-\theta^*),(\theta_i-\theta^*)}$
\end{proof}

\begin{lemma}
Define for $x\in \R^n$ $f^i_t(x)\eqdef\E[\ip{F_{t,i}x,F_{t,i}x}]$ and $g^i_t(x)\eqdef\E[\ip{F_{t,i}x, H F_{t,i}x}]$. Then we have $g^i_{t-1}(x)\leq \alpha^{-1}(f^i_{t-1}(x)-f^i_t(x))$.
\end{lemma}
\begin{proof}
\begin{align*}
f^i_t=&\E\bigg[\E\Big[\E\big[ x^\top F^\top_{t,i} F_{t,i}x|\F_{t}\big]|F_{t-1}\Big]\ldots\bigg]\\
=&\E\bigg[x^\top\ldots\E\Big[(I-\alpha H_{t-1})^\top\E\big[ (I-\alpha H_t)^\top(I-\alpha H_t)|\F_{t}\big]\\&(I-\alpha H_{t-1})|F_{t-1}\Big]\ldots x\bigg]\\
&\leq \E\bigg[x^\top\ldots\E\Big[(I-\alpha H_{t-1})^\top\big[I-\alpha H\big]\\&(I-\alpha H_{t-1})|F_{t-1}\Big]\ldots x\bigg]\\
&\leq f^i_{t-1}(x)-\alpha g^i_{t-1}(x)
\end{align*}
\end{proof}



\begin{align*}
\E<H(\theta_t-\theta^*),(\theta_t-\theta^*)>&=g^0_t(\theta_0-\theta^*)+\alpha\sum_{k=0}^t g^{k+1}_t(M_{k+1})\\
\sum_{i=0}^t\E<A(\theta_i-\theta^*),(\theta_i-\theta^*)>&=\sum_{i=0}^t g^0_i(\theta_0-\theta^*)+\alpha\sum_{k=0}^t g^{k+1}_i(M_{k+1})\\
&\leq\alpha^{-1}(f^0_0(\theta_0-\ts)-f^0_{t+1}(\theta_0-\ts))+\sum_{k=0}^t \alpha^{-1}(f^k_k(M_{k+1})-f^k_{t+1}(M_{k+1}))\\
&\leq\alpha^{-1}(f^0_0(\theta_0-\ts))+\alpha^{-1}(t+1)\sigma\sum_{k=0}^t \alpha^{-1}(f^k_k(M_{k+1}))\\
&\leq\alpha^{-1}\parallel \theta_0-\ts\parallel^2+\alpha^{-1}(t+1)\sigma^2
\end{align*}
\begin{lemma}
In \eqref{inter} $C_t\leq \frac{t+1}{\rho}{(1-\alpha\rho)^{t+1}}$
\end{lemma}
\begin{proof}
\begin{align*}
\E<A(\theta_i-\theta^*),(\theta_i-\theta^*)>&=g^0_i(\theta_0-\theta^*)+\alpha\sum_{k=0}^i g^{k+1}_i(M_{k+1})\\
\end{align*}
\end{proof}
\comment{
\begin{proof}





\begin{align*}
&<A(\theta_i-\theta^*),(I-\alpha A)^{t-i+1} (\theta_i-\theta^*)>\\
&\leq(1-\alpha \rho)^{t-i+1} <(\theta_t-\theta^*),A (\theta_t-\theta^*)>\\
\end{align*}
\begin{align*}
C_t&=\E\sum_{i=0}^t<A(\theta_i-\theta^*),(I-\alpha A)^{t-i+1} (\theta_i-\theta^*)>\\
&\leq\E\sum_{i=0}^t|<A(\theta_i-\theta^*),(I-\alpha A)^{t-i+1} (\theta_i-\theta^*)>|\\
&\leq \E\sum_{i=0}^t(1-\alpha\rho)^{t-i+1}(1-\alpha\rho)^i\\
&\leq \frac{t+1}{\rho}{(1-\alpha\rho)^{t+1}}\\
&\E<\theta_i-\theta^*,\theta_j-\theta^*>\\
&=\alpha^2\E\sum_{k=1}^i\sum_{j=1}^i <F_{i,j+1}M_{j+1},F_{i,k+1} M_{k+1}>\\
&=\alpha^2\E\sum_{k=1}^i<F_{i,k+1}M_{k+1},F_{i,k+1} M_{k+1}>\\
\end{align*}
\end{proof}
\begin{align*}
&\E[<F_{i,k+1}M_{k+1},F_{i,k+1} M_{k+1}>]\\
&=\E[\big[\Big[\bigg[<F_{i,k+1} M_{k+1},F_{i,k+1} M_{k+1}>|\F_{j-1}\bigg]|F_{j-2}\Big]\ldots|\F_{k+1}\big]]\\
&\leq(1-\alpha\rho)^{i-k-1}\sigma^2\\
&\E<\theta_i-\theta^*,\theta_j-\theta^*>\\
&\leq \frac{\alpha\sigma^2}{\rho}\\
&\E[<F_{i,1}(\theta_0-\theta^*),F_{i,1}(\theta_0-\theta^*)>]\\
&=\E[\bigg[\Big[\big[<F_{i,1}(\theta_0-\theta^*),F_{i,1}(\theta_0-\theta^*)>|\F_{j-1}\big]|F_{j-2}\Big]\ldots|\F_{k+1}\bigg]]\\
&\leq (1-\alpha\rho)^{i-k-1}\parallel \theta_0-\theta^*\parallel\\
&\leq (\rho\alpha)^{-1}\parallel \theta_0-\theta^*\parallel
\end{align*}
}

\fi
% Acknowledgements should only appear in the accepted version. 
%\section*{Acknowledgements}
\nocite{langley00}
\bibliography{ref}
\bibliographystyle{icml2016}
\end{document}
