%%%%%%%%%%%%%%%%%%%%%%%%%%%%%%%%%%%%%%%%%%%%%%%%%%%%%%%%%%%%%%%%%%
%%%%%%%% ICML 2016 EXAMPLE LATEX SUBMISSION FILE %%%%%%%%%%%%%%%%%
%%%%%%%%%%%%%%%%%%%%%%%%%%%%%%%%%%%%%%%%%%%%%%%%%%%%%%%%%%%%%%%%%%

% Use the following line _only_ if you're still using LaTeX 2.09.
%\documentstyle[icml2016,epsf,natbib]{article}
% If you rely on Latex2e packages, like most moden people use this:
\documentclass{article}
%!TEX root =  speedylearn.tex
\usepackage[textsize=tiny]{todonotes}
\setlength{\marginparwidth}{13ex}
\newcommand{\todoc}[2][]{\todo[size=\scriptsize,color=blue!20!white,#1]{Csaba: #2}}
\newcommand{\todoch}[2][]{\todo[size=\scriptsize,color=red!20!white,#1]{Chandru: #2}}
\newcommand{\todox}[2][]{\todo[size=\scriptsize,color=green!20!white,#1]{Xiaowei: #2}}
\newcommand{\norm}[1]{\left\| #1\right\|}
\newcommand{\ip}[1]{\langle#1\rangle}
\newcommand{\ber}{\begin{enumerate}[label=(\roman*)]}
\newcommand{\eer}{\end{enumerate}}
\newcommand{\rr}{\rho^{\alpha}_{R}}
\newcommand{\rd}{\rho^{\alpha}_{D}}
\newcommand{\ld}{\lambda_{D}}
\newcommand{\V}{\mathcal{V}}
\newcommand{\B}{\mathcal{B}}
\renewcommand{\P}{\mathcal{P}}
\renewcommand{\H}{\mathcal{H}}
\usepackage{hyperref}
\hypersetup{
    bookmarks=true,         % show bookmarks bar?
    unicode=false,          % non-Latin characters in AcrobatÕs bookmarks
    pdftoolbar=true,        % show AcrobatÕs toolbar?
    pdfmenubar=true,        % show AcrobatÕs menu?
    pdffitwindow=false,     % window fit to page when opened
    pdfstartview={FitH},    % fits the width of the page to the window
    pdftitle={My title},    % title
    pdfauthor={Author},     % author
    pdfsubject={Subject},   % subject of the document
    pdfcreator={Creator},   % creator of the document
    pdfproducer={Producer}, % producer of the document
    pdfkeywords={keyword1} {key2} {key3}, % list of keywords
    pdfnewwindow=true,      % links in new window
    colorlinks=true,       % false: boxed links; true: colored links
    linkcolor=red,          % color of internal links (change box color with linkbordercolor)
    citecolor=blue,        % color of links to bibliography
    filecolor=magenta,      % color of file links
    urlcolor=cyan           % color of external links
}
\usepackage{times}
\usepackage{natbib}
\usepackage{nicefrac}
\usepackage{wrapfig}

\let\temp\epsilon
\let\epsilon\varepsilon

\usepackage{diagbox}
\usepackage{makecell}
\usepackage{pgfplots}
\usepackage{pgf}
\usepackage{amsmath}
\usepackage{amsthm}
\usepackage{amsfonts}
%\usepackage{comment}
%\renewenvironment{comment}{}{}
\usepackage[capitalize]{cleveref}
\usepackage{enumitem}
\usepackage{tikz}
\usepackage{placeins}
\newcommand{\nn}{\nonumber}
\newcommand{\onp}{\emph{on-policy~}}
\newcommand{\ofp}{\emph{off-policy~}}
\newcommand{\Ra}{\Rightarrow}
\renewcommand{\L}{\emph{L}}
\newcommand{\T}{\mathcal{T}}
\newcommand{\I}{\mathcal{I}}
\newcommand{\D}{\mathcal{D}}
\newcommand{\R}{\mathbb{R}}
\newcommand{\E}{\mathbf{E}}
\newcommand{\F}{\mathcal{F}}
\newcommand{\M}{\mathcal{M}}

\newcommand{\ra}{\rightarrow}
\newcommand{\la}{\leftarrow}
\newcommand{\tdo}{TD$(0)$}
\newcommand{\err}{\emph{err}}
\newcommand{\xb}{\bar{x}}
\newcommand{\xs}{x^*}
\newcommand{\tb}{\bar{\theta}}
\newcommand{\ts}{\theta^*}
\newcommand{\ab}{\bar{\alpha}}
\newcommand{\eqdef}{\stackrel{\cdot}{=}}
\newcommand{\eb}{\bar{e}}
\theoremstyle{plain}
\newtheorem{theorem}{Theorem}[]
\newtheorem{lemma}{Lemma}[]
\newtheorem{corollary}{Corollary}

\theoremstyle{definition}
\newtheorem{assumption}{Assumption}[]
\newtheorem{example}{Example}
\newtheorem{remark}{Remark}
\newtheorem{domain}{Domain}
\newtheorem{condition}{Condition}

\usetikzlibrary{intersections}
\usetikzlibrary{arrows,calc,fit,patterns,plotmarks,shapes.geometric,shapes.misc,shapes.symbols,   shapes.arrows,   shapes.callouts,   shapes.multipart,   shapes.gates.logic.US,   shapes.gates.logic.IEC,   er,   automata,   backgrounds,   chains,   topaths,   trees,   petri,   mindmap,   matrix,   calendar,   folding, fadings,   through,   positioning,   scopes,   decorations.fractals,   decorations.shapes,   decorations.text,   decorations.pathmorphing,   decorations.pathreplacing,   decorations.footprints,   decorations.markings, shadows}
\usetikzlibrary{arrows,calc,shapes, snakes, intersections,patterns,shadows}
\tikzstyle{decision}=[diamond,draw]
\tikzstyle{line}=[draw]
\tikzstyle{elli}=[draw,ellipse]
\tikzstyle{arrow} = [thick]
\newcommand{\us}[2]{\underset{#2}{#1}~}
\newcommand{\ous}[3]{\overset{#3}{\underset{#2}{#1}}~}



% use Times
\usepackage{times}
% For figures
\usepackage{graphicx} % more modern
%\usepackage{epsfig} % less modern
\usepackage{subfigure} 

% For citations
\usepackage{natbib}

% For algorithms
\usepackage{algorithm}
\usepackage{algorithmic}

% As of 2011, we use the hyperref package to produce hyperlinks in the
% resulting PDF.  If this breaks your system, please commend out the
% following usepackage line and replace \usepackage{icml2016} with
% \usepackage[nohyperref]{icml2016} above.
\usepackage{hyperref}

% Packages hyperref and algorithmic misbehave sometimes.  We can fix
% this with the following command.
%\newcommand{\theHalgorithm}{\arabic{algorithm}}

% Employ the following version of the ``usepackage'' statement for
% submitting the draft version of the paper for review.  This will set
% the note in the first column to ``Under review.  Do not distribute.''
\usepackage{icml2016} 

% Employ this version of the ``usepackage'' statement after the paper has
% been accepted, when creating the final version.  This will set the
% note in the first column to ``Proceedings of the...''
%\usepackage[accepted]{icml2016}


% The \icmltitle you define below is probably too long as a header.
% Therefore, a short form for the running title is supplied here:
\icmltitlerunning{Iterate averaging}

\begin{document} 

\twocolumn[
\icmltitle{Iterate Averaging in Linear Stochastic Approximation}

%\icmlauthor{Your Name}{email@yourdomain.edu}
%\icmladdress{Your Fantastic Institute,314159 Pi St., Palo Alto, CA 94306 USA}
%\icmlauthor{Your CoAuthor's Name}{email@coauthordomain.edu}
%\icmladdress{Their Fantastic Institute,27182 Exp St., Toronto, ON M6H 2T1 CANADA}

% You may provide any keywords that you 
% find helpful for describing your paper; these are used to populate 
% the "keywords" metadata in the PDF but will not be shown in the document
\icmlkeywords{
stochastic approximation,
iterate averaging,
Polyak-Rupert,
linear stochastic approximation,
linear stochastic recursions,
finite-time expected error bounds
}

\vskip 0.3in
]
%%!TEX root =  speedylearn.tex
\section{Introduction}
Linear stochastic approximation (LSA) algorithms are popular in computing a desired parameter $\ts\in R^n$ (a fixed point or extremum) from noisy data. A desriable feature of the LSA algorithms is that they make only $O(n)$ linear updates. A typical LSA, at each iteration $t\geq 0$, corrects its iterate $\theta_t\in \R^n$, by making an incremental step, thereby, eventually converging to $\ts$ as the iteration proceeds. Many popular algorithms in machine learning such as stochastic gradient descent and temporal difference learning algorithms arising in reinforcement learning are LSA algorithms.\par
One way to handle noise is to use steps $\alpha_t>0$ that are diminishing with respect to $t$ ($alpha_t \ra 0$ as $t\ra\infty$), an approach that can also lead to poor convergence rate. Alternatively, one can use constant step sizes ($\alpha_t=\alpha,~\forall t\geq 0$), and then average the iterates to filter the noise, an approach known as Rupert-Polyak (RP) averaging. In this paper, we wish to understand LSA algorithms with constant step-sizes and Rupert-Polyak averaging of iterates. In particular, we are interested in the finite time performance the LSA algorithms by studing the mean squared error $\E\parallel \tb_t-\ts \parallel^2$, where $\tb\eqdef\frac{1}{t}\sum_{i=0}{t-1}\theta_t$\par is the RP average of the iterates.
Our work is inspired by a related work in \cite{bachaistats}, in which the authors consider the problem of linear prediction with the penalty function as quadratic loss under an i.i.d  (with respect to some unknown distribution) assumption on the data. The linear prediction problem in \cite{bachaistats} is solved by an stochastic gradient descent (SGD) algorithm which is an LSA algorithm as well. In \cite{bachaistats}, the finite time error is split into two terms namely the bias term (due to the initial condition) and the variance term (due to noise). Rupert-Polyak (RP) averaging is revisited in \cite{bachaistats}, to show that, under a constant step-size rule with iterate averaging, a finite time performance of $O(1/t^2)$ and $O(1/t)$ respectively for the bias and the variance terms can be achieved. An important difference in our case is that unlike the linear prediction setting, the matrices $H_t$ are not symmetric in the RL setting.\par
The motivation for our study is the problem of \emph{value function} estimation in reinforcement learning (RL), wherein we are required to estimate the value function corresponding to a Markov reward process\footnote{Please refer \cite{BertB} for a detailed background}. A particular aspect of the RL setting is that the explicit model of the Markov reward process is not known and only data in the form of samples is made available. The temporal difference (TD) learning algorithms have been a popular choice for value function estimation. The TD algroithms are LSA algorithms as well and we wish to understand the implications of constant steps and RP averaging in the context of the TD algorithms.\par

Our specific contributions in the paper are
\begin{itemize}
\item We present the finite time bound for the LSAs with constant step-size and RP averaging. The bound has two terms namely a bias term due to the inital condition $e_0\neq 0$ and a variance term due the noise process $W_{t}$. We present the results in two parts, first for the general case of non-symmetric $H_t$ and one for the case of symmetric $H_t$.
\item We consider three distinct LSAs to solve a given system of linear equations, namely the \emph{fixed-point} LSA (FP-LSA), the \emph{least-squares} LSA (LS-LSA) and the \emph{saddle-point} LSA (SP-LSA). The FP-LSA is based on the idea fixed points and does not involve gradient type updates.  The LS-LSA and SP-LSA are also stochastic gradient algorithms for the quadratic and saddle point objective functions (the gradients are linear for these objectives). We derive specific bounds for these three LSAs.
\item We discuss the implication of our results in the context of RL.
\comment{first consider LSAs where the expected update $g-H\theta_t$ does not correspond to a gradient of any function which we call \emph{fixed-point} LSA (FP-LSA). We then consider two different Stochastic-Gradient-LSA (SG-LSA) algorithms namely Least-Squares-LSA (LS-LSA) and Saddle-Point-LSA (SP-LSA) based respectively on the idea of minimizing the least square and the saddle point objectives. We present finite time performance analysis of all three cases. Our specific contributions are
\item The conditions that guarantee stability and convergence of FP-LSAs is given by stochastic approximation (SA) theory \cite{}. We show that a finite time performance of $O(1/t^2)$ and $O(1/t)$ continue to hold for the bias and the variance terms of $\err_M(t)$ even in the absence of symmetric operators. A downside is that the LS-LSA algorithm needs two independent samples per iteration which might not be realistic in applications.
\item We point out to the fact that the SP-LSA algorthim can potentially diverge. We fix this by suggesting a modified version of the SP-LSA algorithm based on the idea of \emph{Predictor-Corrector} methods known in numerical analysis. However, this fix relies on the access to two independent samples each iteration.
\item We turn towards the theory of product of random matrices to understand stability of constant step-size LSA. We observe that there exists two regimes for the choice of constant step-size. One being conservative with a smaller constant step size resulting in convergence of iterates in the mean square sense, and the other being relaxed with a relatively larger constant step size guaranteeing only asymptotic stability in high probability.
}
\end{itemize}
\comment{
The finite time performance of the LS-LSA corresponds to the linear least squares setting studied in \cite{}. We have included it here as well for the sake of completeness.
The paradigm of reinforcement learning (RL) \cite{} captures the automated learning of agents to act from sampled obtained via direct interaction with the environment. Formally, an action sequence is known as a policy denoted by $\pi$. An important sub-problem in RL is to learn the value of any given policy $\pi$, which is captured by a score/value function $V^\pi$. It is quite common in practice to have a linear representation of the value function i.e., letting $V^\pi=\Phi \theta$, where $\Phi$ is a feature matrix and $\theta^*$ is a weight vector. Another common scenario in RL, known as \emph{off-policy} learning, is encountered when the agent has collected samples using a \emph{behavior} policy $\pi_b$ and needs to learn the value of a \emph{target} policy $\pi_t$. The trivial scenario when $\pi_b=\pi_t$ is known as \emph{on-policy} learning.\par}

%%%%%%%%%%%%%%%%%%%%%%%%%%%%%%%%%%%%%%%%
\section{Introduction}
%%%%%%%%%%%%%%%%%%%%%%%%%%%%%%%%%%%%%%%%
What is the problem setup:
Iterate averaging in linear stochastic approximation with multiplicative i.i.d. noise.

Why should we care?
SA has wide range of applications in science and engineering. Low-cost alternative,
highly suitable to process large data volumes and work in large dimensions.

Motivation: Linear prediction with squared loss and iid sampling. Bach et al. obtained remarkable result:
there exists a constant step-size 
such that expected squared prediction error
after $n$ updates is at most $C/n$ with a universal constant $C>0$, 
uniformly over all problem instances
such that the magnitude of features vectors is bounded by a known constant.
Why remarkable? No dependence on the conditioning of the underlying system.

Constant stepsize helps intuitively because $\dots$ (add explanation).

Question asked: To what extent can this remarkable result generalized beyond linear prediction 
with squared loss?
One potential application domain is
linear value function approximation in reinforcement learning using temporal difference (TD) learning.
Either experience replay in a batch setting, or solving linear systems using TD-style algorithms.

%%%%%%%%%%%%%%%%%%%%%%%%%%%%%%%%%%%%%%%%
\section{Problem Setup and Result}
%%%%%%%%%%%%%%%%%%%%%%%%%%%%%%%%%%%%%%%%
Linear stochastic approximation.

Define only those quantities that are needed to state the main result.

Example: TD(0) learning (don't get blogged down on explaining where the equations are coming from,
just state them, state that they are a special case of ours).

Main result: Expected loss bound.

Theorem: Blah blah
\todoc[inline]{
Can we do the calculation for the structured and unstructured noise together? 
What should happen is that the condition on the stepsize should have two
parts: One part, that depends on the condition number should disappear
when the structured noise condition is met.
%
Also, can we bound the error in $C$-norm for general positive definite $C$?
}

Proof will follow in \cref{sec:proof}.
Make comment on how and why we depart from the analysis of Bach (add citation).


%%%%%%%%%%%%%%%%%%%%%%%%%%%%%%%%%%%%%%%%
\section{Discussion}
%%%%%%%%%%%%%%%%%%%%%%%%%%%%%%%%%%%%%%%%
Goal: Discuss result, implications, limitations.

Say the good things about the result: Fast rate, small constants.
Better than using decreasing step-size?

We could not eliminate the dependence on the condition number:
The result only holds when the stepsize is smaller than a problem-dependent quantity.

Discuss where this is coming from and why this is hard to avoid.

Our recommendation: Keep track of error; if error is blowing up (larger than a preset value), decrease the stepsize
multiplicatively, reset the parameter vector. After $\log(1/\alpha_0)$ resets, each taking at most BLAH time(?),
the algorithm finds $\alpha<\alpha_0$. Good thing: The dependence on $1/\alpha_0$ is mild.

We do not know whether the condition we have is really necessary,
or a uniform bound with a universal step-size is possible.
Discuss proof attempt: Switching stable linear systems.
No known necessary and sufficient condition for the multiplicative ergodic theorem (cite source).
The existing necessary conditions are too stringent (example).

Why i.i.d. noise? 
Extension to martingale noise.
Extension to Markov noise.

What happens under additive noise?

\section{Related Work}
We place our result into the context of related results.
Compare with papers by Bach. Can our result reproduce theirs?
If not why not? 

Compare with Prashanth and Korda:
Only work known to us to bound finite-time error of LSA,
specifically for TD (special case of our setup,
as far as the form of the updates are concerned).

They use decreasing stepsizes -- they must use decreasing stepsizes
as discussed because they assume Markov noise.
Show their bound here (simplified maybe).
Comment on the main qualities of the bound.
Do they get the same rate?
Do they need the same conditions?
Do they depend on the same constants? No.. reciproc of minimum eigenvalue creeps in.
Why is this a problem? Reciproc of minimum eigenvalue is at least $d$ -- bad for huge-dimensional systems.
But it could be much smaller, too, when system is ill-conditioned.

%%%%%%%%%%%%%%%%%%%%%%%%%%%%%%%%%%%%%%%%
\section{Proof}
%%%%%%%%%%%%%%%%%%%%%%%%%%%%%%%%%%%%%%%%
\label{sec:proof}
%%%%%%%%%%%%%%%%%%%%%%%%%%%%%%%%%%%%%%%%
\section{Conclusions}
Iterate averaging: Extremely powerful in linear prediction under quadratic noise.
Here, power is tested more generally.
BLAH-BLAH

Further work:
TD($\lambda$) -- do the results extend? Why not? What can be done?

GTD -- do the results extend? Why not? What can be done?
%%%%%%%%%%%%%%%%%%%%%%%%%%%%%%%%%%%%%%%%
\if0
%!TEX root =  lsa.tex
\section{Problem Setup}\label{sec:prob}
%\begin{table*}[t]
\begin{center}
\resizebox{1\textwidth}{!}{
\begin{tabular}{|c|c|c|c|c|}\hline
Matrix   & $\alpha$             &   $C$     & Noise     &   $E[\norm{C\eb_t}^2]\leq \B^2+\V^2$\\\hline
AS       & $(0,\alpha_{as})$    &   $H$     & AUS        &   $\B^2=\Big(\frac{\norm{\theta_0-\ts}^2_2}{\alpha^2(t+1)^2}\Big)\big(1+2nB_1+n^2B_2\big),~\V^2=\Big(\frac{\sigma^2}{t+1}\Big)\big(1+2nB_1+n^2B_2\big)$\\    \hline
AS       & $(0,\alpha_{as})$    &   $I$     & AUS        &   $\B^2=\norm{H^{-1}}^2\Big(\frac{\norm{\theta_0-\ts}^2_2}{\alpha^2(t+1)^2}\Big)\big(1+2nB_1+n^2B_2\big),~\V^2=\norm{H^{-1}}^2\Big(\frac{\sigma^2}{t+1}\Big)\big(1+2nB_1+n^2B_2\big)$\\    \hline

PS       & $(0,\alpha_{ps})$    &   $H$     & AUS        &   $\B^2=\Big(\frac{\norm{\theta_0-\ts}^2_2}{\alpha^2(t+1)^2}\Big)\big(1+2(1-\alpha\frac{\rho_H}{2})^t+(1-\alpha\rho_H)^t\big),~\V^2=\Big(\frac{\sigma^2}{t+1}\Big)\big(1+5\alpha^{-1}{\rho_H}^{-1})$\\ \hline
PS       & $(0,\alpha_{ps})$    &   $I$     & AUS        &   $\B^2=\norm{H^{-1}}^2\Big(\frac{\norm{\theta_0-\ts}^2_2}{\alpha^2(t+1)^2}\Big)\big(1+2(1-\alpha\frac{\rho_H}{2})^t+(1-\alpha\rho_H)^t\big),~\V^2=\norm{H^{-1}}^2\Big(\frac{\sigma^2}{t+1}\Big)\big(1+5\alpha^{-1}{\rho_H}^{-1})$\\ \hline

SPDS    & $(0,\alpha_{ps})$     &   $H$     & AUS        &   $\B^2=\frac{\norm{\theta_0-\ts}^2_2}{\alpha^2(t+1)^2},~\V^2=\frac{\sigma^2}{t+1}$\\ \hline
SPDS    & $(0,\alpha_{ps})$     &   $H^{\frac{1}{2}}$     & US        &   $\B^2=\norm{H^{-1}}\frac{\norm{\theta_0-\ts}^2_2}{\alpha^2(t+1)^2},~\V^2=\norm{H^{-1}}\frac{\sigma^2}{t+1}$\\ \hline
SPDS    & $(0,\alpha_{ps})$     &   $H^{\frac{1}{2}}$     & US        &   $\B^2=\norm{H^{-1}}\frac{\norm{\theta_0-\ts}^2_2}{\alpha^2(t+1)^2},~\V^2=\norm{H^{-1}}\frac{\sigma^2}{t+1}$\\ \hline

SPDS    & $(0,\alpha_{ps})$     &   $I$     & AUS        &   $\B^2=\norm{H^{-1}}^2\frac{\norm{\theta_0-\ts}^2_2}{\alpha^2(t+1)^2},~\V^2=\norm{H^{-1}}^2\frac{\sigma^2}{t+1}$\\ \hline
SPDS    & $(0,\alpha_{ps})$     &   $I$     & AS        &   $\B^2=\norm{H^{-1}}^2\frac{\norm{\theta_0-\ts}^2_2}{\alpha^2(t+1)^2},~\V^2=\norm{H^{-1}}\frac{\sigma^2}{t+1}$\\ \hline
SPDS    & $(0,\alpha_{ps})$     &   $H^{\frac{1}{2}}$     &AS        &   $\B^2=\frac{\norm{H^{-\frac{1}{2}}(\theta_0-\ts)}^2_2}{\alpha^2(t+1)^2},~\V^2=\frac{\sigma^2}{t+1}$\\ \hline
PS       & $(0,\alpha_{ps})$    &   $I$     & MUS        &   $\B^2=\frac{(\rho_H\alpha)^{-1}(1+4(\rho_H\alpha)^{-1})}{(t+1)^2},~\V^2=(\sigma^2)\frac{\alpha^2(\alpha\rho_H)^{-1}(1+4(\rho_H\alpha)^{-1})}{t+1}$\\ \hline
SPDS    & $(0,\alpha_{ps})$     &   $H$     & MUS        &  $\B^2=\frac{\norm{\theta_0-\ts}^2_2}{\alpha^2(t+1)^2},~\V^2=\frac{\sigma^2}{t+1}$\\ \hline
SPDS    & $(0,\alpha_{ps})$     &   $H^{\frac{1}{2}}$     & MUS        &   $\B^2=\frac{(\rho_H)^{-1}(\alpha)^{-2}}{(t+1)^2},~\V^2=(\sigma^2)\frac{(\rho_H)^{-1}}{t+1}$\\ \hline
SPDS    & $(0,\alpha_{ps})$     &   $H^{\frac{1}{2}}$     & MS        &   $\B^2=\frac{\norm{H^{-\frac{1}{2}}(\theta_0-\ts)}^2_2}{\alpha^2(t+1)^2},~\V^2=\frac{\sigma^2}{t+1}$\\ \hline
\end{tabular}
}
\end{center}
\caption{Main Result}
\label{maintable}
\end{table*}

We consider constant step size linear stochastic approximation (LSA) algorithms of the form
\begin{align}\label{lsa}
\theta_{t}=\theta_{t-1}+\alpha(g_t-H_t\theta_{t-1}),
\end{align}
where $\theta_t\in \R^n$, $\alpha>0$ is a positive step-size, $g_t\in \R^n$ and $H_t\in \R^{n\times n}$. The Polyak-Rupert average of the iterates $\theta_t$ in \eqref{lsa} is defined as
\begin{align}\label{rp} \tb_t\eqdef \frac{1}{t+1}\ous{\sum}{s=0}{t} \theta_s. \end{align}
We are interested in the behaviour of \eqref{lsa} under the following conditions.
\begin{assumption}\label{genlsa}
\begin{enumerate}
\item\label{mart} $H_t\eqdef H+M_t$, $g_t\eqdef g+N_t$, where $M_t\in \R^{n\times n}$ and $N_t\in \R^n$ are zero mean $i.i.d$ sequences, i.e., $\E[N_t]=0,\E[M_t]=0,~\forall t\geq 0$.
\item \label{noise} The noise sequences satisfy $\E[N_t^\top N_t]\leq \sigma_1^2$ and $\E[M_t^\top M_t]\leq \Sigma_2^2$, where $\sigma_1^2>0$ is a positive scalar and $\Sigma^2_2$ is a real symmetric positive definite matrix.
\item \label{mat} $H$ is such that $\forall x\in \R^n$, $\ip{x,Hx}>0$ and let $\ts=H^{-1}g\in\R^n$. Further, $\sigma_2^2$ is a scalar such that ${\ts}^\top\Sigma_2^2\ts\leq \sigma_2^2$ and $\sigma^2\eqdef\sigma_1^2+\sigma_2^2$.
\end{enumerate}
\end{assumption}
Here, $\ts$ is a parameter to be estimated that the LSA in \eqref{lsa} aims to compute from noisy data presented in the form of $g_t$ and $H_t$.\par
Our motivation to study constant step size LSA with RP-averaging stems from the fact that LSAs are common in applications such as temporal difference learning algorithms \cite{td} in reinforcement learning (RL)\cite{rl}, solution to large scale linear systems \cite, and the linear least squares problem in \cite{bachharder}. In particular, constant step size and RP averaging has been shown to have `` amazing" properties in the case of linear least squares problem \cite{bachharder} and we would like to investigate further as to whether these properties generalize to LSA.
\begin{example}[Temporal Difference Learning]
The Temporal Difference learning algorithm with a constant step size $\alpha>0$ is given by
\begin{align}\label{tdzero}
\theta_t=&\theta_{t-1}+\alpha \big(\phi(s_t) R(s_t)\nn\\&+ (\phi(s_t)\phi^\top(s_t)-\gamma \phi(s_t)\phi^\top(s'_t))\theta_{t-1}\big),
\end{align}
where are $s_t,s'_t\in S$ with $S$ being a finite set, $\phi(s)\in \R^n$ are the so called \emph{feature} vectors, $\gamma\in (0,1)$ is a given discount factor and $R\colon S\ra \R$ is map from S to reals. Here, $s_t$ are $i.i.d$ random variables distributed according to the law $s_t\sim \xi$ where $\xi$ is supported on $S$ and $s'_t\sim p(s_t,\cdot)$. The TD algorithm in \eqref{tdzero} can be cast in the general form as presented in \eqref{lsa}, by letting $g_t=\phi(s_t)R(s_t)$ and $H_t=\phi(s_t)\phi^\top(s_t)-\gamma \phi(s_t)\phi^\top(s'_t)$.
\end{example}

\textbf{Linear Stochastic Error Recursion (LSER)} The error dynamics for the LSA in \eqref{lsa} i.e., the dynamics of $e_t\eqdef\theta_t-\ts$ can be written as follows:
\begin{align}\label{errprelim}
&\theta_t-\ts=\theta_{t-1}-\ts+\alpha(g_t -H_t(\theta_t-\ts+\ts)),\text{~or}\nn\\
&e_t=(I-\alpha H_t)e_{t-1}+(N_t-M_t\ts).
\end{align}
In what follows we consider what we call linear stochastic error recursion (LSER) given by
\begin{align}\label{lsergen}
e_t=(I-\alpha H_t)e_{t-1}+\alpha \zeta_t,
\end{align}
where $\zeta_t\eqdef (N_t-M_t\ts)$.
\begin{theorem}\label{maintheorem}
Define $\rho_\alpha\eqdef \us{sup}{\norm{x}\leq 1}\E[x^\top (2H_t-\alpha H_tH_t)x]$. Let $\alpha_{\max}$ be such that $~\forall \alpha\in(0,\alpha_{\max})$ $\rho_{\alpha}>0$. Then, it follows that
\begin{align}
\begin{split}
\MoveEqLeft\E[\norm{H(\tb_t-\ts)}^2]\leq \\
&(1+4\frac1{\alpha\rho_{\alpha}}) \frac{1}{\alpha\rho_{\alpha}} \Big(\frac{\norm{\theta_0-\ts}^2}{(t+1)^2}+\frac{\alpha^2\sigma^2}{t+1} \Big)
\end{split}
\end{align}
\end{theorem}
<<<<<<< Updated upstream
=======

>>>>>>> Stashed changes
\textbf{Discussion}
\begin{enumerate}[leftmargin=*]
\item \textbf{Bias and Variance:} The means-squared error at $t$ is bounded by a sum of two terms. The first term is the bias term given by $\B=(1+4(\alpha\rho_{\alpha})^{-1}) (\alpha\rho_{\alpha})^{-1} \Big(\frac{\norm{\theta_0-\ts}^2}{(t+1)^2}\Big)$.  The \emph{Bias} term that captures the rate at which the initial condition $\norm{\theta_t-\ts}^2$ is forgotten. The second term is the \emph{Variance} given by $\V=(1+4(\alpha\rho_{\alpha})^{-1}) (\alpha\rho_{\alpha})^{-1} \Big(\frac{\alpha^2\sigma^2}{t+1} \Big)$. The variance term that captures the rate at which noise is rejected. The $\B$ and $\V$ terms capture two different sources of errors. To see this, note that in the absence of noise i.e., when $\zeta_t=0,~\forall t\geq 0$, we have $\E[\norm{H\eb_t}^2]=\B$ and in the presence of noise, i.e., $\zeta\neq 0,\forall t\geq 0$ and the perfect initial condition, i.e., $\theta_t=\ts$, we have $\E[\norm{H\eb_t}^2]=\V$.
\item \textbf{Faster Rates:} The bias term term decays $O(1/t^2)$ and the variance term decays $O(1/t)$, which are faster than the rates that can be achieved by a diminishing step size sequence recommended by standard stochastic approximation theory \cite{sa}. In particular, when the step sizes are decaying $O(1/t)$ one can only achieve an rate that is $O(1/t^\mu)$ where $\mu$ is the smallest real part of the eigenvalue of $H$. Thus the choice of a constant step size sequence with iterate averaging on top helps us to eliminate the effect of `ill' conditioning of $H$.
\item \textbf{Problem Dependent terms:} The choice of the step size is unfortunately problem dependent. To see this, the condition that $\rho_{\alpha}>0$ in \Cref{maintheorem} ensures that the expected spectral norm of $\E[(I-\alpha H_t)^\top(I-\alpha H_t)]$ is less than unity. And thus the range of $\alpha$ changes from problem to problem, an issue we further elaborate in \Cref{sec:stepprob}.
\item \textbf{Behaviour for extreme value of $\alpha$:} For smaller values of step size, i.e., $\alpha\approx 0$, the bias term blows up, due to the presence of $\alpha^{-1}$ term. This is due to the fact that the step sizes determine the learning rate and for smaller step sizes the learning rate is slower. However, in this case the noise term does not blow up, a fact that can appreciated by looking at \eqref{lsergen} where $\alpha$ is seen to multiply the noise term $\zeta_t$. In quantitative terms, we can see that the $\alpha^{-2}$ and $\alpha^2$ terms can each other. For larger values of $\alpha$ i.e., $\alpha\ra \alpha_{\max}$, the bounds blow up again, due to the fact that $\rho_{\alpha}\ra 0$ in this case. This is due to the fact that the effect of both noise and initial conditions decay with the contraction factor of $F_{t,i}$, which gets closer to unity as $\alpha\ra\alpha_{\max}$.
\end{enumerate}

\section{Main Results}
\begin{lemma}\label{addstep}
\begin{enumerate}[label=(\roman*)]
\item\label{ascase} For any $H$ that is AS there exists an $\alpha_{\as}>0$ for which it holds that $\rho(F(\alpha))<1,~\forall \alpha\in (0,\alpha_{as})$.
\item\label{pscase} For any $H$ that is PS there exists an $\alpha_{\ps}>0$ for which it holds that $\E[x^\top(H-\alpha H_t^\top H_t)x|\F_{t-1}]> 0$, $~\forall \alpha\in (0,\alpha_{ps})$.
\end{enumerate}
\end{lemma}
We begin by stating a result related to norms of matrices
\begin{lemma}
Let $A=(A_{ij})\in \R^{n\times n}$ be any real matrix, then $\norm{Ax}^2_2\leq n^2 \norm{x}^2_2 \us{\sup}{i,j}|(A)_{ij}|^2$ and $\ip{x,Ax}\leq n \norm{x}^2_2\us{\sup}{i,j}|(A)_{ij}|$, where $(A)_{ij}$ denotes the $ij^{th}$ entry of matrix $A$.
\end{lemma}
\begin{lemma}
For $\alpha \in (0,\alpha_{as})$ $B_1\eqdef\us{\us{\sup}{i,j}}{t\geq 0}|(F^t)_{ij}|$ and $B_2\eqdef\us{\us{\sup}{i,j}}{t\geq 0}|(F_D^{t\top}F_D^t)_{ij}|$ are finite.
\end{lemma}
\begin{theorem}\label{maintheorem}
The values of $\E[\norm{C\eb_t}]$ as listed in \Cref{maintable} hold.
\end{theorem}
\begin{proof}
See supplementary section.
\end{proof}
\section{Discussion of \Cref{maintheorem}}
We can recurse backwards in \eqref{errprelim} to obtain the following:
\begin{align}\label{errrec}
e_t=F_{t,1}e_0+\alpha\ous{\sum}{k=1}{t} F_{t,k+1}\zeta_k.
\end{align}
Looking at \eqref{errrec} it is clear that quantities $F_{t,j}, j=1,\ldots,t$ influence the dynamics of $e_t$. The results of \Cref{maintheorem} can be interpreted as below.
\begin{itemize}[leftmargin=*]
\item \textbf{Bias vs Variance:} All the entries in the \Cref{maintable} have two terms namely $\B$ and $\V$. The $\B$ is the \emph{Bias} term that captures the rate at which the initial condition $\norm{\theta_t-\ts}$ is forgotten. The $\V$ is the \emph{Variance} term that captures the rate at which noise is filtered. The $\B$ and $\V$ terms capture two different sources of errors. To see this, note that in the absence of noise i.e., when $\zeta_t=0,~\forall t\geq 0$, we have $\E[\norm{C\eb_t}^2]=\B$ and in the presence of noise, i.e., $\zeta\neq 0,\forall t\geq 0$ and perfect initial condition, i.e., $\theta_t=\ts$, we have $\E[\norm{C\eb_t}^2]=\V$.
\item \textbf{Error Recursion:} The expression for $\eb_t$ differs in the additive and the multiplicative noise cases and are given as below.
\begin{enumerate}
\item \textbf{Multiplicative Noise:}
\begin{align}
\eb_t&=\frac{1}{t+1}(\ous{\sum}{i=0}{t} F_{i,1} e_0+ \alpha\ous{\sum}{i=1}{t} \ous{\sum}{k=i}{t} F_{k,i+1}  \zeta_i )
\end{align}
Here recall that $F_{i,j}=(I-\alpha H_i)\ldots (I-\alpha H_j)$ are product of random matrices.
\end{enumerate}
\item \textbf{Additive Noise:}
\begin{align}\label{addexpand}
\eb_t&=\frac{1}{t+1}(\ous{\sum}{i=0}{t} F^i e_0+ \alpha\ous{\sum}{i=1}{t} \ous{\sum}{k=i}{t} F^{k-i}  \zeta_i )
\end{align}
Thus in the additive case involves only products of deterministic matrices since $F_{ij}=F^{i-j},\forall i\geq j$.
In comparison to the additive case, the multiplicative case is harder to analyze.

\item \textbf{AS vs PS vs SPDS:} As observed before, the rates in \Cref{maintheorem} are dependent on the nature of the quantities $F_{i,j}$, which in turn depends on the choice of step size and the spectral properties of $H$. Asking for $H$ to be AS is the weakest condition, followed by PS and the strongest condition is to ask for $H$ to be SPD. The relative strength of the three cases depends on the amount knowledge one has over the eigen values. For example, when $H$ is SPD, we know that it has all real and positive eigen values. Similarly, when $H$ is PS then we can ensure that the operator norm of $I-\alpha H$ i.e., $\norm{I-\alpha H}$ to be less than unity. Finally, when $H$ is only AS we can only ensure that the spectral radius is less than unity. It turns out that the differences in spectral properties impact the analysis and also practice.
\item \textbf{Spectral  Radius vs Operator Norm:} All the results in \Cref{maintable} are expressed for a given step size $\alpha>0$. However, the range of valid $\alpha$ changes from case to case.
\begin{enumerate}
\item \textbf{Additive Case:}In \Cref{addstep}-\ref{ascase}, the range $(0,\alpha_{as})$ ensures that the spectral radius of $F$ is less than unity, which in turn enables us to use $\ous{\sum}{i=1}{t}F^{i}=(I-(I-\alpha H)^t)$. Since $SPDS \Ra PS \Ra AS$, the range $(0,\alpha_{as})$ applies to all the three cases. The range $(0,\alpha_{ps})$ only holds for the PS and SPDS cases. It ensures that the operator norm of $F$ is less than unity.
\item \textbf{Multiplicative Case:} Here we deal only with product of random matrices and cannot directly apply use  $\ous{\sum}{i=1}{t}F^{i}=(I-(I-\alpha H)^t)$, and hence the spectral radius does not appear in the analysis. While the condition for $(0,\alpha_{ps})$ reads the same (see \Cref{addstep}-\ref{pscase}) it is stricter in the case of multiplicative noise, since $\E[H^\top_t H_t|\F_{t-1}]$ will be larger in the case of multiplicative noise.  Further, in the case of multiplicative noise, we have to deal with products of random matrices, i.e., $F_{ij}$ and ensure that they are contraction maps. This translates to the condition that $\E[\norm{F_{ij}}^2]= \E[(I-\alpha H_j)\ldots (I-\alpha H_i)^\top (I-\alpha H_i) \ldots (I-\alpha H_j)]<1$. Using \Cref{genlsa}-\ref{secondmom}  $\E[\norm{F_{ij}}]<1$ can be ensured when
\begin{align}\label{mulcond}
\E[(I-\alpha H_i)^\top (I-\alpha H_i)|\F_{i-1}]<1
\end{align}
Note that \eqref{mulcond} is much more stricter than ensuring that the spectral radius of $F$ is less than unity.
\end{enumerate}
\item \textbf{Role of $C$:} The rates are dependent on the choice of $C$. By choosing $C=H$, we have $H(\eb_t)=H(\tb_t-\ts)=H\tb_t-g$, which is the error in the prediction $H\tb_t$. Choosing $C=I$ measures the error in the deviation of the current estimated parameter $\tb_t$ to the ideal parameter $\ts$. As the \Cref{maintable} reveals, $C=H$ seems to yield better rates because while the deviation in the parameter estimate $\tb_t-\ts$ the predicted output $H(\tb_t-\ts)$ is typically scaled by $H$.
\end{itemize}
\begin{table*}
%\resizebox{0.5\textwidth}{!}{
\begin{tabular}{|c|c|c|c|}\hline
Algorithm& Single-Sample & $\sigma^2$ & Rate \\ \hline
LS-LSA& \ding{53} &$\begin{aligned}(\norm{A}^2+\sigma_1^2) (\sigma_2^2+\sigma_1^2\parallel \ts\parallel^2)\end{aligned}$&  $\begin{aligned} \B^2=\frac{\norm{(A^\top A)^{-1}}}{\alpha^2(t+1)^2},\\\V^2=(\sigma^2)\frac{\norm{(A^\top A)^{-1}}}{t+1}\end{aligned}$\\ \hline
FP-LSA& \ding{51} &$\begin{aligned} (\sigma_2^2+\sigma_1^2\norm{\ts}^2)\end{aligned}$& $\begin{aligned}\B^2=(1+4(\alpha\rho_A)^{-1}) (\alpha\rho_A)^{-1} \Big(\frac{\norm{\theta_0-\ts}^2}{(t+1)^2}\Big),~\\ \V^2=(1+4(\alpha\rho_A)^{-1}) (\alpha\rho_A)^{-1} \Big(\frac{\alpha^2\sigma^2}{t+1} \Big)\end{aligned}$\\ \hline
SP-LSA& \ding{51} &\ding{53}& \ding{53}\\ \hline
\end{tabular}
%}
\end{table*}


\comment{
\subsection{Discussion: Additive Noise Case}
 In the case of additive noise, since $M_t=0$, we have $F_{i,j}=F^{i-j}_D$ and hence
\textbf{Role of Contractions:} In this noise model, we need $F$ to have
\subsection{Discussion: Multiplicative Noise Case}
$\E\ip{H\eb_t,H\eb_t}=\frac{1}{(t+1)^2}\E \ip{\ous{\sum}{i=0}{t} F_{i,1}e_0+\alpha \ous{\sum}{k=1}{t} \ous{\sum}{i=k}{t} F_{i,k+1} \zeta_k,\ous{\sum}{i=0}{t} F_{i,1}e_0+\alpha \ous{\sum}{k=1}{t} \ous{\sum}{i=k}{t} F_{i,k+1} \zeta_k}$
\begin{enumerate}
\item \emph{Type I:} Terms involving $\E<F_{i,1}e_0,F_{i,1}e_0>, i,j=0,\ldots, t$ or $<F_{i,k+1}\zeta_k, F_{j,k+1}\zeta_k>,~ k=1,\ldots,t, i,j=k,\ldots,t$.
\item \emph{Type II:} Terms involving $\E<F_{i,k+1}\zeta_k, F_{j,k+1}\zeta_k>, k=1,\ldots,t, i,j=k,\ldots,t$ or inner products of form $<F_{i,1}e_0, F_{j,1}e_0>, i,j=0,\ldots, t$.
\item \emph{Type III:} Terms involving $\E<F_{i,1}e_0, F_{j,k+1}\zeta_k>, i=0,\ldots,t, 1\leq k \leq t$.
\item \emph{Type IV:} Terms involving $\E<F_{i,k+1}\zeta_{k'}, F_{j,k+1}\zeta_k>, k\neq k'$ and $k',k=1,\ldots,t, i,j=k,\ldots,t$.
\end{enumerate}
\textbf{Role of Contractions:}
The \emph{Type III} and \emph{Type IV} terms are zero due to the independence assumption.
\begin{lemma}
For all $i>j$ and $x\in \R^n$ $\E <F_{j,k+1} x,F_{i,k+1}x>=\E<F_{j,k+1}k,(I-\alpha H)^{i-j} F_{j,k+1}x>$
\end{lemma}
}

\section{LSAs to solve linear system of equations}
 We now look at the three different LSAs to solve the linear system given by
\begin{align}\label{linsys}
A\theta=b.
\end{align}
In what follows, we assume that we have access to the stochastic orcales which provide $i.i.d$ samples of $A$ and $b$.
\subsection{Least Squares Linear Stochastic Approximation (LS-LSA)}
When $A$ is not known to satisfy any specific assumption such as AS, PS or SPDS, it is important to enforce stability by designing the LSA appropriately. One way to ensure SPDS is to solve for $A^TA\theta^*=A^\top b$. We call the following stochastic recursion to compute $\theta^*$ a LS-LSA algorithm
\begin{align}\label{lslsa}
\theta_{t+1}=\theta_t+\alpha A^{(1)\top}_t(b_t-A^{(2)}_t\theta_t),
\end{align}
where we assume $b_t$, $A^{(1)}_t$and $A^{(2)}_t$ are $i.i.d$ samples of $b$ and $A$. The LSER corresponding to LS-LSA in \eqref{lslsa} is given by
\begin{align}\label{lslser}
e_{t+1}=&(I-\alpha A^{(1)\top}_tA^{(2)}_t)e_t+\alpha(A^\top M^{(3)}_{t+1}+M^{(1)\top}_{t+1}M^{(3)}_{t+1}\nn\\&-A^\top M^{(2)}_{t+1}\ts-M^{(1)\top}_{t+1}M^{(2)}_{t+1}\ts),
\end{align}
where $M^{(1)}_{t+1}=A^{(1)}_t-A$, $M^{(2)}_{t+1}=A^{(2)}_t-A$ and $M^{(3)}_{t+1}=b_t-b$. Choosing $H_t=A^{(1)\top}_tA^{(2)}_t$, $\zeta_t= A^\top M^{(3)}_{t+1}+M^{(1)\top}_{t+1}M^{(3)}_{t+1}-A^\top M^{(2)}_{t+1}\ts-M^{(1)\top}_{t+1}M^{(2)}_{t+1}\ts$ and $\sigma^2=\parallel A\parallel^2 (\sigma_2^2+\sigma_1^2\parallel \ts\parallel^2)+ \sigma_1^4(1+\parallel \ts\parallel^2)$ we can identify \eqref{lslser} with \eqref{lsergen}, and  the results of \Cref{spdsmn} applies to the LSER in \eqref{lslser}.
\subsection{Fixed Point Linear Stochastic Approximation (FP-LSA)}
When it is known that $A$ is PS,  we can use the following LSA which we call the fixed point LSA (FP-LSA) to compute $\theta^*$:
\begin{align}\label{fplsa}
\theta_{t+1}=\theta_t+\alpha(b_t-A_t),
\end{align}
where we assume that $b_t$ and $A_t$ are $i.i.d$ samples of $b$ and $A$. We now present the LSER corresponding the FP-LSA in \eqref{fplsa}.
\begin{align}\label{fplser}
e_{t+1}=(I-\alpha A_t)e_t+\alpha(M^{(2)}_{t+1}-M^{(1)}_{t+1}\ts),
\end{align}
where $M^{(1)}_{t+1}=A_t-A$, $M^{(2)_{t+1}}=b_t-b$,  and $\ts=A^{-1}b$. Now letting $H_t=A_t$, $\zeta_t=M^{(2)}_{t+1}-M^{(1)}_{t+1}\ts$ and $\sigma=\sigma_1^2+\sigma^2_2 \parallel\ts\parallel^2$ we can identity \eqref{fplser} wit the \eqref{lsergen} and the results of \Cref{psmn} applies to the LSER in \eqref{fplser}.
\subsection{Saddle-Point Linear Stochastic Approximation (SP-LSA)}
An important downside of \Cref{lslsa} is that two independent samples of $A$ are available in the form of $A^{(1)\top}_t$ and $A^{(2)}_t$ which might be possible in certain applications. This issue can be alleviated by looking at the following Saddle-Point objective function
\begin{align}
L(\theta,y)=\min_{\theta\in \R^n}\max{y\in \R^n}<b-A\theta,y>-\frac{1}{2}\parallel y\parallel_M ^2
\end{align}
By noticing the fact that $\nabla_\theta L =-A^\top y$ and $\nabla_{y}L=-A\theta-M$, we now specify the SP-LSA algorithm as follows:
\begin{align}\label{splsa}
\begin{split}
y_{t+1}&=y_t+\alpha (b_t-A_t\theta_t- M y_t)\\
\theta_{t+1}&=\theta_t+\alpha(A_t^\top)y_t
\end{split}
\end{align}
The SLER corresponding to the SP-LSA in \eqref{splsa} is given by
\begin{align}
e_{t+1}=\Big(\begin{bmatrix} I & 0 \\ 0 &I\\\end{bmatrix} -\begin{bmatrix} M & A_t \\ -A_t^\top &0\\\end{bmatrix}\Big)e_t+\Big(\begin{bmatrix} M^{(2)}_{t_1}-M^{(1)}_{t+1}\ts  \\ 0\\\end{bmatrix}\Big),
\end{align}
where $M^{(1)}_{t+1}=A_t-A$, $M^{(2)_{t+1}}=b_t-b$,  and $\ts=A^{-1}b$. Now letting $H_t=\begin{bmatrix} M & A_t \\ -A_t^\top &0\\\end{bmatrix}$, we notice that $H=\E[H_t]=\begin{bmatrix} M & A \\ -A^\top &0\\\end{bmatrix}$, is not $PS$ and hence it is not straightforward to derive finite time bounds. However, by replacing $\alpha$ in \eqref{splsa} by $\alpha_t$ which satisfies \Cref{dimassmp}, the behaviour of the iterates can then be analysed using \Cref{sadim}.
\comment{
Let $x_t=(y_t,\theta_t)$, $g_t=\begin{bmatrix} b_t\\ 0\\\end{bmatrix}$ and $G^{(i)}_t=\begin{bmatrix} M & A^{(i)}_t \\ -A^{(i)\top}_t &0\\\end{bmatrix}$ for $i=1,2$, then consider the following updates
\begin{align}
x'_{t}&=x_t+\alpha (g_t-G^{(1)}_t x_t)\\
x_{t+1}&=x_t+\alpha(g_t- G^{(2)}_t x'_t);
\end{align}
}

\section{Background: Reinforcement Learning}
We now present a brief overview of Markov Decision Processes (MDPs) which is a framework to describe the RL setting, i.e., the agent's interaction with the environment. An MDP is a five tuple and is denoted by $M=\{S,A,P,R,\gamma\}$, where $S$ is the state space, $A$ is the action space, $P=(p_a(s,s'),a\in A,s,s’\in S)$ is the probability transition kernel that specifies the probability of transitioning from state $s$ to $s'$ when the agent taking an action $a$, and $R(s,a)\colon S\times A\ra \R$ is the reward for taking an action $a$ in state $s$ and $0<\gamma<1$ is a given discount factor. A stationary deterministic policy (or simply a policy) is a denoted by $\pi=(\pi(s,\cdot),s\in S)$, where $\pi(s,\cdot)$ is a probability distribution over the set of actions. The infinite horizon discounted reward of a policy $\pi$ is given by, $V_\pi=\mathbf{E}[\sum_{t=0}^\infty \gamma^t R(s_t,a_t)| s_0=s, \pi]$,
and is the discounted sum of rewards starting from state $s$ and taking actions according to policy $\pi$. The value function $V_\pi$ can be computed by solving the Bellman equation (BE) given below:
\begin{align}\label{be}
V_\pi=R_\pi+\gamma P_\pi V_\pi,
\end{align}
where $R_\pi(s)=\sum_{a\in A}\pi(s,a) R(s,a)$ and $p_\pi(s,s’)=\sum_{a\in A}\pi(s,a)p_a(s,s’)$.\par
\subsection{Projected Bellman Equation}
The BE is a linear sytem of equations, and needs to be solved from the samples. However, when the number of states $|S|$ is large, it is common to use a linear representation for value function i.e., to let $V_\pi\approx\Phi \ts$. The value function can then be learnt by solving a projected Bellman equation (PBE) instead of the BE in \eqref{be}. The PBE is given as follows
\begin{align}
\Phi\ts=\Pi (R_\pi+\gamma P_\pi \Phi \ts),
\end{align}
where $\Pi=\Phi(\Phi^\top \Xi\Phi)^{-1}\Phi^\top\Xi$, where is a diagonal matrix with positive diagonal entries $\xi_{s}, s\in S$ such that $\sum_{s\in S} \xi=1$. Note that $\xi_s$ here stands for the weight of the state $s$, and $\Pi V= {\arg\min}_{\theta\in \R^n}\parallel \Phi \theta -V\parallel_{\xi}$. One can re-write the PBE as below
\begin{align}\label{tdlin}
\Phi\ts&= (\Phi^\top \Xi\Phi)^{-1}\Phi^\top\Xi (R_\pi+\gamma P_\pi \Phi \ts)\nn\\
(\Phi^\top \Xi\Phi)\ts&= \Phi^\top\Xi (R_\pi+\gamma P_\pi \Phi \ts)\nn\\
\underbrace{\Phi^\top \Xi(I-\gamma P_\pi)\Phi}_{A}\ts&= \underbrace{\Phi^\top\Xi R_\pi}_{b}
\end{align}
We consider the scenario where we have data $\D=\{(s_i,a_i,s'_i),\forall i=1,\ldots,N\}$, where $s_i\sim \xi$, $a_i\sim \pi_b(s_i,\cdot)$, $s'_i\sim p_{a_i}(s_i,\cdot)$, where $\pi_b$ is the \emph{behavior} policy, which might be different from the \emph{target} policy $\pi$ whose value function $V_\pi$ we are interested in computing. This scenario is known as the \ofp setting. A special case is that of \onp setting wherein $\pi_b=\pi$. A quantity of interest in this context is the importance sampling ratio given by $\rho(s,a)=\frac{\pi(s,a)}{\pi_b(s,a)}$.\par
In the RL setting, model parameters $P^\pi$ and $R^\pi$ are not explicitly available. This implies that we have to solve for \eqref{tdlin} using noisy samples of $A=\Phi^\top \Xi(I-\gamma P_\pi)\Phi$ and $b=\Phi^\top\Xi R_\pi$.\par
\begin{lemma}
Let $A_t=\rho(s_i,a_i)phi^\top(s_i)(\phi(s_i)-\gamma \phi(s'_i))$ and $b_t=\rho(s_t,a_t)\phi^\top(s_t) R(s_t,a_t)$ then we have $\E[A_t|\xi]=A$ and $\E[b_t|\xi]=b$.
\end{lemma}
\section{LSA algorithms in RL}
We now look at LSA algorithms occuring in RL that solve \eqref{tdlin} using noisy samples.
\subsection{Fixed Point LSA}
The vanilla temporal difference (simply TD) algorithm is a LSA to solve \eqref{tdlin}. The TD algorithm is given as follows:
\begin{align}\label{vtd}
\theta_{t+1}=\theta_t+\alpha_t \rho(s_t)\phi^\top(s_t)(R(s_t,a_t)+\gamma \phi(s'_t)\theta_t- \phi(s_t)\theta_t),
\end{align}
Thus results in \Cref{sec:fp} apply to this algorithm
\subsection{Saddle Point LSA}
The gradient temporal difference (GTD) learning algorithms were shown to be gradient algorithms with respect to the saddle point objectives. We present GTD and GTD2 with their saddle point objective functions
\begin{align}\label{gtd}
\begin{split}
\textbf{GTD(GTD2):\quad}y_{t+1}&=y_t+\alpha \rho_t(b_t-A_t\theta_t - M y_t)\\
\theta_{t+1}&=\theta_t+\alpha\rho_t(A^\top_ty_t),
\end{split}
\end{align}
where GTD/GTD2 can be obtained by letting $M=I/C$ respectively in \eqref{gtd}. Results of \Cref{sec:splsa} apply to the GTD algorithms
\comment{
\subsection{Least Squares LSA}
We now present a novel algorithm called the \emph{incremetal-gradient} least squares temporal difference leanring (iGLTSD), named so owing to the fact that it makes use of gradient the least squares objective and at the same times makes incremental updates like the iLSTD algorithm.
\FloatBarrier
\begin{algorithm}[H]
\begin{algorithmic}[1]
\STATE{$s\la s_0, b\la 0, A\la 0, \hat{A}\la 0 \mu\la 0, t\la 0$}
\STATE{Initialize $r_0$ arbitrarily}
\FOR{$t=0,1,2,\ldots,N$}
%\STATE{$\Delta b_t= \phi(s_t) r_t$}
%\STATE{$\Delta A_t= \phi(s_t)(\phi(s_t)-\gamma\phi(s_t))^\top$}
%\STATE{$ b_{t+1}= b_t+\Delta b_t$}
%\STATE{$ A_{t+1}= A_t+\Delta A_t$}
\STATE{$ \mu_{t}= \rho(s_t)\phi^\top(s_t)(\phi(s_t)-\gamma\phi(s'_t))$}
\STATE {$\theta_{t+1}= \theta_t+\alpha \hat{A}\top\mu_t$}
\STATE{$j\la$ choose an random index from $1,\ldots,n$}
\STATE {$\hat{A}= \phi(s_t)(j)(\phi(s_t)-\gamma\phi(s'_t))$}
\ENDFOR
\end{algorithmic}
\end{algorithm}
The analysis of this algorithm follows from \Cref{sec:lslsa}.

}
\subsection{Variants of GTD}
The GTD-\emph{Mirror-Prox} algorithm is given by
\begin{align}
\begin{split}
y^m_{t}&=y_t+\alpha \rho_t(b_t-A_t\theta_t - M y_t)\\
\theta^m_{t}&=\theta_t+\alpha\rho_t(A^\top_ty_t),\\
y_{t+1}&=y_t+\alpha \rho_t(b_t-A_t\theta^m_t - M y^m_t)\\
\theta_{t+1}&=\theta_t+\alpha\rho_t(A^\top_ty^m_t),
\end{split}
\end{align}

\section{RL algorithms: Insights from Spectral Analysis}
\subsection{Price of Off-Policy Stability}
We discuss the price paid by GTD variants to achieve off-policy stability.
For the purpose of our discussions, we study the following modified version of GTD given by:
\begin{align}\label{gtdbeta}
\begin{split}
\textbf{GTD($\beta$)\quad}y_{t+1}&=y_t+\alpha \rho_t(b_t-A_t\theta_t - M y_t)\\
\theta_{t+1}&=\theta_t+\alpha\rho_t\beta(A^\top_ty_t)
\end{split}
\end{align}
We call the above variant GTD($\beta$).
In order to keep the discussion intuitive, we first consider a simple example MDP and then
\begin{example}\label{onestatemdp}
Consider the MPD with $1$ state, $1$ action, the probability transition $p=1$, reward $R=1$, discount factor $\gamma 1-\epsilon$ (for small $\epsilon>0$ ) and feature matrix $\Phi=1$. Here $A=1-\gamma=\epsilon$, $R=1$, $V^*=\frac{1}{\epsilon}$.
\end{example}
We now write down the TD and GTD($\beta$) algorithms for this $1$-state MDP in \Cref{onestatemdp}.
\begin{align}\label{tdonestate}
\begin{split}
\theta_{t}&=\theta_{t-1}+\alpha(1-\epsilon\theta_{t-1})\\
&=(1-\alpha\epsilon)\theta_{t-1}+\alpha 1
\end{split}
\end{align}
The Eigen value of the above TD update is given by $\mu=1-\alpha\epsilon$.
%The Eigen value decay is given by $\mu^t\approx e^{-\alpha\epsilon t}$
\begin{align}\label{gtdonestate}
\begin{bmatrix} y_t \\ \theta_t\\\end{bmatrix}=\Big(\begin{bmatrix} 1&0 \\ 0& 1\\\end{bmatrix} - \alpha\begin{bmatrix} 1&\epsilon \\ -\beta\epsilon& 0\\\end{bmatrix}\Big)\begin{bmatrix} y_t \\ \theta_t\\\end{bmatrix}+\alpha\begin{bmatrix} 1 \\ 0\\\end{bmatrix}
\end{align}
%The Eigen values of $H=\begin{bmatrix} 1-\alpha& -\alpha\epsilon \\ \alpha\beta\epsilon& 1\\\end{bmatrix}$ is given by:
\begin{align}
1-\frac{\alpha}{2}(1\pm\sqrt{1-4\beta\epsilon^2}),
\end{align}
and consider the various cases of $\beta$ as follows.\\
\comment{
\begin{table}
\resizebox{0.5\textwidth}{!}{
\begin{tabular}{|c|c|c|c|}\hline
&$0<\beta\leq1 $ &     $\beta=\frac{1}{4\epsilon^2}$  & $\beta>\frac{1}{4\epsilon^2}$ \\ \hline
$\begin{aligned} \text{Eigen}\\ \text{Value}\end{aligned}$ &$\begin{aligned} &\mu_1=1- \alpha(1-\beta\epsilon^2) \\ &\mu_2=1-\alpha(\beta\epsilon^2) \\ & \end{aligned}$ & $\begin{aligned}\mu_1=1- \frac{\alpha}{2} \\ &\mu_2=1-\frac{\alpha}{2} \\ \end{aligned}$ & $\begin{aligned}\mu_{1}=1-\frac{\alpha}{2}(1+ i\sqrt{4\beta\epsilon^2-1})\\ \mu_2=1-\frac{\alpha}{2}(1- i\sqrt{4\beta\epsilon^2-1})\end{aligned}$\\ \hline
$\begin{aligned} \text{Spectral}\\ \text{Radius}\end{aligned}$ &$\begin{aligned} 1-\alpha(\beta\epsilon^2)  \end{aligned}$ & $1-\frac{\alpha}{2}$ & $\begin{aligned} \sqrt{1-\frac{\alpha}{2}+\alpha^2\beta\epsilon^2}\end{aligned}$\\ \hline
\end{tabular}
}
\end{table}
}
\begin{table}
\resizebox{0.5\textwidth}{!}{
\begin{tabular}{|c|c|c|}\hline
$0<\beta\leq1 $ &     $\beta=\frac{1}{4\epsilon^2}$  & $\beta>\frac{1}{4\epsilon^2}$ \\ \hline
 $\begin{aligned} &\mu_1=1- \alpha(1-\beta\epsilon^2) \\ &\mu_2=1-\alpha(\beta\epsilon^2) \\ & \end{aligned}$ & $\begin{aligned}&\mu_1=1- \frac{\alpha}{2} \\ &\mu_2=1-\frac{\alpha}{2} \\ \end{aligned}$ & $\begin{aligned}\mu_{1}=1-\frac{\alpha}{2}(1+ i\sqrt{4\beta\epsilon^2-1})\\ \mu_2=1-\frac{\alpha}{2}(1- i\sqrt{4\beta\epsilon^2-1})\end{aligned}$\\ \hline
$\begin{aligned} 1-\alpha(\beta\epsilon^2)  \end{aligned}$ & $1-\frac{\alpha}{2}$ & $\begin{aligned} \sqrt{1-\frac{\alpha}{2}+\alpha^2\beta\epsilon^2}\end{aligned}$\\ \hline
\end{tabular}
}
\end{table}
\begin{enumerate}[leftmargin=*]
\item \textbf{Slowness of GTD:} The vanilla GTD algorithm ($\beta=1$) has been observed to be slow in comparison to the TD. The toy example captures this fact in a simple yet clear manner. To see this, observe that the spectral radius of the \eqref{gtdonstate} and \eqref{tdonestate} are $1-\alpha\epsilon^2$ and $1-\alpha\epsilon$ respectively. Thus the rate of convergence of the GTD  and TD in this toy example are $(1-\alpha\epsilon)^t\approx e^{-\alpha\epsilon^2 t}$ and $(1-\alpha\epsilon)^t\approx e^{-\alpha\epsilon t}$ respectively. Thus the performance of the GTD is off by a factor of $\epsilon$.
\item \textbf{Price of Stabiity} Notice that the GTD is stable for the off-policy setting as well. The GTD is two timescale algorithm with the dual variable $y$ estimates $(b-A\theta)$ and the primal updates involve $A^\top(y)=A^\top(b-A\theta)$. Thus the GTD implictly has $A^\top A$ in their update equation and hence it is not surprising that in the toy example $\epsilon^2$ plays a role. This effect continues even in higher dimensions (see \Cref{squared}). While the GTD is stable in all the settings its slowness seems to be the inevitable price that we need to pay.
\item The effect of the condition number can however be eliminated in the case of this toy example. By setting $\alpha=\frac{1}{\epsilon}$ in \eqref{tdonstate} and $\alpha=2$, $\beta=\frac{1}{4\epsilon^2}$ in \eqref{gtdonestate}, we can eliminate the effect of $\epsilon$. However, in higher dimensions the GTD cannot be made as good as TD (see \Cref{}).
\end{enumerate}
\begin{lemma}\label{squared}
When $A$ has a small positive Eigen value say $0<\mu<<1$, then $1-\alpha\mu^2$ is an Eigen value of $F=\begin{bmatrix} (1-\alpha)I & -\alpha A \\ \alpha A^\top& 0\\\end{bmatrix}$.
\end{lemma}
\begin{lemma}
When $\mu_{\min}>0$ and $\mu_{\max}>0$ are the smallest and largest Eigen values of $A$ respectively. Then the best possible spectral radius that can be achieved for the TD and GTD algorithms are given by $1-2\frac{\mu_{\min}}{\mu_{\max}}$ and $1-2\frac{\mu_{\min}^2}{\mu_{\max}^2}$ respectively.
\end{lemma}
\subsection{Shortcomings of GTD-Mirror-Prox}
The GTD-\emph{Mirror-Prox} algorithm is given by
\begin{align}
\begin{split}
y^m_{t}&=y_t+\alpha \rho_t(b_t-A_t\theta_t - M y_t)\\
\theta^m_{t}&=\theta_t+\alpha\rho_t(A^\top_ty_t),\\
y_{t+1}&=y_t+\alpha \rho_t(b_t-A_t\theta^m_t - M y^m_t)\\
\theta_{t+1}&=\theta_t+\alpha\rho_t(A^\top_ty^m_t),
\end{split}
\end{align}
\begin{table}
\resizebox{0.5\textwidth}{!}{
\begin{tabular}{|c|c|c|}\hline
TD& GTD&   GTD-MP \\ \hline
$F_{TD}=(I-\alpha A)$& $F_{GTD}=\begin{bmatrix} (1-\alpha) I&-\alpha A \\ -\alpha A^\top & 0\\\end{bmatrix} $ & $F_{GTDMP}=I-\alpha F_{GTD}+\alpha^2 F^2_{GTD}$ \\ \hline
\end{tabular}
}
\end{table}

\subsection{Maximum Allowable Step-Size}\label{opti}
The condition that $0<\alpha<\alpha_{\max}$ ensures that $\rho_{\alpha}<1$ and in \cite{bachaistats} authors conjecture that this bound on $\alpha_{\max}$ is strict i.e., there exists some initial condition $x_0$ for which the LSA in \eqref{linearrec} is unstable. We now present a theorem from \cite{logexp} and simple counter examples to falsify this conjecture.
\begin{theorem}\label{explog}
Let $\mathcal{H}=(H_t), t\geq 0$ be a stationary process of $n\times n$ real-valued matrices over some probability space $(\Omega,\F,\mathcal{P})$. If $\E\log^+\parallel H_0\parallel<\infty$ (where $\log^+ x$ denotes the positive part of $\log x$), then there exists a $\lambda\in \R$ that satisfies
\begin{align}\label{lambda}
\lambda=\lim_{t}\frac{1}{t}\log\parallel H_t H_{t-1}\ldots H_0\parallel
\end{align}
\end{theorem}
In the case when the implicit relation in \eqref{lambda} yields a $\lambda<0$, it is also implied that $\{z_t\},t\geq 0$ such that $z_t=H_t H_{t-1}\ldots H_0z_0$ is stable.
\begin{example}[Linear Prediction]
Let the data represented as $(input,output)$ be $(X_t,Y_t)\in \{(2,0), (4,0)\}$ with equal probability. The problem of linear prediction is then to find $\theta^*\in R$ such that it minimizes the loss $\E(X_t \theta^* -Y_t)^2$, and the SGD algorithm to solve find $\theta^*$ is given by
\begin{align}
\theta_{t+1}=\theta_t+\alpha(y_t X_t-X_t\otimes X_t\theta_t),
\end{align}
where $\alpha$ is the constant step-size. Now the condition on $\alpha_{\max}$ presented in \Cref{alphacond} and in \cite{bachaistats}, translates to the following numerical condition in this specific example
\begin{align*}
\alpha_{\max}<\frac{2H}{\E[H_t^\top H_t]}=\frac{10}{17}
\end{align*}
We now derive $\alpha^{\lambda}_{\max}$ which is the maximum allowable constant step-size as suggested by the implicit relation in \eqref{lambda}. We have
\begin{align}
\lambda=\frac{1}{2}\log(1-\alpha)+\frac{1}{2}\log(1-\alpha 4).
\end{align}
For stability we need $\lambda<0$, i.e., $-1<(1-\alpha)(1-alpha 4)<1$, which translates to the condition $0<\alpha<\alpha^{\lambda}_{\max}=\frac{5}{4}$. It is clear that $\alpha^{\lambda}_{\max}>\frac{2}{\E[H]}=0.8>\alpha_{\max}$.
\end{example}
A similar counter example can be provided in the asymmetric case as follows
\begin{example}[General LSA]
Consider the LSA in \eqref{linearrec} with $x_t\in \R$ and $H_t\sim \{-1, 2\}$ (with equal probability)  and $g_t=0$. Then 
$
\alpha_{\max}<\frac{2H}{\E[H_t^\top H_t]}=\frac{2}{5}
$
and since 
$
\lambda=\frac{1}{2}\log(1+\alpha)+\frac{1}{2}\log(1-\alpha 2).
$
we have $\alpha^{\lambda}_{\max}=0.78078$. It is clear that $\alpha^{\lambda}_{\max}>\alpha_{\max}$. However, in the asymmetric case we have $\frac{2}{\E[H]}=1>\alpha^{\lambda}_{\max}>\alpha_{\max}$.
\end{example}
%\FloatBarrier
\begin{figure}[htp]
\begin{minipage}{0.5\textwidth}
\resizebox{1.0\textwidth}{!}{
\begin{tabular}{cc}
\begin{tikzpicture}[scale=1,font=\Large,]
    \begin{axis}[
        xlabel=$t$,
        ylabel=$\theta_t$,legend style={at={(0.5,-0.1)},anchor=north}
]

    \addplot[only marks,mark=square,red] plot file {./experiments/symm_stable_samp};
    \addplot[only marks,mark=diamond,blue] plot file {./experiments/asymm_stable_samp};

\addlegendentry{{\color{black}{Example~1,$\alpha=1.1$}}}
\addlegendentry{{\color{black}{Example~2,$\alpha=0.7$}}}


    \addplot[thick,dashed,mark=.,red] plot file {./experiments/symm_stable};
    \addplot[thick,dashed,mark=.,blue] plot file {./experiments/asymm_stable};

    \end{axis}
    \end{tikzpicture}

&
\begin{tikzpicture}[scale=1,font=\Large]
    \begin{axis}[
        xlabel=$t$,
        ylabel=$\theta_t$,legend style={at={(0.5,-0.1)},anchor=north}
]

    \addplot[only marks,mark=square,red] plot file {./experiments/symm_unstable_samp};
    \addplot[only marks,mark=diamond,blue] plot file {./experiments/asymm_unstable_samp};

\addlegendentry{{\color{black}{$\theta_t$ Example~1,$\alpha=1.3$}}}
\addlegendentry{{\color{black}{$\theta_t\times 10^{25}$Example~2,$\alpha=0.8$}}}


    \addplot[thick,dashed,mark=.,red] plot file {./experiments/symm_unstable};
    \addplot[thick,dashed,mark=.,blue] plot file {./experiments/asymm_unstable};

    \end{axis}
    \end{tikzpicture}

\end{tabular}
}
\end{minipage}
\end{figure}



\begin{table}
\begin{center}
\resizebox{0.5\textwidth}{!}{
\begin{tabular}{|c|c|c|c|c|c|c|}\hline
System  &   Matrix      & Noise & $H^{(1)}$ &  $H^{(2)}$   &   $\zeta^{(1)}$   &   $\zeta^{(2)}$   \\\hline
$S_1$   & AS          &AUS    &$\begin{bmatrix} 1 & 100 \\ 0 &1\\\end{bmatrix}$ &$\begin{bmatrix} 1 & 100 \\ 0 &1\\\end{bmatrix}$ &$\begin{bmatrix} U_1 \\ U_2\\\end{bmatrix}$ &$\begin{bmatrix} U_1 \\ U_2\\\end{bmatrix}$ \\ \hline
$S_2$   & AS          &AUS    &$\begin{bmatrix} 1 & 0.1 \\ -0.1 &0\\\end{bmatrix}$ &$\begin{bmatrix} 1 & 0.1 \\ -0.1 &0\\\end{bmatrix}$ &$\begin{bmatrix} U_1 \\ U_2\\\end{bmatrix}$ &$\begin{bmatrix} U_1 \\ U_2\\\end{bmatrix}$ \\ \hline

$S_3$   &PS          & AUS   &$\begin{bmatrix} 1 & 1 \\ 0 &1\\\end{bmatrix}$ &$\begin{bmatrix} 1 & 1 \\ 0 &1\\\end{bmatrix}$ &$\begin{bmatrix} U_1 \\ U_2\\\end{bmatrix}$ &$\begin{bmatrix} U_1 \\ U_2\\\end{bmatrix}$ \\ \hline
$S_4$   &SPDS          & AS   &$\begin{bmatrix} 1 & 0 \\ 0 &1\\\end{bmatrix}$ &$\begin{bmatrix} 1 & 0 \\ 0 &1\\\end{bmatrix}$ &$\begin{bmatrix} U_1 \\ U_2\\\end{bmatrix}$ &$\begin{bmatrix} U_1 \\ U_2\\\end{bmatrix}$ \\ \hline
$S_5$   &SPDS          & AUS   &$\begin{bmatrix} 1 & 0 \\ 0 &0.01\\\end{bmatrix}$ &$\begin{bmatrix} 1 & 0 \\ 0 &0.01\\\end{bmatrix}$ &$\begin{bmatrix} U_1 \\ U_2\\\end{bmatrix}$ &$\begin{bmatrix} U_1 \\ U_2\\\end{bmatrix}$ \\ \hline
$S_6$   &SPDS          & AS   &$\begin{bmatrix} 1 & 0 \\ 0 &0.01\\\end{bmatrix}$ &$\begin{bmatrix} 1 & 0 \\ 0 &0.01\\\end{bmatrix}$ &$\begin{bmatrix} U_1 \\ 0\\\end{bmatrix}$ &$\begin{bmatrix} 0 \\ 0.1 U_2\\\end{bmatrix}$ \\ \hline
$S_7$   &AS          & MUS   &$\begin{bmatrix} 1 & 1.1 \\ -0.9 &0\\\end{bmatrix}$ &$\begin{bmatrix} 1 & -0.9 \\ -0.8 &0 \\\end{bmatrix}$ &$\begin{bmatrix} U_1 \\ U_2\\\end{bmatrix}$ &$\begin{bmatrix} U_1 \\ U_2\\\end{bmatrix}$ \\ \hline
$S_8$   &PS          & MUS   &$\begin{bmatrix} 1 & 0.2 \\ 0.8 &0.1\\\end{bmatrix}$ &$\begin{bmatrix} 1 & 1.8 \\ -0.8 &0.1\\\end{bmatrix}$ &$\begin{bmatrix} U_1 \\ U_2\\\end{bmatrix}$ &$\begin{bmatrix} U_1 \\ U_2\\\end{bmatrix}$ \\ \hline
$S_9$   &SPDS          & MS   &$\begin{bmatrix} 1 & 0 \\ 0 &0\\\end{bmatrix}$ &$\begin{bmatrix} 0 & 0 \\ 0 &0.01\\\end{bmatrix}$ &$\begin{bmatrix} U_1 \\ 0\\\end{bmatrix}$ &$\begin{bmatrix} 0 \\ 0.1 U_2\\\end{bmatrix}$ \\ \hline
$S_{10}$   &SPDS          & MUS   &$\begin{bmatrix} 1 & 0 \\ 0 &0\\\end{bmatrix}$ &$\begin{bmatrix} 0 & 0 \\ 0 &0.01\\\end{bmatrix}$ &$\begin{bmatrix} U_1 \\ U_2\\\end{bmatrix}$ &$\begin{bmatrix} U_1 \\ U_2\\\end{bmatrix}$ \\ \hline
$S_{11}$   &SPDS          & MS   &$\begin{bmatrix} 1 & 0 \\ 0 &0\\\end{bmatrix}$ &$\begin{bmatrix} 0 & 0 \\ 0 &1\\\end{bmatrix}$ &$\begin{bmatrix} U_1 \\ U_2\\\end{bmatrix}$ &$\begin{bmatrix} U_1 \\ U_2\\\end{bmatrix}$ \\ \hline
\end{tabular}
}
\end{center}
\end{table}


\fi
%\input{pltadd}
%\input{back}
%\section{Implication of the Results}
\emph{Temporal Difference} (TD) learning algorithms \cite{} are widely used to learn value functions. A desirable feature of these algorithms that they are LSA algorithms that perform only $O(n)$ computations per time step. The vanilla TD (or simply TD) algorithm was first introduced in \cite{}. An unresolved issue for quite some time was the divergent behavior of TD in the \emph{off-policy} setting, where the sample were obtained from a policy different from the one whose value function needs to be computed. The Gradient Temporal Difference (GTD) learning algorithm \cite{} addressed the issue of divergence and are provably stable in \emph{off-policy} scenarios.
%Thus the GTD methods solve \eqref{linsys} in the general case when $\pi_t\neq\pi_b$.
%The \ofp convergence issue has also been addressed differently by the least-squares TD (LSTD) algorithm, which constructs $A$ and $b$ in \eqref{lsa}, and obtain $\theta^*=A^{-1}b$. As a consequence of the %matrix inversions involved, the LTSD performs a minimum of $O(n^2)$ computations, which can be a possible downside when $n$ is large.\par
Recently, newer variants of GTD namely, projected GTD2 and GTD-\emph{Mirror-Prox} were proposed in \cite{}. The authors in \cite{} observe that the GTD algorithm can be derived as a true stochastic gradient algorithms with repsect to a primal-dual saddle point objective function. This observation enables application of finite-time results for stochastic gradient (SG) algorithms to the GTD algorithm and at the same time, derive the newer variants based on the stochastic \emph{Mirror-Prox} algorithm \cite{}.\par
While several variants of TD algorithms have been developed, and have been empirically tested, as far as we gather, there are a number of poorly understood aspects of these algorithms, which we list as under.
\begin{itemize}[leftmargin=*]
\item Stochastic approximation \cite{SA} theory typically needs diminishing and square summable step-sizes. Further, SA theory deals with study of ordinary differnetial equations (ODEs) and lead only to asymptotic rates. The diminishing nature of the step-size schedules results in poor asymptotics, with the rates of convergence depending on the spectrum of the matrices involved in the SA updates. This is undesirable since the matrix (see \eqref{linsys}) involves quantities which $i$) are unkown and $ii$) vary across environment.
\item It is quite common in literature to sweep the step-size to choose the right setting that results in best performance or sometimes even to achieve a convergent behavior \cite{}.
\item It has been observed in experiments that the GTD is slow. While its variant GTD2 is faster than GTD, it is still slower compared to TD in the \onp setting. While, it is fine to draw understanding from experiments, it is however necessary to tie such observations to a theoretical phenomenon in a principled manner.
\item The GTD-MP is based on the SMP algorithm, which as \cite{} notes is especially effective when the original non-smooth function admits a smooth saddle-point representation. However, in the case of GTD the original as well as the saddle-point objective functions are, and hence the intuition why GTD-MP is effective in the GTD setting is absent.
\end{itemize}
Almost all the TD algorithms (except the LSTD) perform only $O(n)$ computations per time step and their updates are linear in $\theta$. Hence, we feel that it makes sense to study them (TD and GTD algorithms) under the framework of linear stochastic approximation (LSA) schemes. Acknowledging the linear nature of the updates helps us to obtain a qualitative and quantitative understanding of the performance of the various TD algorithms. In particular, the spectral nature of the matrices involved in the linear updates dictate the behavior of these algorithm. For example, it is well known that the divergence of TD in the \ofp setting can be directly attributed to the fact that the $A$ matrix cannot be ensured to have all eigen values with strictly positive real parts. Also, the LSA framework helps throws light on the fact that finite time behavior of these algorithms involves understanding products of random matrices (an aspect unnoticed in TD literature).\par
In this paper, we make the following specific contributions
\begin{itemize}[leftmargin=*]
\item \textbf{Towards TD:} We study the vanilla TD algorithm with constant step-size and Rupert-Polyak (CSRP) iterate averaging. Specifically, we study the behavior of $\theta_{t+1}=\theta_t+\alpha (b_t-A_t)$, where $b_t$ and $A_t$ are noisy samples of $b$ and $A$ (in \eqref{linsys}) respectively. The finite time error is split into two terms namely the bias term (due to the initial condition) and the variance term (due to noise). We show that CSRP results in a finite time performance of $O(1/t^2)$ and $O(1/t)$ respectively for the bias and the variance terms and the constant do not depend on the condition number of $A$.
\item \textbf{Towards general LSAs}: We show a weaker result for general linear iterates of the form $\theta_{t+1}=\theta_t+\alpha (g(b_t,A_t)-H(A_t)\theta_t)$, where $g$ and $H$ are appropriate functions. This formulation helps us to study other LSA algorithms such as GTD, GTD-MP, iLSTD. Even in this setting, finite time performance of $O(1/t^2)$ and $O(1/t)$ respectively for the bias and the variance terms holds, however the constants have dependence on the condition number of $H(A)$.
\item \textbf{Towards Step-sizes and Stability:} We turn towards theory of product of random matrices to comment about conditions on the constant step size that leads to convergent algorithms. In paritcular, we show that are two distinct regimes, one being conservative with a smaller constant step size resulting in convergence of iterates in the mean square sense, and the other being relaxed with a relatively larger constant step size guaranteeing only asymptotic stability in high probability.
\item \textbf{Towards GTD:} We show that the GTD does not admit convergence in the mean square sense, due to block zero entries in the leading diagonal of the update matrix, result in a sub-system that can diverge. We tie the Mirror-Prox idea to \emph{Predictor-Corrector} method in numerical analysis, which enables us to show that the GTD-MP eliminates the divergence issue by ensuring positive entries in the leading diagonal. This provides the much needed intuition that was lacking with respect to why the Mirror-Prox idea makes a difference in the case of GTD. Further, the results for general LSA schemes applies to GTD-MP, resulting in finite-time performance bounds for the mean squared error.
\item \textbf{Towards iLSTD:} The incremental version of LSTD algorithm (iLSTD) in \cite{} is also an LSA algorithm. It follows that iLSTD does not converge in the \ofp setting even when LSTD does. We propose a stable version of iLSTD which converges in the \ofp setting.
\item \textbf{Towards Open Issues:} We note that the GTD-MP algorithm is incorrect when the samples are obtained from a single trajectory.
\end{itemize}
The key take aways of the contributions are listed as under.
\begin{itemize}[leftmargin=*]
\item  The CSRP minimizes the overhead on tuning the step-sizes, in that it is enough to choose a constant step-size without bothering about the entire sequence of step-sizes. This constant step size can be found out by experiments.
\item The distinct regimes of step-size implies that observing a handful of convergent trajectories in practice might not mean that the algorithm is converging in mean squared sense.
\item The `slowness' of the GTD (and its variants) can be easily observed by looking at the spectrum of the update equation.
\end{itemize}
In addition to the contributions listed above, our aim is to stitch together relvant theoretical and practical aspects of the TD algorithm with an aim to gain an overall perspective of what makes them tick. We also present some experiments on simple domains to illustrate intended the message.
\comment{\subsection{Related Work}
The analsyis in this work is inspired by a related work in \cite{bachaistats}, in which the authors consider the problem of linear prediction with the penalty function as quadratic loss under an i.i.d  (with respect to some unknown distribution) assumption on the data. The linear prediction problem in \cite{bachaistats} is solved by an stochastic gradient descent (SGD) algorithm which is an LSA algorithm as well.  An important difference in our case is that unlike the linear prediction setting, the matrices involved in the TD updates are not symmetric. We nevertheless show that analysis on the lines of \cite{bachaistats} still holds for the LSAs in the RL setting, thereby presenting an analysis under more generalized assumptions i.e., in the absence of symmetry.
}

%\input{stab}
%\section{Implication of the Results}
\emph{Temporal Difference} (TD) learning algorithms \cite{} are widely used to learn value functions. A desirable feature of these algorithms that they are LSA algorithms that perform only $O(n)$ computations per time step. The vanilla TD (or simply TD) algorithm was first introduced in \cite{}. An unresolved issue for quite some time was the divergent behavior of TD in the \emph{off-policy} setting, where the sample were obtained from a policy different from the one whose value function needs to be computed. The Gradient Temporal Difference (GTD) learning algorithm \cite{} addressed the issue of divergence and are provably stable in \emph{off-policy} scenarios.
%Thus the GTD methods solve \eqref{linsys} in the general case when $\pi_t\neq\pi_b$.
%The \ofp convergence issue has also been addressed differently by the least-squares TD (LSTD) algorithm, which constructs $A$ and $b$ in \eqref{lsa}, and obtain $\theta^*=A^{-1}b$. As a consequence of the %matrix inversions involved, the LTSD performs a minimum of $O(n^2)$ computations, which can be a possible downside when $n$ is large.\par
Recently, newer variants of GTD namely, projected GTD2 and GTD-\emph{Mirror-Prox} were proposed in \cite{}. The authors in \cite{} observe that the GTD algorithm can be derived as a true stochastic gradient algorithms with repsect to a primal-dual saddle point objective function. This observation enables application of finite-time results for stochastic gradient (SG) algorithms to the GTD algorithm and at the same time, derive the newer variants based on the stochastic \emph{Mirror-Prox} algorithm \cite{}.\par
While several variants of TD algorithms have been developed, and have been empirically tested, as far as we gather, there are a number of poorly understood aspects of these algorithms, which we list as under.
\begin{itemize}[leftmargin=*]
\item Stochastic approximation \cite{SA} theory typically needs diminishing and square summable step-sizes. Further, SA theory deals with study of ordinary differnetial equations (ODEs) and lead only to asymptotic rates. The diminishing nature of the step-size schedules results in poor asymptotics, with the rates of convergence depending on the spectrum of the matrices involved in the SA updates. This is undesirable since the matrix (see \eqref{linsys}) involves quantities which $i$) are unkown and $ii$) vary across environment.
\item It is quite common in literature to sweep the step-size to choose the right setting that results in best performance or sometimes even to achieve a convergent behavior \cite{}.
\item It has been observed in experiments that the GTD is slow. While its variant GTD2 is faster than GTD, it is still slower compared to TD in the \onp setting. While, it is fine to draw understanding from experiments, it is however necessary to tie such observations to a theoretical phenomenon in a principled manner.
\item The GTD-MP is based on the SMP algorithm, which as \cite{} notes is especially effective when the original non-smooth function admits a smooth saddle-point representation. However, in the case of GTD the original as well as the saddle-point objective functions are, and hence the intuition why GTD-MP is effective in the GTD setting is absent.
\end{itemize}
Almost all the TD algorithms (except the LSTD) perform only $O(n)$ computations per time step and their updates are linear in $\theta$. Hence, we feel that it makes sense to study them (TD and GTD algorithms) under the framework of linear stochastic approximation (LSA) schemes. Acknowledging the linear nature of the updates helps us to obtain a qualitative and quantitative understanding of the performance of the various TD algorithms. In particular, the spectral nature of the matrices involved in the linear updates dictate the behavior of these algorithm. For example, it is well known that the divergence of TD in the \ofp setting can be directly attributed to the fact that the $A$ matrix cannot be ensured to have all eigen values with strictly positive real parts. Also, the LSA framework helps throws light on the fact that finite time behavior of these algorithms involves understanding products of random matrices (an aspect unnoticed in TD literature).\par
In this paper, we make the following specific contributions
\begin{itemize}[leftmargin=*]
\item \textbf{Towards TD:} We study the vanilla TD algorithm with constant step-size and Rupert-Polyak (CSRP) iterate averaging. Specifically, we study the behavior of $\theta_{t+1}=\theta_t+\alpha (b_t-A_t)$, where $b_t$ and $A_t$ are noisy samples of $b$ and $A$ (in \eqref{linsys}) respectively. The finite time error is split into two terms namely the bias term (due to the initial condition) and the variance term (due to noise). We show that CSRP results in a finite time performance of $O(1/t^2)$ and $O(1/t)$ respectively for the bias and the variance terms and the constant do not depend on the condition number of $A$.
\item \textbf{Towards general LSAs}: We show a weaker result for general linear iterates of the form $\theta_{t+1}=\theta_t+\alpha (g(b_t,A_t)-H(A_t)\theta_t)$, where $g$ and $H$ are appropriate functions. This formulation helps us to study other LSA algorithms such as GTD, GTD-MP, iLSTD. Even in this setting, finite time performance of $O(1/t^2)$ and $O(1/t)$ respectively for the bias and the variance terms holds, however the constants have dependence on the condition number of $H(A)$.
\item \textbf{Towards Step-sizes and Stability:} We turn towards theory of product of random matrices to comment about conditions on the constant step size that leads to convergent algorithms. In paritcular, we show that are two distinct regimes, one being conservative with a smaller constant step size resulting in convergence of iterates in the mean square sense, and the other being relaxed with a relatively larger constant step size guaranteeing only asymptotic stability in high probability.
\item \textbf{Towards GTD:} We show that the GTD does not admit convergence in the mean square sense, due to block zero entries in the leading diagonal of the update matrix, result in a sub-system that can diverge. We tie the Mirror-Prox idea to \emph{Predictor-Corrector} method in numerical analysis, which enables us to show that the GTD-MP eliminates the divergence issue by ensuring positive entries in the leading diagonal. This provides the much needed intuition that was lacking with respect to why the Mirror-Prox idea makes a difference in the case of GTD. Further, the results for general LSA schemes applies to GTD-MP, resulting in finite-time performance bounds for the mean squared error.
\item \textbf{Towards iLSTD:} The incremental version of LSTD algorithm (iLSTD) in \cite{} is also an LSA algorithm. It follows that iLSTD does not converge in the \ofp setting even when LSTD does. We propose a stable version of iLSTD which converges in the \ofp setting.
\item \textbf{Towards Open Issues:} We note that the GTD-MP algorithm is incorrect when the samples are obtained from a single trajectory.
\end{itemize}
The key take aways of the contributions are listed as under.
\begin{itemize}[leftmargin=*]
\item  The CSRP minimizes the overhead on tuning the step-sizes, in that it is enough to choose a constant step-size without bothering about the entire sequence of step-sizes. This constant step size can be found out by experiments.
\item The distinct regimes of step-size implies that observing a handful of convergent trajectories in practice might not mean that the algorithm is converging in mean squared sense.
\item The `slowness' of the GTD (and its variants) can be easily observed by looking at the spectrum of the update equation.
\end{itemize}
In addition to the contributions listed above, our aim is to stitch together relvant theoretical and practical aspects of the TD algorithm with an aim to gain an overall perspective of what makes them tick. We also present some experiments on simple domains to illustrate intended the message.
\comment{\subsection{Related Work}
The analsyis in this work is inspired by a related work in \cite{bachaistats}, in which the authors consider the problem of linear prediction with the penalty function as quadratic loss under an i.i.d  (with respect to some unknown distribution) assumption on the data. The linear prediction problem in \cite{bachaistats} is solved by an stochastic gradient descent (SGD) algorithm which is an LSA algorithm as well.  An important difference in our case is that unlike the linear prediction setting, the matrices involved in the TD updates are not symmetric. We nevertheless show that analysis on the lines of \cite{bachaistats} still holds for the LSAs in the RL setting, thereby presenting an analysis under more generalized assumptions i.e., in the absence of symmetry.
}

%\section{Asymptotic analysis: The ODE method}
We now review some of the results in stochastic approximation theory \cite{SA}. Consider the following LSA with diminishing step sizes given by
\begin{align}\label{lsadim}
\theta_t=\theta_{t-1}+\alpha_t(b_t-H_t\theta_{t-1}),
\end{align}
where $\alpha_t$ satifies \Cref{dimassmp}
\begin{assumption}\label{dimassmp}
$\us{\lim}{t\ra\infty }\alpha_t=0, \us{\sum}{t\geq 0}\alpha_t=\infty, \us{\sum}{t\geq 0}\alpha_t^2<\infty$
\end{assumption}
The results in this section hold for the AS case and diminishing step sizes (see \Cref{} ), and are based on the ordinary differential equation (ODE) method \cite{SA,Kush}. The ODE method concerns itself with asymptotic behaviour of \eqref{lsadim}.
The idea here is to associate the following ODE in \eqref{ode} with \eqref{lsadim}:
\begin{align}\label{ode}
\dot{\theta(t)}=g-H\theta(t),
\end{align}
where $t\in\R$ and $t\geq 0$. Note that time $t$ in \eqref{ode} is continuous and time $t$ in \eqref{lsadim} is integer valued. In order to maintain the disctinction between discrete and continuous times, we denote the former by $t_d$ and the latter by $t_c$.
The ODE method centres around the following transformation/interpretation of the `discrete' time in \eqref{ode} to/as the `continuous' time in \eqref{lsadim}.
\begin{align}\label{stepacc}
t_c(t_d)=\ous{\sum}{s=0}{t_d}\alpha_s,
\end{align}
i.e., the continuous time $t_c$ corresponding to the discrete time $t_d$ is merely the step sizes accumulated during the discrete time period from $0$ to $t_d$. The iterates of \eqref{lsadim} can then be used to define a continuous time trajectory $\hat{\theta(t)}$ as follows:
\begin{align}\label{inter}
\hat{\theta(t)}=\theta_{t_d}+(\theta_{t_d+1}-\theta_{t_d})\frac{t-t_c(t_d) }{t_c(t_d+1)-t_c(t_d)}, ~\forall~t\in[t_c(t_d),t_c(t_d+1)].
\end{align}
Note in \eqref{inter} the continuous time trajectory $\hat{\theta(t)}$ is obtained by \emph{interpolating} the iterates of \eqref{lsadim}.
We now state a result that shows the trajectory of the ODE \eqref{ode} and the interpolated trajectory in \eqref{inter} are `close' to each other asymptotically.
\begin{theorem}\label{sadim}
Let $\theta^s(t), t\geq s$ be the trajectory to the ODE \eqref{ode} with $\theta^s(s)=\hat{\theta(s)}$. It holds \emph{almost surely} that for any $T>0$
\begin{align}
\us{\lim}{s\ra\infty} \us{\sup}{t\in[s,s+T]}\norm{\hat{\theta(t)}- \theta^s(t)} = 0,
\end{align}
and the iterates $\theta_t \ra \ts$ as $t\ra\infty$
\end{theorem}
\begin{proof}
See Chapter~$2$, \cite{SA}.
\end{proof}
We can now comment on the asymptotic convergence rates that are expected to hold in lieu of the \Cref{track}.
\subsection{Asymptotic Convergence Rates}\label{initial}
We denote the spectrum of $H$ by $\{\mu_i,i=1,\ldots,n\}$. It is known from standard results that the trajectory $\theta(t)$ the ODE \eqref{ode} with initial condition $\theta(0)=\theta_0$ is given by
\begin{align}\label{oderate}
\theta(t)-\ts=\sum_{i=1}^n C_i e^{-\mu_i t},
\end{align}
where $C=(C_i,i=1,\ldots,n)\in \R^n$ are real coefficients. The time $t\geq 0$ in \eqref{oderate} is the continous time defined in \eqref{stepacc}. It is easy to see from \eqref{oderate} that terms in the summation are dependent on the Eigen values values and the accumulation of step sizes as given by \eqref{stepacc}. For instance, if the step-size are chosen to be $\alpha_t=C/t$, then
\begin{align}\label{biasforget}e^{-\mu_i\sum_{0\leq k<t}\alpha_t}\approx e^{-\mu_i Clog s}=O(1/s^{\mu_i C})\end{align}


%\subsection{Gradient Temporal Difference Learning}
The instability of TD($0$) is due to the fact that it is not a true gradient descent algorithm. The first gradient-TD (GTD) algorithm was proposed by \citet{sutton2009convergent} and is based on minimizing the \emph{norm of the expected TD update} (NEU) given by
\begin{align}\label{neu}
NEU(\theta)=\parallel b_\pi -A_\pi\theta\parallel^2
=\E[\rho_t\phi_t^\top\delta_t(\theta)]^\top\E[\rho_t\phi_t^\top\delta_t(\theta)]
\end{align}
The GTD scheme based is on the gradient of the above expression which is given by (dropping subscript $t$ for convenience) $-\frac{1}{2}\nabla NEU(\theta)=\E[\rho (\phi-\gamma\phi')^\top \phi]\E[\rho \phi^\top\delta(\theta)]$. Since the gradient is a product of two expectation we cannot use a sample product (due to the presence of correlations). The GTD addresses this issue by estimating $\E[\rho\delta\phi^\top]$ in a separate recursion. The GTD updates can be given by
\begin{align}
\begin{split}
\textbf{GTD:\quad}y_{t+1}&=y_t+\alpha_t(\rho_t\phi^\top\delta_t -y_t)\\
\theta_{t+1}&=\theta_t+\alpha_t\rho_t(\phi_t-\gamma\phi'_t)^\top\phi_ty_t
\end{split}
\end{align}
Notice that $y$ updates are noisy Euler discretization of the ODE $\dot{y}=\E[\rho\delta\phi^\top]-y(t)$. The overall design of GTD is given by $\D_{GTD}=\langle \begin{bmatrix}b_\pi\\ 0\\ \end{bmatrix},\begin{bmatrix}-I &-A^\mu_\pi \\ {A^\mu_\pi}^\top &0 \\ \end{bmatrix}\rangle$.\par
%$\{g_1,H_1\}$ where $g_1=\begin{bmatrix}b_\pi\\ 0\\ \end{bmatrix}$, $H_1=\begin{bmatrix}-I &-\Phi^\top D_\mu (\Phi -\gamma P_\pi\Phi) \\ (\Phi -\gamma P_\pi\Phi)^\top D_\mu\Phi &0 \\ \end{bmatrix}$.\par
Instead of NEU, the \emph{mean-square projected Bellman Error} (MSPBE) can also be minimized. The MSPBE is defined as
\begin{align}\label{mspbe}
MSPBE(\theta)=\parallel J_\theta-\Pi T_\pi J_\theta \parallel^2_D
\end{align}
The GTD2 algorithm was proposed in \cite{gtdref} based on minimizing \eqref{mspbe}. The GTD2 updates are given by
\begin{align}
\begin{split}
\textbf{GTD2:\quad}y_{t+1}&=y_t+\beta_t\phi_t^\top(\rho_t\delta_t-\phi_t y_t)\\
\theta_{t+1}&=\theta_t+\alpha_t\rho_t(\phi_t-\gamma\phi’_t)^\top\phi_t y_t
\end{split}
\end{align}
The design of GTD2 is given by $\D_{GTD2}=\langle \begin{bmatrix}b_\pi\\ 0\\ \end{bmatrix}, \begin{bmatrix}-M &-A^\mu_\pi \\ {A^\mu_\pi}^\top &0 \\ \end{bmatrix}\rangle$, where $M=\Phi^\top D_\mu\Phi$.\par
The GTD-\emph{Mirror Prox}(GTD-MP) algorithm is given by the following update rule:
\begin{align}\label{gtdmp}
\begin{split}
\textbf{GTD-MP:} y_t^m=y_t+\alpha_t{\phi_t}^\top(R(s_t,a_t)+\gamma (\phi'_t-\phi_t)\theta_t),\\ \theta_t^m=\theta_t+\alpha_t({\phi_t}-\gamma\phi'_t )^\top\phi_ty_t,\\
 y_{t+1}=y_t+\alpha_t{\phi_t}^\top(R(s_t,a_t)+\gamma(\phi'_t-\phi_t)\theta^m_t), \\ \theta_{t+1}=\theta_t+\alpha_t({\phi_t}-\gamma\phi'_t )^\top\phi_ty^m_t,
\end{split}
\end{align}
The GTD-MP algorithm in \eqref{gtdmp} is the PC discretization of the GTD algorithm with the design matrix as $\D_{GTD-MP}=\langle (I-\alpha_t H_{GTD})g_{GTD}, H_{GTD}-\alpha_t H^2_{GTD}\rangle$. In a similar fashion, one can derive the GTD2-MP algorithm as the PC discretization of GTD2 algorithm.\par

%\begin{abstract}
Linear stochastic approximation (LSA) algorithms arise quite naturally in various applications such linear least squares problem, solution to large scale linear systems and temporal difference learning algorithms. LSA schemes are often employed to solve for a desired parameter from noisy observations. In this paper, we are in the setting of LSA algorithms that employ a constant step size with iterate averaging under the presence of multiplicative noise and are interested in studying the mean-squared error (MSE) of the averaged iterates with respect to the desired solution. Our study is motivated by the recent results for an important sub-class of our setting namely linear least squares problem  wherein ``amazing'' properties such as instance independent choice for the step size and instance independent rate of convergence of the MSE have been demonstrated. In this paper, we ask the question whether these ``amazing" properties hold in the general setting that we consider.\par
We show that in the setting considered step sizes cannot be chosen in an instance independent fashion. Further, we show that while a rate of $O(\frac{1}{t})$ can be obtained for the MSE the constants are problem specific. Thus we observe that the ``amazing" properties that hold for the linear least squares problem do not hold in general. On the positive side, constant step size with iterate averaging is still an improvement over diminishing step size schemes which can yield only a $O(\frac{1}{t^\mu})$ ($\mu<1$) rate for the MSE.
    
\end{abstract}


%%!TEX root =  lsa.tex
\section{Problem Setup}\label{sec:prob}
%\begin{table*}[t]
\begin{center}
\resizebox{1\textwidth}{!}{
\begin{tabular}{|c|c|c|c|c|}\hline
Matrix   & $\alpha$             &   $C$     & Noise     &   $E[\norm{C\eb_t}^2]\leq \B^2+\V^2$\\\hline
AS       & $(0,\alpha_{as})$    &   $H$     & AUS        &   $\B^2=\Big(\frac{\norm{\theta_0-\ts}^2_2}{\alpha^2(t+1)^2}\Big)\big(1+2nB_1+n^2B_2\big),~\V^2=\Big(\frac{\sigma^2}{t+1}\Big)\big(1+2nB_1+n^2B_2\big)$\\    \hline
AS       & $(0,\alpha_{as})$    &   $I$     & AUS        &   $\B^2=\norm{H^{-1}}^2\Big(\frac{\norm{\theta_0-\ts}^2_2}{\alpha^2(t+1)^2}\Big)\big(1+2nB_1+n^2B_2\big),~\V^2=\norm{H^{-1}}^2\Big(\frac{\sigma^2}{t+1}\Big)\big(1+2nB_1+n^2B_2\big)$\\    \hline

PS       & $(0,\alpha_{ps})$    &   $H$     & AUS        &   $\B^2=\Big(\frac{\norm{\theta_0-\ts}^2_2}{\alpha^2(t+1)^2}\Big)\big(1+2(1-\alpha\frac{\rho_H}{2})^t+(1-\alpha\rho_H)^t\big),~\V^2=\Big(\frac{\sigma^2}{t+1}\Big)\big(1+5\alpha^{-1}{\rho_H}^{-1})$\\ \hline
PS       & $(0,\alpha_{ps})$    &   $I$     & AUS        &   $\B^2=\norm{H^{-1}}^2\Big(\frac{\norm{\theta_0-\ts}^2_2}{\alpha^2(t+1)^2}\Big)\big(1+2(1-\alpha\frac{\rho_H}{2})^t+(1-\alpha\rho_H)^t\big),~\V^2=\norm{H^{-1}}^2\Big(\frac{\sigma^2}{t+1}\Big)\big(1+5\alpha^{-1}{\rho_H}^{-1})$\\ \hline

SPDS    & $(0,\alpha_{ps})$     &   $H$     & AUS        &   $\B^2=\frac{\norm{\theta_0-\ts}^2_2}{\alpha^2(t+1)^2},~\V^2=\frac{\sigma^2}{t+1}$\\ \hline
SPDS    & $(0,\alpha_{ps})$     &   $H^{\frac{1}{2}}$     & US        &   $\B^2=\norm{H^{-1}}\frac{\norm{\theta_0-\ts}^2_2}{\alpha^2(t+1)^2},~\V^2=\norm{H^{-1}}\frac{\sigma^2}{t+1}$\\ \hline
SPDS    & $(0,\alpha_{ps})$     &   $H^{\frac{1}{2}}$     & US        &   $\B^2=\norm{H^{-1}}\frac{\norm{\theta_0-\ts}^2_2}{\alpha^2(t+1)^2},~\V^2=\norm{H^{-1}}\frac{\sigma^2}{t+1}$\\ \hline

SPDS    & $(0,\alpha_{ps})$     &   $I$     & AUS        &   $\B^2=\norm{H^{-1}}^2\frac{\norm{\theta_0-\ts}^2_2}{\alpha^2(t+1)^2},~\V^2=\norm{H^{-1}}^2\frac{\sigma^2}{t+1}$\\ \hline
SPDS    & $(0,\alpha_{ps})$     &   $I$     & AS        &   $\B^2=\norm{H^{-1}}^2\frac{\norm{\theta_0-\ts}^2_2}{\alpha^2(t+1)^2},~\V^2=\norm{H^{-1}}\frac{\sigma^2}{t+1}$\\ \hline
SPDS    & $(0,\alpha_{ps})$     &   $H^{\frac{1}{2}}$     &AS        &   $\B^2=\frac{\norm{H^{-\frac{1}{2}}(\theta_0-\ts)}^2_2}{\alpha^2(t+1)^2},~\V^2=\frac{\sigma^2}{t+1}$\\ \hline
PS       & $(0,\alpha_{ps})$    &   $I$     & MUS        &   $\B^2=\frac{(\rho_H\alpha)^{-1}(1+4(\rho_H\alpha)^{-1})}{(t+1)^2},~\V^2=(\sigma^2)\frac{\alpha^2(\alpha\rho_H)^{-1}(1+4(\rho_H\alpha)^{-1})}{t+1}$\\ \hline
SPDS    & $(0,\alpha_{ps})$     &   $H$     & MUS        &  $\B^2=\frac{\norm{\theta_0-\ts}^2_2}{\alpha^2(t+1)^2},~\V^2=\frac{\sigma^2}{t+1}$\\ \hline
SPDS    & $(0,\alpha_{ps})$     &   $H^{\frac{1}{2}}$     & MUS        &   $\B^2=\frac{(\rho_H)^{-1}(\alpha)^{-2}}{(t+1)^2},~\V^2=(\sigma^2)\frac{(\rho_H)^{-1}}{t+1}$\\ \hline
SPDS    & $(0,\alpha_{ps})$     &   $H^{\frac{1}{2}}$     & MS        &   $\B^2=\frac{\norm{H^{-\frac{1}{2}}(\theta_0-\ts)}^2_2}{\alpha^2(t+1)^2},~\V^2=\frac{\sigma^2}{t+1}$\\ \hline
\end{tabular}
}
\end{center}
\caption{Main Result}
\label{maintable}
\end{table*}

We consider constant step size linear stochastic approximation (LSA) algorithms of the form
\begin{align}\label{lsa}
\theta_{t}=\theta_{t-1}+\alpha(g_t-H_t\theta_{t-1}),
\end{align}
where $\theta_t\in \R^n$, $\alpha>0$ is a positive step-size, $g_t\in \R^n$ and $H_t\in \R^{n\times n}$. The Polyak-Rupert average of the iterates $\theta_t$ in \eqref{lsa} is defined as
\begin{align}\label{rp} \tb_t\eqdef \frac{1}{t+1}\ous{\sum}{s=0}{t} \theta_s. \end{align}
We are interested in the behaviour of \eqref{lsa} under the following conditions.
\begin{assumption}\label{genlsa}
\begin{enumerate}
\item\label{mart} $H_t\eqdef H+M_t$, $g_t\eqdef g+N_t$, where $M_t\in \R^{n\times n}$ and $N_t\in \R^n$ are zero mean $i.i.d$ sequences, i.e., $\E[N_t]=0,\E[M_t]=0,~\forall t\geq 0$.
\item \label{noise} The noise sequences satisfy $\E[N_t^\top N_t]\leq \sigma_1^2$ and $\E[M_t^\top M_t]\leq \Sigma_2^2$, where $\sigma_1^2>0$ is a positive scalar and $\Sigma^2_2$ is a real symmetric positive definite matrix.
\item \label{mat} $H$ is such that $\forall x\in \R^n$, $\ip{x,Hx}>0$ and let $\ts=H^{-1}g\in\R^n$. Further, $\sigma_2^2$ is a scalar such that ${\ts}^\top\Sigma_2^2\ts\leq \sigma_2^2$ and $\sigma^2\eqdef\sigma_1^2+\sigma_2^2$.
\end{enumerate}
\end{assumption}
Here, $\ts$ is a parameter to be estimated that the LSA in \eqref{lsa} aims to compute from noisy data presented in the form of $g_t$ and $H_t$.\par
Our motivation to study constant step size LSA with RP-averaging stems from the fact that LSAs are common in applications such as temporal difference learning algorithms \cite{td} in reinforcement learning (RL)\cite{rl}, solution to large scale linear systems \cite, and the linear least squares problem in \cite{bachharder}. In particular, constant step size and RP averaging has been shown to have `` amazing" properties in the case of linear least squares problem \cite{bachharder} and we would like to investigate further as to whether these properties generalize to LSA.
\begin{example}[Temporal Difference Learning]
The Temporal Difference learning algorithm with a constant step size $\alpha>0$ is given by
\begin{align}\label{tdzero}
\theta_t=&\theta_{t-1}+\alpha \big(\phi(s_t) R(s_t)\nn\\&+ (\phi(s_t)\phi^\top(s_t)-\gamma \phi(s_t)\phi^\top(s'_t))\theta_{t-1}\big),
\end{align}
where are $s_t,s'_t\in S$ with $S$ being a finite set, $\phi(s)\in \R^n$ are the so called \emph{feature} vectors, $\gamma\in (0,1)$ is a given discount factor and $R\colon S\ra \R$ is map from S to reals. Here, $s_t$ are $i.i.d$ random variables distributed according to the law $s_t\sim \xi$ where $\xi$ is supported on $S$ and $s'_t\sim p(s_t,\cdot)$. The TD algorithm in \eqref{tdzero} can be cast in the general form as presented in \eqref{lsa}, by letting $g_t=\phi(s_t)R(s_t)$ and $H_t=\phi(s_t)\phi^\top(s_t)-\gamma \phi(s_t)\phi^\top(s'_t)$.
\end{example}

\textbf{Linear Stochastic Error Recursion (LSER)} The error dynamics for the LSA in \eqref{lsa} i.e., the dynamics of $e_t\eqdef\theta_t-\ts$ can be written as follows:
\begin{align}\label{errprelim}
&\theta_t-\ts=\theta_{t-1}-\ts+\alpha(g_t -H_t(\theta_t-\ts+\ts)),\text{~or}\nn\\
&e_t=(I-\alpha H_t)e_{t-1}+(N_t-M_t\ts).
\end{align}
In what follows we consider what we call linear stochastic error recursion (LSER) given by
\begin{align}\label{lsergen}
e_t=(I-\alpha H_t)e_{t-1}+\alpha \zeta_t,
\end{align}
where $\zeta_t\eqdef (N_t-M_t\ts)$.
\begin{theorem}\label{maintheorem}
Define $\rho_\alpha\eqdef \us{sup}{\norm{x}\leq 1}\E[x^\top (2H_t-\alpha H_tH_t)x]$. Let $\alpha_{\max}$ be such that $~\forall \alpha\in(0,\alpha_{\max})$ $\rho_{\alpha}>0$. Then, it follows that
\begin{align}
\begin{split}
\MoveEqLeft\E[\norm{H(\tb_t-\ts)}^2]\leq \\
&(1+4\frac1{\alpha\rho_{\alpha}}) \frac{1}{\alpha\rho_{\alpha}} \Big(\frac{\norm{\theta_0-\ts}^2}{(t+1)^2}+\frac{\alpha^2\sigma^2}{t+1} \Big)
\end{split}
\end{align}
\end{theorem}
<<<<<<< Updated upstream
=======

>>>>>>> Stashed changes
\textbf{Discussion}
\begin{enumerate}[leftmargin=*]
\item \textbf{Bias and Variance:} The means-squared error at $t$ is bounded by a sum of two terms. The first term is the bias term given by $\B=(1+4(\alpha\rho_{\alpha})^{-1}) (\alpha\rho_{\alpha})^{-1} \Big(\frac{\norm{\theta_0-\ts}^2}{(t+1)^2}\Big)$.  The \emph{Bias} term that captures the rate at which the initial condition $\norm{\theta_t-\ts}^2$ is forgotten. The second term is the \emph{Variance} given by $\V=(1+4(\alpha\rho_{\alpha})^{-1}) (\alpha\rho_{\alpha})^{-1} \Big(\frac{\alpha^2\sigma^2}{t+1} \Big)$. The variance term that captures the rate at which noise is rejected. The $\B$ and $\V$ terms capture two different sources of errors. To see this, note that in the absence of noise i.e., when $\zeta_t=0,~\forall t\geq 0$, we have $\E[\norm{H\eb_t}^2]=\B$ and in the presence of noise, i.e., $\zeta\neq 0,\forall t\geq 0$ and the perfect initial condition, i.e., $\theta_t=\ts$, we have $\E[\norm{H\eb_t}^2]=\V$.
\item \textbf{Faster Rates:} The bias term term decays $O(1/t^2)$ and the variance term decays $O(1/t)$, which are faster than the rates that can be achieved by a diminishing step size sequence recommended by standard stochastic approximation theory \cite{sa}. In particular, when the step sizes are decaying $O(1/t)$ one can only achieve an rate that is $O(1/t^\mu)$ where $\mu$ is the smallest real part of the eigenvalue of $H$. Thus the choice of a constant step size sequence with iterate averaging on top helps us to eliminate the effect of `ill' conditioning of $H$.
\item \textbf{Problem Dependent terms:} The choice of the step size is unfortunately problem dependent. To see this, the condition that $\rho_{\alpha}>0$ in \Cref{maintheorem} ensures that the expected spectral norm of $\E[(I-\alpha H_t)^\top(I-\alpha H_t)]$ is less than unity. And thus the range of $\alpha$ changes from problem to problem, an issue we further elaborate in \Cref{sec:stepprob}.
\item \textbf{Behaviour for extreme value of $\alpha$:} For smaller values of step size, i.e., $\alpha\approx 0$, the bias term blows up, due to the presence of $\alpha^{-1}$ term. This is due to the fact that the step sizes determine the learning rate and for smaller step sizes the learning rate is slower. However, in this case the noise term does not blow up, a fact that can appreciated by looking at \eqref{lsergen} where $\alpha$ is seen to multiply the noise term $\zeta_t$. In quantitative terms, we can see that the $\alpha^{-2}$ and $\alpha^2$ terms can each other. For larger values of $\alpha$ i.e., $\alpha\ra \alpha_{\max}$, the bounds blow up again, due to the fact that $\rho_{\alpha}\ra 0$ in this case. This is due to the fact that the effect of both noise and initial conditions decay with the contraction factor of $F_{t,i}$, which gets closer to unity as $\alpha\ra\alpha_{\max}$.
\end{enumerate}

%\input{mainresults}
%\input{background}
%\input{ana}
%\input{relwork}
%\input{stab}
%\input{Discussion}
\if0
\begin{table*}[t]
\begin{center}
%\resizebox{1\textwidth}{!}{
\begin{tabular}{|c|c|c|c|c|}\hline
Matrix   & $\alpha$             &   $C$     & Noise     &   Theorem \\ \hline
AS       & $(0,\alpha_{as})$    &   $H$     & AUS        &   \Cref{asaus}\\    \hline

PS       & $(0,\alpha_{ps})$    &   $H$     & AUS        &   \Cref{psaus}\\ \hline

SPDS    & $(0,\alpha_{ps})$     &   $H$     & AUS        &   \Cref{spdsaus}\\ \hline
SPDS    & $(0,\alpha_{ps})$     &   $H^{\frac{1}{2}}$     &AS        &   \Cref{spdsas}\\ \hline

SPDS    & $(0,\alpha_{ps})$     &   $H^{\frac{1}{2}}$     & AUS        &   \Cref{spdsaushalf}\\ \hline

AS       & $(0,\alpha_{as})$    &   $I$     & AUS        &   \Cref{blanket}\\    \hline
PS       & $(0,\alpha_{ps})$    &   $I$     & AUS        &   \Cref{blanket}\\ \hline

SPDS    & $(0,\alpha_{ps})$     &   $I$     & AUS        &   \Cref{blanket}\\ \hline
SPDS    & $(0,\alpha_{ps})$     &   $I$     & AS        &   \Cref{blanket}\\ \hline
PS       & $(0,\alpha_{ps})$    &   $I$     & MUS        &   \Cref{psmus}\\ \hline
SPDS    & $(0,\alpha_{ps})$     &   $H$     & MUS        &  \Cref{spdsm}\\ \hline
SPDS    & $(0,\alpha_{ps})$     &   $H^{\frac{1}{2}}$     & MS        &   \Cref{spdsms}\\ \hline
SPDS    & $(0,\alpha_{ps})$     &   $H^{\frac{1}{2}}$     & MUS        &   \Cref{spdsmus}\\ \hline
\end{tabular}
%}
\end{center}
\caption{Main Result}
\label{maintable}
\end{table*}

\section{Supplementary Material}
\begin{align*}
\theta_{t+1}&=\theta_t+\alpha(b_t-A_t\theta_t)\\
\theta_{t+1}-\theta^*&=\theta_t-\theta^*+\alpha(b_t-A_t(\theta_t-\theta^*+\theta^*))\\
e_{t+1}&=(I-\alpha A_t) e_t+\alpha(M^{(1)}_{t+1} -M^{(2)}_{t+1}\theta^*)\\
\end{align*}
Let $F_{j,i}\eqdef\Pi_{t=i}^{j} (I-\alpha A_t), \forall j\geq i$ and $F_{j,i}\eqdef I,~\forall i<j$ and $M_{t+1}=M^{(1)}_{t+1}-M^{(2)}_{t+1}\theta^*$.
\begin{align}\label{exp}
\theta_t-\theta^*=F_{t,1}(\theta_0-\theta^*)+\alpha\sum_{k=1}^t F_{t,k+1} M_{k+1}
\end{align}
\begin{lemma}
Let $\parallel \cdot\parallel$ be the operator norm, then $\parallel (I-\alpha A)\parallel \leq 1-\alpha \rho$.
\end{lemma}
\begin{proof}
\begin{align*}
\parallel (I-\alpha A)\parallel&=\sup_{\theta\in \R^n | \parallel x\parallel=1}\theta^T (I-\alpha A) \theta\\
&=1-\alpha \inf_{\theta\in \R^n | \parallel x\parallel=1}\theta^T  A \theta\\
&=1-\alpha\rho
\end{align*}
\end{proof}
\begin{lemma}
For any $x_t\in \R^n$ that is $\F_t$ measurable and $\forall ~i > t$ if follows that $\E[x_t^\top F_{i,t+1}M_{t+1}]=0$
\end{lemma}
\small
\begin{proof}
\begin{align*}
&\E[x_t^\top F_{i,t+1}M_{t+1}]\\
%=\E\big[\E[ x_t^\top F_{i,t+1}M_{t+1}|\F_t]\big]\\
=&\E\Bigg[\E\bigg[\E\Big[\E\big[ x_t^\top F_{i,t+1}M_{t+1}|\F_{i}\big]|F_{i-1}\Big]\ldots|\F_{t}\bigg]\Bigg]\\
=&\E\Bigg[ x_t^\top \E\bigg[\E\Big[\E\big[(I-\alpha A_i)|\F_{i}\big]|\ldots F_{t+1}\Big]M_{t+1}|\F_{t}\bigg]\Bigg]\\
=&\E\Bigg[ x_t^\top \E\bigg[(I-\alpha A)^{t-i+1}M_{t+1}|\F_{t}\bigg]\Bigg]\\
=&\E\Bigg[ x_t^\top (I-\alpha A)^{t-i+1}\E\bigg[M_{t+1}|\F_{t}\bigg]\Bigg]\\
=&0
\end{align*}
\end{proof}
$\E\sum_{i=0}^t\sum_{j=0}^t<A(\theta_i-\theta^*),A(\theta_j-\theta^*)>=\E\sum_{i=0}^t<A(\theta_i-\theta^*),A(\theta_j-\theta^*)>+ 2\E\sum_{i=0}^t\sum_{j=i+1}^t<A(\theta_i-\theta^*),A(\theta_j-\theta^*)$. Now
\begin{align}\label{inter}
&\E\sum_{i=0}^t\sum_{j=i+1}^t<A(\theta_i-\theta^*),A(\theta_j-\theta^*)>\nn\\
&=\E\sum_{i=0}^t\sum_{j=i+1}^t<A(\theta_i-\theta^*),A\big[F_{j,i+1} (\theta_i-\theta^*)+\alpha\sum_{k=i+1}^j F_{j,k+1}M_{k+1}\big]>\nn\\
&=\E\sum_{i=0}^t\sum_{j=i+1}^t<A(\theta_i-\theta^*),A F_{j,i+1} (\theta_i-\theta^*)>\nn\\
&=\E\sum_{i=0}^t\sum_{j=i+1}^t<A(\theta_i-\theta^*),A(I-\alpha A)^{j-i} (\theta_i-\theta^*)>\nn\\
&=\E\sum_{i=0}^t<A(\theta_i-\theta^*),\alpha^{-1}AA^{-1}[(I-\alpha A)-(I-\alpha A)^{t-i+1}] (\theta_i-\theta^*)>\nn\\
&\leq\E\sum_{i=0}^t <A(\theta_i-\theta^*),\alpha^{-1}(I-\alpha A)(\theta_i-\theta^*)>+C_t\nn\\
&\leq\alpha^{-1}\E\sum_{i=0}^t<A(\theta_i-\theta^*),(\theta_i-\theta^*)>+ C_t
%&=\alpha^{-1}\E\sum_{i=0}^t<A(\theta_i-\theta^*),(\theta_i-\theta^*)>-\E\sum_{i=0}^t<A(\theta_i-\theta^*),A(\theta_i-\theta^*)>+ C_t\nn\\
\end{align}
\begin{lemma}
For any $x\in \R^n$, there exists $\ab$ such that $x^\top \E[(I-\alpha A_t)(I-\alpha A_t)|\F_t]x \leq x^\top(I-\alpha A)x, ~\forall \alpha\in (0,\ab)$.
\end{lemma}
\begin{proof}
\begin{align*}
&x^\top\E[(I-\alpha A_t)(I-\alpha A_t)|\F_t]x\\
&=x^\top\big(I-\alpha A -\alpha( A- \alpha\E[A_t^\top A_t|\F_t])\big)x\\
\end{align*}
\end{proof}
\begin{lemma}
Define for $x\in \R^n$ $f^i_t(x)\eqdef\E[<F_{t,i}x,F_{t,i}x>]$ and $g^i_t(x)\eqdef\E[<F_{t,i}x, A F_{t,i}x>]$. Then we have $g^i_{t-1}(x)\leq \alpha^{-1}(f^i_{t-1}(x)-f^i_t(x))$.
\end{lemma}
\begin{proof}
\begin{align*}
f^i_t=&\E\bigg[\E\Big[\E\big[ x^\top F^\top_{t,i} F_{t,i}x|\F_{t}\big]|F_{t-1}\Big]\ldots\bigg]\\
=&\E\bigg[x^\top\ldots\E\Big[(I-\alpha A_{t-1})^\top\E\big[ (I-\alpha A_t)^\top(I-\alpha A_t)|\F_{t}\big]\\&(I-\alpha A_{t-1})|F_{t-1}\Big]\ldots x\bigg]\\
&\leq f^i_{t-1}(x)-\alpha g^i_{t-1}(x)
\end{align*}
\end{proof}
\begin{lemma}
For $i\neq j$, $\E[<F_{t,i+1}M_{i+1},F_{t,j+1} M_{j+1}>]=0$
\end{lemma}
\begin{proof}
W.l.o.g suppose $i>j$
\begin{align}
&\E[<F_{t,i+1} M_{i+1},F_{t,j+1} M_{j+1}>]\nn\\
\label{diffindexinter}=&\E[ M^\top_{i+1}(I-\alpha A_{i+1})^\top\ldots(I-\alpha A_t)^\top(I-\alpha A_t)\ldots(I-\alpha A_{j+1}) M_{j+1}]\\
\label{diffindex}=&\E[ \sum_{l} G_l],
%&=\E\Bigg[\E\bigg[\E\Big[\E\big[<F_{i,k+1} M_{k+1},F_{i,k+1} M_{k+1}>|\F_{j-1}\big]|F_{j-2}\Big]\ldots|\F_{k+1}\bigg]\Bigg]\\
\end{align}
where $l$ is any index and $G_l$s in \eqref{diffindex} are terms obtained by expanding the product in \eqref{diffindexinter} using the fact that $A_t=A+M_{t+1},~\forall t\geq 0$. Note that $G_l$ is a product involving powers of $\alpha A$, vectors such as $M^\top_{q+1}$ and $M_{p+1}$ for some $i\geq p,q\geq j$. For any given $G_l$ and let $r(l)$ be the largest index such that either $M^\top_{r(l)+1}$ or $M_{r(l)+1}$ is present in that term.
\begin{align*}
\E[ \sum_{l} G_l]=  \sum_{l} \E\big[\E[G_l|\F_{r(l)}]\big]=0
\end{align*}
\end{proof}
\begin{align*}
\E<A(\theta_t-\theta^*),(\theta_t-\theta^*)>&=g^0_t(\theta_0-\theta^*)+\alpha\sum_{k=0}^t g^{k+1}_t(M_{k+1})\\
\sum_{i=0}^t\E<A(\theta_i-\theta^*),(\theta_i-\theta^*)>&=\sum_{i=0}^t g^0_i(\theta_0-\theta^*)+\alpha\sum_{k=0}^t g^{k+1}_i(M_{k+1})\\
&\leq\alpha^{-1}(f^0_0(\theta_0-\ts)-f^0_{t+1}(\theta_0-\ts))+\sum_{k=0}^t \alpha^{-1}(f^k_k(M_{k+1})-f^k_{t+1}(M_{k+1}))\\
&\leq\alpha^{-1}(f^0_0(\theta_0-\ts))+\alpha^{-1}(t+1)\sigma\sum_{k=0}^t \alpha^{-1}(f^k_k(M_{k+1}))\\
&\leq\alpha^{-1}\parallel \theta_0-\ts\parallel^2+\alpha^{-1}(t+1)\sigma^2
\end{align*}
\begin{lemma}
In \eqref{inter} $C_t\leq \frac{t+1}{\rho}{(1-\alpha\rho)^{t+1}}$
\end{lemma}
\begin{proof}
\begin{align*}
\E<A(\theta_i-\theta^*),(\theta_i-\theta^*)>&=g^0_i(\theta_0-\theta^*)+\alpha\sum_{k=0}^i g^{k+1}_i(M_{k+1})\\
\end{align*}
\end{proof}
\comment{
\begin{proof}
\begin{align*}
&<A(\theta_i-\theta^*),(I-\alpha A)^{t-i+1} (\theta_i-\theta^*)>\\
&\leq(1-\alpha \rho)^{t-i+1} <(\theta_t-\theta^*),A (\theta_t-\theta^*)>\\
\end{align*}
\begin{align*}
C_t&=\E\sum_{i=0}^t<A(\theta_i-\theta^*),(I-\alpha A)^{t-i+1} (\theta_i-\theta^*)>\\
&\leq\E\sum_{i=0}^t|<A(\theta_i-\theta^*),(I-\alpha A)^{t-i+1} (\theta_i-\theta^*)>|\\
&\leq \E\sum_{i=0}^t(1-\alpha\rho)^{t-i+1}(1-\alpha\rho)^i\\
&\leq \frac{t+1}{\rho}{(1-\alpha\rho)^{t+1}}\\
&\E<\theta_i-\theta^*,\theta_j-\theta^*>\\
&=\alpha^2\E\sum_{k=1}^i\sum_{j=1}^i <F_{i,j+1}M_{j+1},F_{i,k+1} M_{k+1}>\\
&=\alpha^2\E\sum_{k=1}^i<F_{i,k+1}M_{k+1},F_{i,k+1} M_{k+1}>\\
\end{align*}
\end{proof}
\begin{align*}
&\E[<F_{i,k+1}M_{k+1},F_{i,k+1} M_{k+1}>]\\
&=\E[\big[\Big[\bigg[<F_{i,k+1} M_{k+1},F_{i,k+1} M_{k+1}>|\F_{j-1}\bigg]|F_{j-2}\Big]\ldots|\F_{k+1}\big]]\\
&\leq(1-\alpha\rho)^{i-k-1}\sigma^2\\
&\E<\theta_i-\theta^*,\theta_j-\theta^*>\\
&\leq \frac{\alpha\sigma^2}{\rho}\\
&\E[<F_{i,1}(\theta_0-\theta^*),F_{i,1}(\theta_0-\theta^*)>]\\
&=\E[\bigg[\Big[\big[<F_{i,1}(\theta_0-\theta^*),F_{i,1}(\theta_0-\theta^*)>|\F_{j-1}\big]|F_{j-2}\Big]\ldots|\F_{k+1}\bigg]]\\
&\leq (1-\alpha\rho)^{i-k-1}\parallel \theta_0-\theta^*\parallel\\
&\leq (\rho\alpha)^{-1}\parallel \theta_0-\theta^*\parallel
\end{align*}
}

\fi
% Acknowledgements should only appear in the accepted version. 
%\section*{Acknowledgements}
\nocite{langley00}
\bibliography{ref}
\bibliographystyle{icml2016}
\end{document}
