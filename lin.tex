\section{LSAs to solve linear system of equations}
We now study three different LSAs based on distinct approaches to solve the linear system \eqref{linsys} given by:
\begin{align}\label{linsys}
A\theta=b.
\end{align}
We call them three algorithms, the fixed-point/least-squares/saddle-point linear stochastic approximation algorithms (FP/LS/SP-LSA) respectively. In what follows, we assume that
$A$ is invertible i.e., there exists $\ts=A^{-1} b$ and $A_t,A'_t, t\geq 0$ are sequences of $i.i.d$ samples of $A$ and $b_t,t\geq 0$ is a sequence of $i.i.d$ samples of $b$.\\
\textbf{FP-LSA}
\begin{align}\label{fplsa}
\theta_{t+1}=\theta_t+\alpha(b_t-A_t),
\end{align}
\textbf{LS-LSA}
\begin{align}\label{lslsa}
\theta_{t+1}=\theta_t+\alpha A^\top_t(b_t-A'_t\theta_t),
\end{align}
\comment{Further, let $M^{(1)}_{t+1}=A^{(1)}_t-A$, $M^{(2)}_{t+1}=A^{(2)}_t-A$ and $M^{(3)}_{t+1}=b_t-b$ be the martingale difference sequences satisfying \Cref{genlsa}-\eqref{mart}. The LSER corresponding to LS-LSA in \eqref{lslsa} is given by
\begin{align}\label{lslser}
e_{t+1}=&(I-\alpha A^{(1)\top}_tA^{(2)}_t)e_t+\alpha(A^\top M^{(3)}_{t+1}+M^{(1)\top}_{t+1}M^{(3)}_{t+1}\nn\\&-A^\top M^{(2)}_{t+1}\ts-M^{(1)\top}_{t+1}M^{(2)}_{t+1}\ts),
\end{align}
where $M^{(1)}_{t+1}=A^{(1)}_t-A$, $M^{(2)}_{t+1}=A^{(2)}_t-A$ and $M^{(3)}_{t+1}=b_t-b$. Choosing $H_t=A^{(1)\top}_tA^{(2)}_t$, $\zeta_t= A^\top M^{(3)}_{t+1}+M^{(1)\top}_{t+1}M^{(3)}_{t+1}-A^\top M^{(2)}_{t+1}\ts-M^{(1)\top}_{t+1}M^{(2)}_{t+1}\ts$ and $\sigma^2=\parallel A\parallel^2 (\sigma_2^2+\sigma_1^2\parallel \ts\parallel^2)+ \sigma_1^4(1+\parallel \ts\parallel^2)$ we can identify \eqref{lslser} with \eqref{lsergen}, and  the results of \Cref{spdsmn} applies to the LSER in \eqref{lslser}.
}
\textbf{SP-LSA}
\begin{align}\label{splsa}
\begin{split}
y_{t+1}&=y_t+\alpha (b_t-A_t\theta_t- M y_t)\\
\theta_{t+1}&=\theta_t+\alpha(A_t^\top)y_t
\end{split}
\end{align}
\textbf{Remarks:}
\begin{itemize}[leftmargin=*]
\item In addition to invertibility of $A$, the FP-LSA algorithm needs further assumption that $A$ is AS.
\item When it is not known whether $A$ is AS one has to resort to LS-LSA, where the idea is to solve $A^{\top} A\theta=A^{\top} b$. However a downside of the LS-LSA is that it requires \emph{double-sampling} i.e., $A'_t,A_t$ need to be independent, which cannot be ensured in many applications when the data is not available in a batch form.
\item The SP-LSA is based on the saddle point formulation. The algorithm is a stochastic gradient algorithm with respect to the saddle point objective given by
\begin{align}
L(\theta,y)=\min_{\theta\in \R^n}\max{y\in \R^n}<b-A\theta,y>-\frac{1}{2}\parallel y\parallel_M ^2
\end{align}
Notice that $\nabla_\theta L =-A^\top y$ and $\nabla_{y}L=-A\theta-M$.
\end{itemize}



\comment{
\subsection{Saddle-Point Linear Stochastic Approximation (SP-LSA)}
An important downside of \Cref{lslsa} is that two independent samples of $A$ are available in the form of $A^{(1)\top}_t$ and $A^{(2)}_t$ which might be possible in certain applications. This issue can be alleviated by looking at the following Saddle-Point objective function
\begin{align}
L(\theta,y)=\min_{\theta\in \R^n}\max{y\in \R^n}<b-A\theta,y>-\frac{1}{2}\parallel y\parallel_M ^2
\end{align}
By noticing the fact that $\nabla_\theta L =-A^\top y$ and $\nabla_{y}L=-A\theta-M$, we now specify the SP-LSA algorithm as follows:
The SLER corresponding to the SP-LSA in \eqref{splsa} is given by
\begin{align}
e_{t+1}=\Big(\begin{bmatrix} I & 0 \\ 0 &I\\\end{bmatrix} -\begin{bmatrix} M & A_t \\ -A_t^\top &0\\\end{bmatrix}\Big)e_t+\Big(\begin{bmatrix} M^{(2)}_{t_1}-M^{(1)}_{t+1}\ts  \\ 0\\\end{bmatrix}\Big),
\end{align}
where $M^{(1)}_{t+1}=A_t-A$, $M^{(2)_{t+1}}=b_t-b$,  and $\ts=A^{-1}b$. Now letting $H_t=\begin{bmatrix} M & A_t \\ -A_t^\top &0\\\end{bmatrix}$, we notice that $H=\E[H_t]=\begin{bmatrix} M & A \\ -A^\top &0\\\end{bmatrix}$, is not $PS$ and hence it is not straightforward to derive finite time bounds. However, by replacing $\alpha$ in \eqref{splsa} by $\alpha_t$ which satisfies \Cref{dimassmp}, the behaviour of the iterates can then be analysed using \Cref{sadim}.
}
\comment{
Let $x_t=(y_t,\theta_t)$, $g_t=\begin{bmatrix} b_t\\ 0\\\end{bmatrix}$ and $G^{(i)}_t=\begin{bmatrix} M & A^{(i)}_t \\ -A^{(i)\top}_t &0\\\end{bmatrix}$ for $i=1,2$, then consider the following updates
\begin{align}
x'_{t}&=x_t+\alpha (g_t-G^{(1)}_t x_t)\\
x_{t+1}&=x_t+\alpha(g_t- G^{(2)}_t x'_t);
\end{align}
}
