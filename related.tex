%!TEX root =  lsa.tex
%%%%%%%%%%%%%%%%%%%%%%%%%%%%%%%%%%%%%%%%
\section{Related Work}
\label{sec:related}
\comment{
We place our result into the context of related results.
Compare with papers by Bach. Can our result reproduce theirs?
State his result!
If not why not? 

Compare with Prashanth and Korda:
Only work known to us to bound finite-time error of LSA,
specifically for TD (special case of our setup,
as far as the form of the updates are concerned).

They use decreasing stepsizes -- they must use decreasing stepsizes
as discussed because they assume Markov noise.
Show their bound here (simplified maybe).
Comment on the main qualities of the bound.
Do they get the same rate?
Do they need the same conditions?
Do they depend on the same constants? No.. reciproc of minimum eigenvalue creeps in.
Why is this a problem? Reciproc of minimum eigenvalue is at least $d$ -- bad for huge-dimensional systems.
But it could be much smaller, too, when system is ill-conditioned.
}
The proof technique for \cref{maintheorem} has been adopted from Appendix $B$ of \cite{bachharder} . However, there are some critical differences which are listed below.
\begin{itemize}
\item Linear least squares problem is considered in \cite{bachharder} and the random matrices involved are known to be symmetric positive definite. This enables the authors in \cite{bachharder} to define operators from the space of symmetric matrices to the space of symmetric matrices and carry out the computations making use of such operators. In \cref{genlsa} the matrix is known only to be (laxly) positive definite, i.e., the lack of symmetry.
Thus we cannot define appropriate linear operators, and instead we have resort to an analysis that makes use of only the expected norms of the random matrices involved.
\item Another significant difference is the assumption of structured noise in \cite{bachharder}, which does not apply in our case, simply due to the fact that we don't have any symmetric matrices at all in our scheme of things.
\end{itemize}
