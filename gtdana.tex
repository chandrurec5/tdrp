\section{Price of Off-Policy Stability}
We discuss the price paid by GTD variants to achieve off-policy stability.
For the purpose of our discussions, we study the following modified version of GTD given by:
\begin{align}\label{gtdbeta}
\begin{split}
\textbf{GTD($\beta$)\quad}y_{t+1}&=y_t+\alpha \rho_t(b_t-A_t\theta_t - M y_t)\\
\theta_{t+1}&=\theta_t+\alpha\rho_t\beta(A^\top_ty_t)
\end{split}
\end{align}
We call the above variant GTD($\beta$).
In order to keep the discussion intuitive, we first consider a simple example MDP and then
\begin{example}\label{onestatemdp}
Consider the MPD with $1$ state, $1$ action, the probability transition $p=1$, reward $R=1$, discount factor $\gamma 1-\epsilon$ (for small $\epsilon>0$ ) and feature matrix $\Phi=1$. Here $A=1-\gamma=\epsilon$, $R=1$, $V^*=\frac{1}{\epsilon}$.
\end{example}
We now write down the TD and GTD($\beta$) algorithms for this $1$-state MDP in \Cref{onestatemdp}.
\begin{align}\label{tdonestate}
\begin{split}
\theta_{t}&=\theta_{t-1}+\alpha(1-\epsilon\theta_{t-1})\\
&=(1-\alpha\epsilon)\theta_{t-1}+\alpha 1
\end{split}
\end{align}
The Eigen value of the above TD update is given by $\mu=1-\alpha\epsilon$.
%The Eigen value decay is given by $\mu^t\approx e^{-\alpha\epsilon t}$
\begin{align}\label{gtdonestate}
\begin{bmatrix} y_t \\ \theta_t\\\end{bmatrix}=\Big(\begin{bmatrix} 1&0 \\ 0& 1\\\end{bmatrix} - \alpha\begin{bmatrix} 1&\epsilon \\ -\beta\epsilon& 0\\\end{bmatrix}\Big)\begin{bmatrix} y_t \\ \theta_t\\\end{bmatrix}+\alpha\begin{bmatrix} 1 \\ 0\\\end{bmatrix}
\end{align}
%The Eigen values of $H=\begin{bmatrix} 1-\alpha& -\alpha\epsilon \\ \alpha\beta\epsilon& 1\\\end{bmatrix}$ is given by:
\begin{align}
1-\frac{\alpha}{2}(1\pm\sqrt{1-4\beta\epsilon^2}),
\end{align}
and consider the various cases of $\beta$ as follows.\\
\comment{
\begin{table}
\resizebox{0.5\textwidth}{!}{
\begin{tabular}{|c|c|c|c|}\hline
&$0<\beta\leq1 $ &     $\beta=\frac{1}{4\epsilon^2}$  & $\beta>\frac{1}{4\epsilon^2}$ \\ \hline
$\begin{aligned} \text{Eigen}\\ \text{Value}\end{aligned}$ &$\begin{aligned} &\mu_1=1- \alpha(1-\beta\epsilon^2) \\ &\mu_2=1-\alpha(\beta\epsilon^2) \\ & \end{aligned}$ & $\begin{aligned}\mu_1=1- \frac{\alpha}{2} \\ &\mu_2=1-\frac{\alpha}{2} \\ \end{aligned}$ & $\begin{aligned}\mu_{1}=1-\frac{\alpha}{2}(1+ i\sqrt{4\beta\epsilon^2-1})\\ \mu_2=1-\frac{\alpha}{2}(1- i\sqrt{4\beta\epsilon^2-1})\end{aligned}$\\ \hline
$\begin{aligned} \text{Spectral}\\ \text{Radius}\end{aligned}$ &$\begin{aligned} 1-\alpha(\beta\epsilon^2)  \end{aligned}$ & $1-\frac{\alpha}{2}$ & $\begin{aligned} \sqrt{1-\frac{\alpha}{2}+\alpha^2\beta\epsilon^2}\end{aligned}$\\ \hline
\end{tabular}
}
\end{table}
}
\begin{table}
\resizebox{0.5\textwidth}{!}{
\begin{tabular}{|c|c|c|}\hline
$0<\beta\leq1 $ &     $\beta=\frac{1}{4\epsilon^2}$  & $\beta>\frac{1}{4\epsilon^2}$ \\ \hline
 $\begin{aligned} &\mu_1=1- \alpha(1-\beta\epsilon^2) \\ &\mu_2=1-\alpha(\beta\epsilon^2) \\ & \end{aligned}$ & $\begin{aligned}&\mu_1=1- \frac{\alpha}{2} \\ &\mu_2=1-\frac{\alpha}{2} \\ \end{aligned}$ & $\begin{aligned}\mu_{1}=1-\frac{\alpha}{2}(1+ i\sqrt{4\beta\epsilon^2-1})\\ \mu_2=1-\frac{\alpha}{2}(1- i\sqrt{4\beta\epsilon^2-1})\end{aligned}$\\ \hline
$\begin{aligned} 1-\alpha(\beta\epsilon^2)  \end{aligned}$ & $1-\frac{\alpha}{2}$ & $\begin{aligned} \sqrt{1-\frac{\alpha}{2}+\alpha^2\beta\epsilon^2}\end{aligned}$\\ \hline
\end{tabular}
}
\end{table}
\begin{enumerate}[leftmargin=*]
\item \textbf{Slowness of GTD:} The vanilla GTD algorithm ($\beta=1$) has been observed to be slow in comparison to the TD. The toy example captures this fact in a simple yet clear manner. To see this, observe that the spectral radius of the \eqref{gtdonstate} and \eqref{tdonestate} are $1-\alpha\epsilon^2$ and $1-\alpha\epsilon$ respectively. Thus the rate of convergence of the GTD  and TD in this toy example are $(1-\alpha\epsilon)^t\approx e^{-\alpha\epsilon^2 t}$ and $(1-\alpha\epsilon)^t\approx e^{-\alpha\epsilon t}$ respectively. Thus the performance of the GTD is off by a factor of $\epsilon$.
\item \textbf{Price of Stabiity} Notice that the GTD is stable for the off-policy setting as well. The GTD is two timescale algorithm with the dual variable $y$ estimates $(b-A\theta)$ and the primal updates involve $A^\top(y)=A^\top(b-A\theta)$. Thus the GTD implictly has $A^\top A$ in their update equation and hence it is not surprising that in the toy example $\epsilon^2$ plays a role. This effect continues even in higher dimensions (see \Cref{squared}). While the GTD is stable in all the settings its slowness seems to be the inevitable price that we need to pay.
\item The effect of the condition number can however be eliminated in the case of this toy example. By setting $\alpha=\frac{1}{\epslion}$ in \eqref{tdonstate} and $\alpha=2$, $\beta=\frac{1}{4\epsilon^2}$ in \eqref{gtdonestate}, we can eliminate the effect of $\epsilon$. However, in higher dimensions the GTD cannot be made as good as TD (see \Cref{}).
\end{enumerate}
\begin{lemma}\label{squared}
When $A$ has a small positive Eigen value say $0<\mu<<1$, then $1-\alpha\mu^2$ is an Eigen value of $F=\begin{bmatrix} (1-\alpha)I & -alpha A \\ \alpha A^\top& 0\\\end{bmatrix}$.
\end{lemma}
\begin{lemma}
When $\mu_{\min}>0$ and $\mu_{\max}>0$ are the smallest and largest Eigen values of $A$ respectively. Then the best possible spectral radius that can be achieved for the TD and GTD algorithms are given by $1-2\frac{\mu_{\min}}{\mu_{\max}}$ and $1-2\frac{\mu_{\min}^2}{\mu_{\max}^2}$ respectively.
\end{lemma}
